\documentclass[9pt,twoside,a4paper]{article}
\usepackage[utf8]{inputenc}
\usepackage[french]{babel}
\usepackage[T1]{fontenc}
\usepackage{geometry}
\geometry{a4paper, margin=1.35cm}
\usepackage{array}
%% \usepackage{amsfonts,amssymb,amsmath,amsthm}

%% \title{BaSIC : Le jeu de rôle de base}
%% \author{BaSIC \copyright Casus Belli/Multisim sous licence Chaosium}

\usepackage{multicol} % pour {\'e}crire dans certaines zones en colonnes : \begin{multicols}{nb colonnes}…\end{multicols}

% % % en-tete et pieds de page configurables : fancyhdr.sty
\usepackage{fancyhdr}
\pagestyle{fancy}
\fancyhf{}
\fancyhead[LO, RE]{  }
\fancyfoot[LE, RO]{ \hfill \textbf{\emph{BaSIC}} le jeu de rôle de base \textbf{Page \thepage} }
\renewcommand{\headrulewidth}{0.0pt}
\renewcommand{\footrulewidth}{0.0pt}
\fancypagestyle{plain}{
	\fancyhead{}
	\renewcommand{\headrulewidth}{0pt}
}

%% \renewcommand\section%
%% 		{\@startsection {section}{1}{\z@}%
%% 		{\hline}%
%% 		{\hline}%
%% 		{\reset@font\Large\bfseries}}

\usepackage{titlesec}	% in preamble
\titleformat{\section}
  { \centering \normalfont\Large\bfseries }	% format of title
  { \makebox[1pc][c]{\thesection\hspace{1pc}} } % label
  { 0pt }				% length of separation between label and title
  { \hrule }			% before-code hook
  [ \hrule ]
\titlespacing*{\section}{0pc}{36pt}{18pt}[10pc]

\begin{document}

\setlength\parindent{0pt} % \noindent for all document

%% \maketitle
\Large{\textbf{BaSIC : Le jeu de rôle de base}}~\\
\large{BaSIC \copyright Casus Belli/Multisim sous licence Chaosium} %% ~\\

\begin{multicols}{2}

\section*{INTRODUCTION}

S'il y a un système de jeu de rôle dont on connaît les qualités et les défauts, les possibilités et les limites, c'est bien le Basic Role Playing, le système de l'éditeur américain Chaosium. Il fut mis au point à l'aube du jeu de rôle par Steve Perrin et Lynn Willis pour le jeu RuneQuest, puis adapté pour les autres jeux de Chaosium (Elric, Hawkinson, L'appel de Cthulhu, Ellquest) et plus tard Multisim en acquit les droits en France pour son jeu Nephilim. BaSIC est la version la plus épurée du système Chaosium. Un système qui n'est pas forcément le plus réaliste, mais en tout cas le plus intuitif, le plus facile à intégrer, à jouer et à faire jouer, et le plus couramment pratiqué par un vaste nombre de joueurs !~\\

Cette version a été coordonnée par Tristan Lhomme. Introduction extraite d'un préambule de Didier Guiserix paru dans le hors-série Casus Belli n$^{\circ}$19 "BaSIC, le jeu de rôle de base" (1997).

\section*{CRÉER UN PERSONNAGE}

\subsection*{Définitions}

Un personnage est défini par deux séries de chiffres : ses caractéristiques et ses compétences. Les caractéristiques définissent ce qu'est le personnage, et les compétences ce qu'il sait. Votre personnage sera plus ou moins fort, plus ou moins intelligent, plus ou moins adroit, et cela dépend largement du hasard. En revanche, c'est vous qui choisirez si, au cours de sa vie, il a appris à se battre ou s'il a préféré se tourner vers l'érudition, la médecine…

\subsection*{Les caractéristiques}

Elles sont au nombre de sept. Pour un être normal, leur valeur varie entre 3 et 18. Les personnages des joueurs étant des héros, donc par définition un peu meilleurs que la moyenne, leurs caractéristiques iront de 8 à 18.~\\

\vfill~\columnbreak

\textbf{FORce} : C'est une mesure de la puissance musculaire du personnage. Cette caractéristique donne une idée de ce qu'il peut soulever, porter, pousser ou tirer. À titre indicatif, un personnage avec une FORce de 8 est un quidam moyen, qui n'a plus fait de sport après le lycée. Avec une FORce de 18, c'est un athlète qui joue dans la même cour qu'Arnold Schwarzenegger.~\\

\textbf{CONstitution} : Cette caractéristique rend compte de la vitalité et de la santé du personnage. Plus sa CONstitution est élevée, mieux le personnage résistera à la fatigue, aux coups, à la maladie, etc. Avec une CONstitution de 8, c'est un individu normal, peut-être un peu plus sujet aux rhumes que la moyenne. Avec une CONstitution de 18, il n'a probablement jamais été malade de sa vie.~\\

\textbf{TAIlle} : En fait, cette caractéristique englobe la taille et le poids. Être grand et massif vous permet de faire plus mal lors des combats, mais peut poser problème dans d'autres circonstances, par exemple lorsqu'il faut se faufiler par un passage étroit. Pour un personnage masculin, une TAIlle de 8 est légèrement inférieure à la moyenne (dans les 1,65 m pour 60 kg). Une TAIlle de 18 peut indiquer un géant très maigre, ou un individu de taille moyenne mais avec un embonpoint.~\\

\textbf{DEXtérité} : Comme son nom l'indique, cette caractéristique mesure la vivacité et la rapidité physique du personnage. Elle joue un rôle primordial dans les combats. En effet, les personnages dotés d'une DEXtérité élevée agissent plus souvent en premier. Avec une DEXtérité de 8, un personnage est très moyen, et sera bien avisé de ne pas se risquer, par exemple, dans une bagarre de café. Avec une DEXtérité de 18, il est extrêmement rapide et, potentiellement, pourrait jongler avec cinq couteaux et autant de torches enflammées.~\\

\textbf{APParence} : Votre personnage est-il beau ou pas ? S'habille-t-il avec élégance ? Sait-il faire bonne impression au premier contact ? L'APParence mesure tous ces facteurs. Un personnage doté d'une APParence de 8 est le genre d'individu que peu de gens remarquent au milieu d'une foule (ce qui peut, parfois, être un avantage). En revanche, avec une APParence de 18, il attirera l'attention partout où il passe.~\\

\textbf{INTelligence} : Elle décrit les capacités d'apprentissage, de mémorisation et d'analyse de votre personnage. Elle ne remplace pas l'intelligence du joueur, mais gère, à sa place, un certain nombre de tâches fastidieuses. Si votre personnage se retrouve en possession d'un message codé, par exemple, c'est son INTelligence qui servira à déterminer s'il parvient à le déchiffrer. Avec une INTelligence de 8, un personnage n'est pas très malin. À 18, c'est un génie.~\\

\textbf{POUvoir} : Cette caractéristique mesure la volonté du personnage, sa force d'âme et, dans certains univers de jeu, sa capacité à utiliser d'éventuels pouvoirs mentaux (magie ou pouvoirs psioniques). Il sert également à calculer sa Chance (voir plus bas). Avec un POUvoir de 8, votre personnage sera quelqu'un de relativement effacé, sans grandes dispositions pour le paranormal. Avec un POUvoir de 18, au contraire, il sera très impressionnant, et/ou aura un grand potentiel magique.~\\

\subsection*{Les compétences}

Les compétences sont des domaines de connaissance que votre personnage a eu l'occasion d'étudier au cours de sa vie. Elles sont différentes d'un univers de jeu à l'autre. Exemples: se servir d'une épée, parler une langue étrangère, conduire une automobile… Les compétences se mesurent sur 100. Elles sont détaillées au chapitre "Les compétences".

\section*{LA CRÉATION PAS À PAS}

\subsection*{Première étape : le héros}

Demandez au meneur de jeu quel "genre de héros" serait à l'aise dans l'histoire qu'il va vous faire jouer. Cela vous donnera une idée des personnages dont il a besoin. Si son scénario est une intrigue compliquée à la cour d'un roi, évitez de créer un clone de Conan le Barbare, il n'aurait rien à y faire !

\subsection*{Deuxième étape : les caractéristiques}

Munissez-vous d'une feuille de personnage, d'un bout de brouillon, d'un crayon et d'une gomme. Lancez sept fois de suite 2d6+6. Notez les sept résultats sur votre brouillon, puis répartissez-les à votre guise entre les caractéristiques. Toutes les caractéristiques sont utiles, mais elles ne servent pas toutes à la même chose. Si vous avez envie d'un personnage très physique, plutôt combattant, favorisez la FORce et la DEXtérité. Si vous avez plutôt envie de jouer un magicien, le POUvoir est plus important.

\subsection*{Troisième étape : les valeurs dérivées}

Les caractéristiques servent à calculer cinq valeurs dérivées. Suivez les instructions ci-dessous et reportez les résultats sur votre feuille de personnage.~\\

\textbf{Points de vie (PV)} : Ils mesurent l'état physique du personnage. Lorsqu'il est blessé, il perd des points de vie. S'il tombe à 0 point de vie, il est mort! Faites la somme de la TAIlle et de la CONstitution, divisez le résultat par deux (en arrondissant à l'entier supérieur, si besoin est).~\\

\textbf{Points d'énergie (PE)} : Ils servent, si l'univers de jeu le permet, à faire de la magie (ou à utiliser des pouvoirs psioniques). Votre capital de points d'énergie est égal à votre POUvoir.~\\

\textbf{Bonus aux dommages} : Plus un personnage est grand et fort, plus il fera mal lors des combats au corps à corps. Additionnez la TAIlle et la FORce de votre personnage, et reportez-vous à la table des bonus aux dommages. %% ~\\

\begin{center}
	\begin{tabular}{ c c }
		%% \multicolumn{2}{c}{\textbf{TABLE DES BONUS AUX DOMMAGES}}	\\
		\multicolumn{2}{c}{\textbf{TABLE DES BONUS}}	\\
		\multicolumn{2}{c}{\textbf{AUX DOMMAGES}}		\\
		\hline
		\textbf{FORce + TAIlle} & \textbf{Bonus} 		\\
		02 à 24 & Aucun \\
		25 à 32 & +1d3 \\
		33 à 40 & +1d6 \\
		41 à 60 & +2d6 \\
		\hline
	\end{tabular}
\end{center}~\\

\textbf{Le jet d'Idée} : Êtes-vous sûr qu'il soit opportun d'engager votre personnage dans ce tunnel empli de fumée rougeâtre, d'où sortent des ricanements démoniaques ? Quel est ce petit détail que vous avez oublié, et qui était visiblement d'une importance capitale pour le scénario ? Le jet d'Idée est un jet de dés qui, s'il est réussi, permet au meneur de jeu de vous souffler une information oubliée ou de vous suggérer un plan d'action, si vous êtes perdu. Contrairement à la plupart des valeurs figurant dans le cadre des caractéristiques, le jet d'Idée est noté sur 100, comme une compétence. Pour connaître sa valeur, multipliez l'INTelligence du personnage par 5 et reportez le résultat dans la case correspondante.~\\

\textbf{Le jet de Chance} : Votre personnage est en équilibre précaire au bord d'une falaise. Va-t-il tomber ? Cela dépend s'il a ou non de la chance. Cette valeur est souvent très utile pour simuler le "mouvement du monde" autour des personnages. Comme le jet d'Idée, le jet de Chance est un pourcentage, noté sur 100. Il se calcule en multipliant le POUvoir par 5.~\\

\subsection*{Quatrième étape : les compétences}

Regardez la liste des professions disponible dans votre univers de jeu et choisissez-en une. Vous avez 300 points à répartir dans les compétences qui en découlent (sans obligation de les prendre toutes). Ensuite, vous pourrez répartir 150 points parmi toutes les autres compétences de la feuille de personnage, à votre guise (même dans celles que vous avez déjà prises).~\\

Comment répartir ces points ? Il suffit de décider du nombre de points que vous allouez à une compétence, de l'ajouter à la base de cette dernière (la valeur entre parenthèses qui se trouve à côté de son nom sur la feuille de personnage). Ainsi, si vous décidez d'allouer 50 points à l'Esquive, qui a un score de base de 25, vous dépensez 50 points, et votre personnage se trouve doté d'une Esquive ayant un score de 75. 1 point signifie 1 \% de chance de réussite dans la compétence. A sa création, un personnage ne peut pas avoir plus de 90 dans un compétence, quelle qu'elle soit. Par ailleurs, il est inutile de le doter de scores supérieurs à 75 \%; un score aussi élevé signifie qu'il réussira une action trois fois sur quatre, ce qui devrait largement suffire… Il vaut mieux être un <<généraliste>> capable de faire moyennement bien un grand nombre de choses, qu'un "spécialiste" hyper compétent, ne sachant rien faire en dehors de son domaine de compétence très étroit. Si, après quelques parties, ce choix ne vous convient plus, créez un autre type de personnage.~\\

\emph{Conseil aux débutants : }C'est le moment de discuter avec les autres joueurs. Essayez de créer des personnages complémentaires, par exemple l'un sera bagarreur, l'autre plutôt diplomate; le groupe gagnera en efficacité.

\vfill~\columnbreak

\subsubsection*{Liste des professions avec compétences}

\textbf{Exemples pour un univers médiéval-fantastique}

\begin{itemize}
	\setlength{\parskip}{0pt}
	\setlength{\itemsep}{0pt plus 1pt}
    \item[$\bullet$] \textbf{Chasseur} : Athlétisme, Armes d'hast, Armes de jet, Artisanat, Connaissance de la nature, Orientation, Secourisme, Survie.
    \item[$\bullet$] \textbf{Chevalier} : Athlétisme, Equitation, Droit et usages, Armes de mêlée, Armes d'hast, Esquive, Vigilance, Lire et écrire.
    \item[$\bullet$] \textbf{Erudit} : Alchimie, Connaissance de la nature, Connaissance des peuples, Droit et usages, Légendes, Lire et écrire, Potions et herbes, Sagacité.
    \item[$\bullet$] \textbf{Fermier} : Armes d'hast, Armes de lancer, Artisanat, Commerce, Connaissance de la nature, Orientation, Potions et herbes, Secourisme.
    \item[$\bullet$] \textbf{Magicien} : Alchimie, Droit et usages, Légendes, Lire et écrire, Persuasion, Sagacité, Vigilance, 3 sortilèges au choix.
    \item[$\bullet$] \textbf{Soldat} : Athlétisme, Armes de mêlée, Armes d'hast, Armes de tir, Esquive, Discrétion, Vigilance, Secourisme.
    \item[$\bullet$] \textbf{Troubadour} : Artisanat (musique), Connaissance des peuples ou des religions, Culture générale, Droit et usages, Légendes, Lire et écrire, Persuasion, Sagacité.
    \item[$\bullet$] \textbf{Voleur} : Armes de lancer, Armes de mêlée, Cascade, Discrétion, Esquive, Persuasion, Sagacité, Serrurerie.
\end{itemize}~\\

\textbf{Exemples pour un univers contemporain}

\begin{itemize}
	\setlength{\parskip}{0pt}
	\setlength{\itemsep}{0pt plus 1pt}
    \item[$\bullet$] \textbf{Agent de renseignement} : Cascade, Déguisement, Droit et usages, Histoire et géographie, Informatique, Lutte, Renseignement, Sciences appliquées, Serrurerie.
    \item[$\bullet$] \textbf{Journaliste} : Bibliothèque, Conduire (auto ou moto), Connaissance de la rue, Culture générale, Langue étrangère, Persuasion, Sagacité, Vigilance.
    \item[$\bullet$] \textbf{Homme d'affaire} : Commerce, Comptabilité, Droit, Informatique, Langue étrangère, Persuasion, Sagacité, (plus une autre au choix).
    \item[$\bullet$] \textbf{Militaire} : Cascade, Conduire (au choix), Leadership, Navigation, Pilotage (au choix), Sabotage, Lutte, Armes de poing, Fusil.
    \item[$\bullet$] \textbf{Spécialiste en sciences humaines} : Bibliothèque, Culture générale, Droit et usages, Langue étrangère, Histoire et géographie, Orientation, Sciences sociales, Survie.
\end{itemize}~\\

%% \vfill~\columnbreak

\textbf{Cinquième étape : les finitions}

Le jeu de rôle est un moyen de raconter des histoires, comme les romans ou le cinéma, et les histoires sont toujours meilleures lorsqu'elles ont des héros crédibles. Pour l'instant, votre personnage est un tas de chiffres gribouillés au crayon sur un formulaire abscons. Pour lui insuffler un peu de vie, la première chose à faire est de remplir son "état civil".

$\bullet$ \textbf{La description} : Il existe une foule de détails qu'aucun jet de dés ne définira à votre place. Quelle est la couleur des yeux de votre personnage ? Celle de ses cheveux? Est-il droitier ou gaucher ? Comment s'habille-t-il ? A-t- il des expressions favorites ? Un accent ? Prenez un moment pour regarder comment ses caractéristiques interagissent. S'il a une TAIlle élevée et une faible FORce, par exemple, sa masse est probablement due à de la graisse, et pas à du muscle. En revanche, un individu avec une TAIlle faible, une bonne CONstitution et une DEXtérité élevée sera sûrement un petit bonhomme nerveux, avec des réflexes rapides.~\\
$\bullet$ \textbf{Le comportement} : Pensez à votre personnage comme à une vraie personne. A-t-il des goûts particuliers? Des choses dont il a horreur? Une passion pour les rousses ? L'habitude de commander un poulet rôti et de la tarte au citron dans toutes les auberges où il déjeune ? Essayez de trouver un ou deux détails qui le caractériseront, mais limitez-vous à cela. Sinon votre personnage risque d'être une collection de tics.~\\ 
$\bullet$ \textbf{L'histoire personnelle} : Maintenant que vous avez une assez bonne image de lui dans le présent, remontez un peu dans son passé (vous pouvez aussi partir de son passé et construire son apparence présente à partir de là. Les deux méthodes marchent aussi bien l'une que l'autre, l'essentiel est que vous vous sentiez à l'aise avec celle que vous choisirez). Où a-t-il grandi ? A-t-il encore ses parents ? Des frères et des soeurs ? Quelles sont ses relations avec eux ? Pourquoi ?~\\
$\bullet$ \textbf{La motivation} : Idéalement, il faudrait que votre personnage ait une bonne raison de se jeter dans l'aventure. Des motivations aussi passe-partout que "accumuler de l'or et de la gloire" ou "sauver le monde du méchant sorcier/démon/savant fou" fonctionnent toujours parfaitement, même si elles ont beaucoup servi, mais vous pourrez peut-être en trouver une autre dans le passé du personnage (du genre "retrouver sa soeur, enlevée par des extraterrestres il y a dix ans"). %%~\\

\section*{LES RÈGLES DE BASE}

\subsection*{Question de bon sens}

$\bullet$ \textbf{Première règle} : Une action impossible à rater réussit toujours.~\\
$\bullet$ \textbf{Deuxième règle} : Une action impossible à réussir échoue toujours.~\\
$\bullet$ \textbf{Troisième règle} : Pour tous les autres cas, faites un jet de compétence.~\\
$\bullet$ \textbf{Quatrième règle} : Lorsqu'il est impossible de faire un jet de compétence, utilisez un jets de caractéristique ou un jet d'opposition. Tout ce qui suit se contente de développer ces principes.~\\

\textbf{Action immanquables}~\\
Les actions physiques et intellectuelles tentées dans des situations banales réussissent toujours. Inutile de lancer les dés pour savoir si un personnage arrive à mettre un pied devant l'autre pour sortir de chez lui et aller acheter du pain !

\textbf{Actions impossibles}~\\
Les actions absurdes et irréalisables échouent toujours. Si un personnage saute d'un avion sans parachute et tente de rester en l'air en battant des bras, ce n'est pas la peine de chercher le point de règle qui lui permettra de s'en sortir: il va finir sa vie beaucoup plus bas, sous forme de flaque. Le meneur de jeu dispose d'une certaine latitude dans l'interprétation de ce qui est "absurde", et cela peut varier d'un univers à l'autre. Par exemple, dans le cas d'une aventure à la James Bond où le spectaculaire prime sur le réalisme, il est parfaitement possible de décider que notre aéronaute pourra, peut-être, avec un jet de Chance, s'accrocher à la queue de l'avion…

\subsection*{Jets de compétence}

La plupart des situations à résoudre se situent entre ces deux extrêmes, dans des circonstances où l'échec et la réussite sont tous deux possibles, et dont leurs résultats vont avoir des conséquences sur le reste de l'aventure. Pour simuler cette incertitude, on a recours à un jet de dés. Comme nous l'avons vu au chapitre précédent, les compétences sont mesurées sur 100. 
Un personnage avec 40 dans une compétence a en fait 40 \% de chance de réussir à l'utiliser. Pour utiliser une compétence, il faut lancer 1d100. Si le résultat obtenu est inférieur ou égal au score de la compétence, l'action est réussie (c'est pourquoi on dit faire un jet sous une compétence). Sinon, c'est un échec.~\\

%% \vfill~\columnbreak

\textbf{Maladresse.} Lorsque le résultat est compris entre 96 et 00, non seulement l'action est manquée, mais en plus, l'échec est particulièrement grave (par exemple: au cours d'une poursuite en voiture non seulement le poursuivant ne rattrape pas sa cible, mais en plus il tombe dans un fossé). Cela s'appelle une maladresse.~\\
\textbf{Réussite critique.} En revanche, avec un résultat compris entre 01 et 05, l'action est réussie de manière particulièrement brillante (non seulement vous rattrapez le poursuivi mais, en plus, vous avez évité le chien qui traversait la route). Cela s'appelle une réussite critique.~\\

\textbf{Bonus et malus}~\\
En règle générale, le jet de dés se fait "sous" la valeur de la compétence mais, parfois, le meneur de jeu peut décider que la situation mérite un ajustement dans un sens ou dans l'autre. Dans ce cas, il fixe un bonus ou un malus, qui vient s'ajouter (ou se soustraire) à la compétence du personnage. Le jet de dés est réussi s'il est inférieur ou égal à la valeur modifiée de la compétence. Le meneur de jeu peut fixer des bonus ou des malus supérieurs à + ou - 20 \%, mais ce n'est pas conseillé. Si l'action est vraiment très facile (ou très difficile), elle rentre probablement dans la catégorie des actions immanquables ou impossibles.

\begin{center}
	\begin{tabular}{ c c c }
		\multicolumn{3}{c}{\textbf{TABLE DE CIRCONSTANCES}}	\\
		\hline
		\textbf{Modif} & \textbf{Circonstance} & \textbf{Exemple} \\
		-20\% & Très difficile & A \\
		-10\% & Difficile & B \\
		+10\% & Facile & C \\
		+20\% & Plus que facile & D \\
		\hline
	\end{tabular}
\end{center}

A Tirer de nuit sur cible mobile, ou désamorcer une bombe qui va sauter d'une seconde à l'autre.~\\
B Tirer sur cible mobile ou de petite taille, crocheter une serrure particulièrement bien conçue.~\\
C Tirer sur cible immobile et de grande taille, escalader un mur avec de nombreuses prises.~\\
D Suivre un flâneur qui ne se méfie pas, se souvenir que la bastille a été prise un 14 juillet. %%~\\ 

\subsection*{Expérience}

Les sept caractéristiques ne peuvent pas augmenter, pas plus que leurs dérivées. Les scores de Chance ou Idée ne bougeront jamais, pas plus que le total des points de vie ou le bonus aux dommages. En revanche, les compétences peuvent augmenter, selon le principe: c'est en forgeant qu'on devient forgeron… Si, au cours d'une aventure, un personnage a réussi une action importante, le meneur de jeu peut l'autoriser à cocher la petite case qui se trouve à côté de la compétence correspondante.

À la fin de l'aventure, le personnage peut tenter de l'améliorer. Pour cela, le joueur lance ldl00. S'il obtient un score supérieur à la valeur de la compétence, le personnage a retiré quelque chose de l'expérience, et la compétence augmente d'ldl0 points. S'il obtient un score inférieur ou égal à la valeur de la compétence, la situation ne lui a rien appris… À partir de 90 \% dans une compétence, les règles changent un peu. Le personnage a une chance égale à son score d'INTelligence de s'améliorer et, même s'il réussit, il ne progressera que d'1 point. 

\section*{Astuces}

\subsection*{Les jets de caractéristique}

Il existe parfois des situations où aucune compétence ne s'applique. Supposons, par exemple, qu'un personnage (masculin) soit face à une charmante jeune fille, et qu'il désire qu'elle le remarque. Il n'existe aucune compétence Faire bonne impression aux jolies filles. En revanche, la caractéristique APParence est là pour ça ! Dans ce genre de cas, la caractéristique concernée est multipliée par un nombre (le multiplicateur) entre 1 et 5. Faites un jet d'1d100 sous cette valeur, comme si c'était une compétence. Le multiplicateur dépend de la difficulté de l'action: de x 5 (facile) à x 1 (très difficile), la plupart du temps ce sera x 3. Si la jeune fille de l'exemple est seule et cherche quelqu'un à qui parler, le jet d'APParence du personnage se fera sous son APP x 5. En revanche, si elle est avec son petit ami, le meneur de jeu demandera un jet sous l'APP x 1 (et peut décider que le petit ami manifeste sa jalousie, en cas de réussite !).

\subsection*{Les jets en opposition}

Parfois, vous aurez besoin de savoir ce qui se passe lorsqu'un personnage lutte contre quelque chose qui lui résiste. Arrivera-t-il à enfoncer cette porte ? Parviendra-t-il à se libérer de l'étreinte de la pieuvre géante ? La table de résistance est un outil multi-usage, qui sert à chaque fois qu'il est nécessaire d'opposer deux caractéristiques, une "active" (celle du personnage ou de la chose qui agit) et une "passive" (celle du personnage ou de la chose qui tente de résister à l'action). Il suffit de chercher dans cette table la colonne correspondant à la valeur de la caractéristique "active" et la ligne correspondant à la valeur de la caractéristique "passive".

À l'intersection, figure le nombre maximum qu'il faut obtenir avec 1d100 pour que l'action entreprise avec la caractéristique active soit réussie (exactement comme si c'était une compétence, à ceci près que les notions de réussite critique et de maladresse ne s'appliquent pas). Toutes les caractéristiques ou presque peuvent être opposées les unes aux autres. En général, cette table sert surtout aux duels FOR/FOR ou POU/POU, mais la pratique en fait découvrir beaucoup d'autres utilisations. Supposons par exemple que votre personnage veuille déchiffrer un message codé particulièrement compliqué. Le meneur de jeu peut décider d'opposer le score en INT du personnage à un nombre représentant la difficulté du document. Supposons que l'INT du personnage soit de 14, et que le meneur de jeu estime la difficulté du document à 18. Un coup d'oeil à la table montre que les chances de déchiffrer le document sont de 30 \%.

\end{multicols}

\begin{center}
\begin{tabular}{ c c c c c c c c c c c c c c c c c c c c c c }
		\multicolumn{21}{c}{\textbf{TABLE DE RÉSISTANCE}}		\\
		\hline
 		\multicolumn{21}{c}{\textbf{Caractéristique active}}	\\
		   & 01 & 02 & 03 & 04 & 05 & 06 & 07 & 08 & 09 & 10 & 11 & 12 & 13 & 14 & 15 & 16 & 17 & 18 & 19 & 20 & 21 \\
		01 & 50 & 55 & 60 & 65 & 70 & 75 & 80 & 85 & 90 & 95 & - & - & - & - & - & - & - & - & - & - & - \\
		02 & 45 & 50 & 55 & 60 & 65 & 70 & 75 & 80 & 85 & 90 & 95 & - & - & - & - & - & - & - & - & - & - \\
		03 & 40 & 45 & 50 & 55 & 60 & 65 & 70 & 75 & 80 & 85 & 90 & 95 & - & - & - & - & - & - & - & - & - \\
		04 & 35 & 40 & 45 & 50 & 55 & 60 & 65 & 70 & 75 & 80 & 85 & 90 & 95 & - & - & - & - & - & - & - & - \\
		05 & 30 & 35 & 40 & 45 & 50 & 55 & 60 & 65 & 70 & 75 & 80 & 85 & 90 & 95 & - & - & - & - & - & - & - \\
		06 & 25 & 30 & 35 & 40 & 45 & 50 & 55 & 60 & 65 & 70 & 75 & 80 & 85 & 90 & 95 & - & - & - & - & - & - \\
		07 & 20 & 25 & 30 & 35 & 40 & 45 & 50 & 55 & 60 & 65 & 70 & 75 & 80 & 85 & 90 & 95 & - & - & - & - & - \\
		08 & 15 & 20 & 25 & 30 & 35 & 40 & 45 & 50 & 55 & 60 & 65 & 70 & 75 & 80 & 85 & 90 & 95 & - & - & - & - \\
		09 & 10 & 15 & 20 & 25 & 30 & 35 & 40 & 45 & 50 & 55 & 60 & 65 & 70 & 75 & 80 & 85 & 90 & 95 & - & - & - \\
		10 & 05 & 10 & 15 & 20 & 25 & 30 & 35 & 40 & 45 & 50 & 55 & 60 & 65 & 70 & 75 & 80 & 85 & 90 & 95 & - & - \\
		11 & - & 05 & 10 & 15 & 20 & 25 & 30 & 35 & 40 & 45 & 50 & 55 & 60 & 65 & 70 & 75 & 80 & 85 & 90 & 95 & - \\
		12 & - & - & 05 & 10 & 15 & 20 & 25 & 30 & 35 & 40 & 45 & 50 & 55 & 60 & 65 & 70 & 75 & 80 & 85 & 90 & 95 \\
		13 & - & - & - & 05 & 10 & 15 & 20 & 25 & 30 & 35 & 40 & 45 & 50 & 55 & 60 & 65 & 70 & 75 & 80 & 85 & 90 \\
		14 & - & - & - & - & 05 & 10 & 15 & 20 & 25 & 30 & 35 & 40 & 45 & 50 & 55 & 60 & 65 & 70 & 75 & 80 & 85 \\
		15 & - & - & - & - & - & 05 & 10 & 15 & 20 & 25 & 30 & 35 & 40 & 45 & 50 & 55 & 60 & 65 & 70 & 75 & 80 \\
		16 & - & - & - & - & - & - & 05 & 10 & 15 & 20 & 25 & 30 & 35 & 40 & 45 & 50 & 55 & 60 & 65 & 70 & 75 \\
		17 & - & - & - & - & - & - & - & 05 & 10 & 15 & 20 & 25 & 30 & 35 & 40 & 45 & 50 & 55 & 60 & 65 & 70 \\
		18 & - & - & - & - & - & - & - & - & 05 & 10 & 15 & 20 & 25 & 30 & 35 & 40 & 45 & 50 & 55 & 60 & 65 \\
		19 & - & - & - & - & - & - & - & - & - & 05 & 10 & 15 & 20 & 25 & 30 & 35 & 40 & 45 & 50 & 55 & 60 \\
		20 & - & - & - & - & - & - & - & - & - & - & 05 & 10 & 15 & 20 & 25 & 30 & 35 & 40 & 45 & 50 & 55 \\
		21 & - & - & - & - & - & - & - & - & - & - & - & 05 & 10 & 15 & 20 & 25 & 30 & 35 & 40 & 45 & 50 \\
		\hline
	\end{tabular}
\end{center}

%% \clearpage

\begin{multicols}{2}

\subsection*{Un peu de calcul}

Si vous devez confronter des valeurs qui ne se trouvent pas sur la table de résistance, utilisez la formule suivante : Le pourcentage de chance de base est de 50 \% - (carac. passive x 5) + (carac. active x 5). Ainsi, pour une valeur passive de 53 et une valeur active de 45, la chance de base de l'action sera de 50 (53 x 5) + (45 x 5), soit 50 - 265 + 225, soit un malheureux 10 \%. La plupart du temps, vous aurez aussi vite fait de ramener les valeurs en cause à des nombres figurant sur la table, en y soustrayant un même nombre. Ainsi, dans l'exemple précédent, il suffit de soustraire 40 à 53 et 45, et de faire la confrontation avec 13 et 5.

\vfill~\columnbreak

\subsection*{Temps de jeu}

Le temps de jeu et le temps réel sont deux notions différentes, qui ont peu de rapport l'une avec l'autre.~\\
Un personnage peut passer plusieurs jours penché sur un grimoire ardu à déchiffrer, mais il ne faudra que quelques secondes au meneur de jeu pour résumer aux joueurs ce qu'il contient. Le temps de jeu est presque toujours une notion fluide. C'est au meneur de jeu, en tant que narrateur, de le faire avancer plus ou moins rapidement, en fonction des actions des joueurs et des besoins de son scénario. Pour utiliser une compétence, il faut de quelques secondes à quelques heures, en moyenne quelques minutes. Par exemple, un jet de Bibliothèque correspondra à plusieurs heures, alors qu'un jet de Cascade correspondra à quelques secondes.~\\
%% ~\\ %% \vfill~\columnbreak
Servez-vous de votre bon sens pour déterminer la durée de chaque tentative, si cela devait devenir important. C'est rarement le cas et, lorsque ça l'est, la véritable durée d'un jet de dés est "ce qui arrange le meneur de jeu pour que l'histoire reste intéressante et crédible". Il existe cependant un cas où le temps est géré de manière beaucoup plus rigide : le combat. Pendant un combat, le temps est découpé en rounds de quelques secondes, au cours desquels les actions s'enchaînent dans un ordre précis. Vous trouverez tous les détails au chapitre "Le combat".

\subsection*{Mouvement et poursuites}

Les personnages (et tous les autres êtres humains) se déplacent en moyenne de 8 mètres par round. C'est leur valeur de Mouvement. Lorsque que vous mettez en scène un combat entre humains, ne vous occupez pas de cette valeur, sauf pour rappeler aux personnages qu'ils ne peuvent pas accomplir des actions impossibles du genre courir jusqu'à leur voiture, prendre un fusil de chasse et revenir, le tout en un round.~\\
Les animaux et les monstres ont une valeur de Mouvement diférente des humains. Elle est là pour vous donner un ordre d'idée : un loup court nettement plus vite qu'un humain, alors que celui-ci n'a pas trop de mal à distancer un escargot.~\\
En cas de poursuite (des personnages par un monstre ou le contraire), estimez l'avance dont dispose le poursuivi. Soustrayez le mouvement du poursuivi de celui du poursuivant. Vous obtenez un nombre qui représente la distance en mètres que gagne le poursuivant à chaque round (si cette valeur est négative, le poursuivant gagne du terrain). La plupart des créatures ne peuvent pas courir à fond de train pendant plus que leur CON en round.~\\

\emph{Exemple : }Un personnage (Mouvement 8) est poursuivi par un cavalier (Mouvement du cheval : 12). Il a 20 mètres d'avance. Son poursuivant gagne 4 mètres par round, et le rattrapera en 5 rounds.

\subsection*{Blessures}

Les points de vie mesurent l'état physique du personnage. Lorsque ce compteur est à son maximum, le personnage est en pleine forme. Lorsqu'il est blessé, il subit des dommages autrement dit il perd des points de vie. Lorsqu'il blesse un adversaire, il lui inflige des dommages. Les points de dommages sont donc simplement les points de vie perdus. Ainsi, un personnage qui subit trois points de dommages perd trois points de vie.

%% \vfill~\columnbreak

$\bullet$ \textbf{Blessures graves} : Si un personnage perd en une seule fois la moitié du nombre actuel de ses points de vie, il est blessé grièvement. Faites un jet d'1d100 sous sa CON x 5 pour voir s'il supporte le choc. En cas de réussite, le personnage peut continuer à agir normalement. Si le jet est raté, le personnage perd connaissance pour un laps de temps égal à (21 - CON) en minutes. A son réveil, il sera très faible, et le restera jusqu'à ce qu'il ait été soigné.~\\
$\bullet$ \textbf{Agonie} : Un personnage qui se trouve à 1, 2 ou 3 points de vie doit faire un jet de CON x 3 par round. S'il le rate, il perd connaissance pour un laps de temps égal à (21 - CON) en heures. Il est plus que temps de le transporter à l'hôpital !~\\
$\bullet$ \textbf{Mort} : Un personnage qui tombe à 0 point de vie ou moins meurt instantanément. La médecine conventionnelle ne peut plus rien pour lui (en revanche, la magie reste une possibilité… parfois). La mort est définitive. Il n'y a plus qu'à créer un autre personnage, en souhaitant qu'il ait plus de chance que son prédécesseur. %%~\\

\subsection*{Causes de blessures}

En dehors du combat, qui est suffisamment important pour mériter un chapitre entier, de nombreuses autres causes peuvent faire perdre des points de vie aux personnages. Voici un petit échantillonnage des sources possibles de blessures. Notez que cette liste n'est pas limitative, même si elle devrait suffire à vous occuper pendant un bon moment (et à vous donner des idées !).

$\bullet$ \textbf{Chutes} : Retirez 1d6 points de vie par tranche de 3 mètres de chute. Un jet d'Athlétisme ou de Cascade réussi permet d'annuler la perte d'1d6 points. Notez qu'une chute vraiment importante est presque toujours mortelle pour un personnage. Mais en tant que meneur de jeu, essayez de ne pas en arriver là trop souvent (les jets de Chance sont bien utiles pour se raccrocher aux branches à la dernière seconde !).~\\
$\bullet$ \textbf{Feux} : 
    \begin{itemize}
		\item Si l'on s'en sert comme arme, une torche enflammée inflige 1d6 points de dommages par round. Un personnage touché par une torche doit faire un jet de Chance. S'il le rate, ses vêtements prennent feu, et il subit 1d6 points de dommages par round (voir cette notion dans le chapitre "Le combat"), jusqu'à ce qu'il ait réussi un nouveau jet de Chance (pour éteindre les flammes), ou qu'il se soit jeté à l'eau, ou qu'on l'ait enroulé dans une couverture, etc.
		\newline
		\item Tomber dans un feu de camp inflige 1d6 +2 points de dommages. En dehors de ça, suivez la même règle que pour les torches.
		\item Un personnage pris dans un incendie doit réussir un jet de Chance par round pour que ses vêtements ne s'enflamment pas. S'il le rate, il perd 1d6 points de vie par round. Par ailleurs, il risque de mourir asphyxié (voir plus bas).
		\item Un personnage qui a perdu plus de la moitié de ses points de vie à cause du feu perd également ld3 points d'APParence.
	\end{itemize}
$\bullet$ \textbf{Asphyxie/noyade} : Lorsqu'un personnage est exposé à des gaz toxiques, immergé dans un liquide, ou étranglé par un individu mal intentionné, on commence à découper le temps en rounds. Au premier round, le personnage doit réussir un jet de CON x l0 avec ldl00. Au deuxième round, le jet est sous CON x 9. Au troisième, de CON x 8, et ainsi de suite jusqu'au dixième round, où le jet se fait sous CON x l (pour les rounds suivants, on ne descend pas en dessous de la CON x l). Dès qu'un de ces jets est raté, le personnage subit 1d6 points de dommages. Il continuera à perdre 1d6 points de vie par round tant qu'il ne sera pas sorti de la zone dangereuse ou secouru.~\\ 
$\bullet$ \textbf{Explosions} : Dans les univers où ils existent, les explosifs infligent des dommages dans un certain rayon. De nos jours, une bombe artisanale fera 10d6 points de dommages là où elle explose, 9d6 de dommages dans un rayon de 3 mètres autour d'elle, 8d6 de dommages entre 3 et 6 mètres, 7d6 entre 6 et 9 mètres, et ainsi de suite. 10d6 de dommages sont une bonne base de travail, mais ce chiffre change selon la nature exacte de la bombe. Un bricolage artisanal avec de la poudre noire ne fera que 5d6 de dommages au point d'explosion; en revanche, un missile ou un obus moderne peut infliger jusqu'à 20d6 points de dommages à l'endroit où il frappe.~\\
$\bullet$ \textbf{Poison} : Les poisons en tout genre sont définis par une seule caractéristique : leur VIRulence, qui est généralement comprise entre 5 et 20. Si un personnage est en contact avec un poison (parce qu'il a été mordu par une vipère, qu'il a respiré des gaz toxiques ou mangé un plat assaisonné par quelqu'un qui lui veut du mal), opposez la VIR du produit à la CON du personnage sur la table de résistance. La VIR est la caractéristique active et, si elle triomphe de la CON, le personnage perd un nombre de points de vie égal à la VIR du poison. Si le personnage résiste, il perd quand même des points de vie (la moitié de la VIR, en général, parfois moins). Notez que la plupart des poisons n'agissent pas instantanément. Leur effet se déclare au bout d'1d6 rounds minimum, et les plus insidieux mettent des heures. Un jet réussi de Médecine (ou son équivalent, selon les univers) permet de diviser les dommages par deux, ou les supprimer complètement si le soigneur fait une réussite critique. Quelques exemples de VIRulences: arsenic 15, curare 20, venin de cobra 16, champignons vénéneux 6 à 15, somnifères modernes 13 (n'infligent pas de dommages, mais plongent dans un profond sommeil pour 2d6 heures).~\\
$\bullet$ \textbf{Maladie} : Les règles sur les maladies sont très proches de celles sur les poisons. Les maladies ont également une VIRulence, et s'attaquent également à la CONstitution. La principale différence est le rythme auquel elles infligent des dommages. Un personnage malade perd ld3 points de vie par jour, à concurrence de la VIRulence de la maladie. S'il a survécu, son organisme a triomphé des microbes, et il commence à récupérer les points de vie perdus. La compétence Secourisme n'a aucune incidence sur les maladies, pas plus que sur les empoisonnements. Quelques exemples de VIRulence: rhume 3 à 5, grippe 4 à 8, pneumonie 6 à 10, choléra 12 à 15, peste 14 à 20. %%~\\

\subsection*{Guérison}

Bien entendu, un personnage blessé finira par guérir. Ce n'est qu'une question de temps… Livré à lui-même, un personnage récupère 1d3 points de vie par semaine de temps de jeu. S'il reste au lit et ne se livre à aucune activité fatigante, il récupère 1d6 points de vie par semaine. Dans un hôpital moderne, ce chiffre monte à 2d3 points de vie par semaine. Les effets des compétences Secourisme et Médecine sont développés dans le chapitre sur les compétences. La seule règle importante sur la guérison est la suivante: en aucune circonstance il n'est possible de dépasser son total initial de points de vie.

\section*{LES COMPÉTENCES}

Les compétences représentent les domaines de connaissance que le personnage a appris. Elles peuvent varier d'un univers de jeu à l'autre et sont sujettes à amélioration. Dans la liste ci-dessous, la valeur entre parenthèses, à côté du nom de la compétence, correspond à son pourcentage de base, autrement dit la valeur "sous" laquelle vous faites un jet de dés si vous n'avez dépensé aucun point pour cette compétence lors de la création de personnage. Un personnage qui n'a rien investi en Athlétisme, par exemple, a quand même 15 \% de chances de réussir une action physique simple : effectuer un rétablissement, escalader un arbre, sauter un fossé… Les compétences dont le nom est suivi d'une astérisque (*) ont une particularité : c'est le meneur de jeu qui lance les dés à la place du joueur, de manière à ce que ce dernier ne voie pas le résultat. Par exemple, un personnage cherche un indice dans une maison cambriolée. Le MJ fait le jet de Chercher. Réussi, il annonce au personnage qu'il trouve quelque chose ; raté, il annonce que le personnage n'a rien trouvé (ce qui ne veut pas dire qu'il n'y avait rien à trouver!).

\subsection*{Liste universelle}

$\bullet$ \textbf{Art/Artisanat (05\%)} : Il s'agit en fait d'une famille de compétences, permettant de fabriquer des objets de première nécessité et/ou des œuvres d'art.~\\
$\bullet$ \textbf{Athlétisme (15\%)} : Cette compétence regroupe toutes les activités physiques : course, nage, saut, escalade.~\\
$\bullet$ \textbf{Bricolage (10\%)} : Le contenu de cette compétence change selon les univers de jeu.~\\
$\bullet$ \textbf{Cascade (10\%)} : Cette compétence combine l'agilité et la souplesse du personnage.~\\
$\bullet$ \textbf{Chercher* (20\%)} : On utilise Chercher pour fouiller un endroit.~\\
$\bullet$ \textbf{Culture générale (20\%)} : Cette compétence simule une connaissance superficielle d'un grand nombre de sujets historiques, culturels ou scientifiques.~\\
$\bullet$ \textbf{Commerce (20\%)} : La version médiévale permet de savoir où acheter et vendre quelles marchandises, et de marchander de manière efficace.~\\
$\bullet$ \textbf{Connaissance de la rue (10\%)} : Cette compétence permet de savoir où trouver des contacts : indics, tenanciers de bars louches, fabricants de faux papiers, etc.~\\
$\bullet$ \textbf{Déguisement (10\%)} : Grâce à Déguisement, un personnage peut modifier son apparence.~\\
$\bullet$ \textbf{Discrétion (15\%)} : Demandez un jet de Discrétion lorsque les personnages ont besoin de se déplacer silencieusement.~\\
$\bullet$ \textbf{Droit, administration, usages (10\%)} : Cette compétence est un mélange de politesse, de questions posées aux bonnes personnes et de connaissances livresques.~\\
$\bullet$ \textbf{Équitation (20\%)} : Cette compétence permet d'utiliser les animaux de monte.~\\
$\bullet$ \textbf{Esquiver (25\%)} : L'Esquive est une compétence précieuse en combat.~\\
$\bullet$ \textbf{Langue natale (80\%)} : Dans les univers où l'instruction est obligatoire, cette compétence recouvre la lecture, l'écriture et la communication orale.~\\
$\bullet$ \textbf{Langue étrangère (00\%)} : Chaque langue étrangère est une compétence distincte.~\\
$\bullet$ \textbf{Leadership (15\%)} : Leadership permet de commander un groupe, de l'organiser, de le motiver.~\\
$\bullet$ \textbf{Navigation (00\%)} : La Navigation est l'art de s'orienter sur l'eau, de manœuvrer un navire.~\\
$\bullet$ \textbf{Orientation* (15\%)} : L'Orientation sert à ne pas se perdre dans un environnement peu familier.~\\
$\bullet$ \textbf{Persuasion (15\%)} : Cette compétence sert à convaincre autrui du bien-fondé de ses arguments.~\\
$\bullet$ \textbf{Sagacité (20\%)} : Cette compétence permet au personnage qui l'utilise d'avoir une idée de l'humeur et des motivations d'un personnage non-joueur.~\\
$\bullet$ \textbf{Secourisme (30\%)} : Grâce à Secourisme, on peut ranimer les personnages inconscients, et surtout soigner les blessés.~\\
$\bullet$ \textbf{Survie (10\%)} : Faites faire des jets de Survie lorsqu'un personnage se trouve dans un environnement hostile.~\\
$\bullet$ \textbf{Vigilance (20\%)} : Utilisez cette compétence lorsqu'un personnage file un suspect, essaye d'écouter une conversation ou de remarquer un indice. %%~\\

\subsection*{Liste spécifique (fantastique-contemporain)}

$\bullet$ \textbf{Bibliothèque (25\%)} : Cette compétence permet d'utiliser une bibliothèque publique.~\\
$\bullet$ \textbf{Comptabilité (00\%)} : La compétence Comptabilité permet de se repérer dans les comptes d'un particulier ou d'une entreprise.~\\
$\bullet$ \textbf{Conduire… (20\%)} : La compétence Conduire est, en fait, triple.~\\
$\bullet$ \textbf{Histoire et géographie (10\%)} : Cette compétence fournit des informations dans ces deux domaines.~\\
$\bullet$ \textbf{Informatique (20\%)} : De nos jours, n'importe qui peut allumer un ordinateur et apprendre à se servir d'un logiciel.~\\
$\bullet$ \textbf{Médecine (00\%)} : Un jet réussi dans cette compétence permet de redonner 1d6 points de vie à un blessé.~\\
$\bullet$ \textbf{Paranormal (20\%)} : Cette compétence regroupe des informations sur une foule de sujets.~\\
$\bullet$ \textbf{Piloter… (00\%)} : Comme la conduite, le pilotage est en fait un groupe de compétences.~\\
$\bullet$ \textbf{Plongée (00\%)} : Cette compétence permet de savoir se débrouiller avec des bouteilles.~\\
$\bullet$ \textbf{Renseignements (20\%)} : Cette compétence concerne tout ce qui a trait au monde de l'espionnage et du crime organisé.~\\
$\bullet$ \textbf{Sabotage (05\%)} : Le Sabotage permet de manier des explosifs, fabriquer des cocktails Molotov.~\\
$\bullet$ \textbf{Science appliquée (00\%)} : Permet d'identifier et d'utiliser du matériel de haute technologie.~\\
$\bullet$ \textbf{Science pure (00\%)} : Cette compétence regroupe la plupart des sciences "dures".~\\
$\bullet$ \textbf{Sciences sociales (00\%)} : Cette compétence regroupe tout ce qui concerne les sciences humaines.~\\
$\bullet$ \textbf{Serrurerie (15\%)} : C'est l'art et la manière d'ouvrir une serrure sans laisser de traces. %%~\\

\subsection*{Liste spécifique (médiéval-fantastique)}

$\bullet$ \textbf{Alchimie (00\%)} : Cette compétence correspond à une connaissance primitive de la chimie.~\\
$\bullet$ \textbf{Art de la guerre (00\%)} : Cette compétence permet de dresser un plan de bataille.~\\
$\bullet$ \textbf{Civilisations anciennes (10\%)} : Cette compétence donne des informations sur les peuples qui ont précédé les cultures actuelles.~\\
$\bullet$ \textbf{Conduite d'attelage (15\%)} : L'art et la manière d'atteler un chariot, un carrosse.~\\
$\bullet$ \textbf{Connaissance de la nature (10\%)} : Cette compétence regroupe des connaissances de base sur les animaux, les plantes, la météo.~\\
$\bullet$ \textbf{Connaissance des peuples (20\%)} : La Connaissance des peuples donne des informations plus ou moins précises sur "l'étranger".~\\
$\bullet$ \textbf{Connaissance des religions (10\%)} : Comme la précédente, cette compétence fournit des informations, mais sur un domaine plus restreint.~\\
$\bullet$ \textbf{Démolition/sape (00\%)} : Démolition est l'art et la manière de creuser des galeries sous un rempart.~\\
$\bullet$ \textbf{Dressage (15\%)} : Le Dressage permet de mater un animal sauvage ou de l'apprivoiser.~\\
$\bullet$ \textbf{Héraldique (00\%)} : Cette compétence permet de reconnaître les blasons des familles nobles.~\\
$\bullet$ \textbf{Légendes* (05\%)} : La possession de cette compétence indique que le personnage connaît de nombreuses légendes.~\\
$\bullet$ \textbf{Lire et écrire (00\%)} : Dans les mondes où la plupart des gens n'ont pas accès à l'instruction, la lecture et l'écriture sont une compétence à part entière.~\\
$\bullet$ \textbf{Potions et herbes (10\%)} : En dehors de son nom, Potions et herbes est exactement identique à Médecine. %%~\\

\vfill~\columnbreak

\section*{LE COMBAT}

Les règles sur le combat peuvent paraître compliquées. Ellesreposent sur des principes simples, mais à la première lecture, vous risquez d'être noyé sous le flot de petits détails qui les rendent plus réalistes, mais moins intelligibles. Ne vous découragez pas, et relisez-les deux ou trois fois, en mettant en scène un simulacre de combat si besoin est. Une fois que vous aurez compris la structure du round, l'initiative et, pour les armes blanches, la notion de parade, vous aurez l'essentiel. Le reste viendra petit à petit, sur le tas, au fur et à mesure de vos besoins.

\subsection*{Round}

Le round sert à découper le combat de manière à ce qu'il soit jouable. C'est une unité de temps au cours de laquelle tous les participants du combat ont l'occasion d'agir au moins une fois. En temps de jeu sa durée est variable et correspond à peu près à une dizaine de secondes. En temps réel, il dure généralement quelques minutes. Chaque round se divise en plusieurs étapes :~\\
1 - Détermination de l'initiative ; ~\\
2 - Déclaration d'intention des PJ et des PNJ ; ~\\
3 - Résolution des actions, selon les règles détaillées au paragraphe "Ordre des attaques".~\\
Lorsque tout le monde a agi, le round suivant peut commencer, et l'on revient à l'étape 1.

\subsection*{Initiative}

Au début de chaque round, chaque participant au combat lance 1d6 et y ajoute la valeur de sa DEXtérité, puis annonce le résultat. Celui qui a obtenu le total le plus élevé agit en premier, puis c'est le tour du deuxième, du troisième, etc, jusqu'à celui qui a obtenu le plus mauvais total. Si deux joueurs ont obtenu le même total, ils lancent 1d6 pour se départager. Celui qui fait le meilleur score agit avant l'autre. Si un joueur et un personnage contrôlé par le meneur de jeu font le même total, c'est le joueur qui a l'initiative.

\subsection*{Déclaration d'intention}

Le meneur de jeu procède à un tour de table, demandant à chaque joueur ce qu'il compte faire pendant ce round. Les actions possibles sont nombreuses : attaquer, parer, esquiver, se déplacer…~\\
Inutile de donner des réponses trop détaillées. Il suffit d'annoncer quelque chose du genre "je tape sur le gros PNJ tatoué" ou "je me cache sous la table". En fait, se lancer dans des récits détaillés comme "je tape sur le gros tatoué et alors je le touche à la tête et je l'assomme et quand on le réveille il nous dit tout et on résout le scénario" fait perdre du temps à tout le monde car, de toute façon, ça ne se passera pas comme ça…~\\
De son côté, le meneur de jeu annonce aux joueurs ce que semblent vouloir faire les personnages qu'il contrôle. Là encore, il n'y a pas besoin d'être très précis. Dire "le magicien retrousse ses manches et commence à gesticuler" est préférable à "le magicien se prépare à lancer un sort de Boule de feu géante qui va tous vous carboniser". Les personnages seront peut-être très content que le magicien les laisse tranquille ce round-ci et ils se concentreront sur d'autres personnages non-joueurs… Au round suivant ils le regretteront amèrement, mais ils n'ont aucun moyen de le savoir. %% ~\\

\subsection*{Ordre des attaques}

À l'intérieur du round, on distingue trois passes, qui se succèdent toujours dans le même ordre. Parfois, personne ne pourra agir au cours d'une passe. Dans ce cas, ignorez-la et passez à la suivante.

$\bullet$ \textbf{Première passe :} les personnages (PJ et PNJ) qui ont des armes à feu ou des armes à projectiles prêtes à servir agissent en premier, par ordre décroissant d'initiative. Appuyer sur une gâchette ou lâcher la corde d'un arc prend nettement moins longtemps que de donner un coup de couteau...~\\
$\bullet$ \textbf{Deuxième passe :} une fois que ces premiers tirs ont eu lieu, on prend les personnages qui n'ont pas encore agi, toujours par ordre décroissant d'initiative. Ils peuvent attaquer, parer, se déplacer... Procédez aux jets de compétence appropriés au fur et à mesure.
$\bullet$ \textbf{Troisième passe :} les personnages qui ont des armes à feu pouvant tirer deux fois par round ont droit à leur deuxième tir, toujours par ordre d'initiative décroissante. Les personnages qui devaient dégainer leur arme ou encocher une flèche agissent à ce moment (dans le cas des armes à feu, ils perdent leur second tir).~\\
Cette troisième passe terminée, le round s'achève.

\subsection*{Esquiver et parer}

En plus de l'attaque et du déplacement, les personnages ont deux autres possibilités: l'esquive et la parade.~\\
$\bullet$ \textbf{L'Esquive} est une compétence à part entière. Lors des déclarations d'intention, un personnage qui désire esquiver une attaque l'annonce, et précise quel adversaire il souhaite esquiver. Un personnage qui esquive ne peut pas attaquer. En revanche, il peut parer l'attaque. L'Esquive peut également servir à rompre le combat: dans ce cas, le personnage fait un jet d'Esquive au début du round et, s'il le réussit, on considère qu'il s'est désengagé et n'est plus en danger (sauf, bien sûr, d'éventuels tireurs).~\\ 
$\bullet$ \textbf{La parade}, quant à elle, n'est pas vraiment une compétence, c'est seulement une autre manière d'utiliser les compétences d'attaque. Lors d'un corps à corps, un personnage qui est la cible d'une attaque peut parer avec son arme (épée, bâton ... ). Il fait un jet de compétence, comme pour une attaque. S'il est réussi, le coup est arrêté par l'arme, et le personnage ne subit pas de dommages. Il est possible de parer le coup d'un adversaire plus rapide que soi, puis d'attaquer le moment venu, ou le contraire. En revanche, on ne peut pas parer plusieurs fois par round. Attention : on ne peut ni parer, ni esquiver les balles.

\subsection*{Blesser l'adversaire}

Il suffit de réussir un jet sous la compétence qui régit l'arme utilisée. Si l'adversaire n'esquive pas ou rate sa parade, il est blessé et perd des points de vie. Pour savoir combien, on lance un certain nombre de dés, en fonction de l'arme. Chaque arme fait un certain nombre de dés de dommages (voir la table des armes). S'il s'agit d'une arme de corps à corps, on y ajoute le bonus aux dommages de l'attaquant. Bien sûr, s'il s'agit d'une arme à distance ce bonus ne joue pas. On lance les dés pour calculer le nombre de points de dommages infligés, on y soustrait l'éventuelle protection de l'adversaire, et on retire ce qui reste de ses points de vie.

\subsection*{Protections}

Un combattant désireux de vivre vieux ne se lance pas dans la mêlée sans protections. Celles-ci sont de trois types: les armures, les boucliers et le terrain lui-même.~\\
$\bullet$ \textbf{Les armures.} La nature des armures change d'un univers à l'autre (cotte de mailles, gilet pare-balles... ), mais leur fonction ne varie pas : elles absorbent une partie des dommages infligés par une attaque. Soustrayez le chiffre indiqué dans la table des armures aux points de dommages infligés au personnage. Seul l'éventuel excédent lui est infligé. Cette protection est efficace à chaque attaque et ne "s'use" jamais.~\\
$\bullet$ \textbf{Les boucliers.} Ils apportent un bonus au jet de parade, au prix d'un malus équivalent au jet d'attaque. Un grand écu ajuste le jet de compétence de +/-20 \%, un petit bouclier rond (ou un couvercle de poubelle) se contente de le modifier de +/-10 \%.~\\
$\bullet$ \textbf{Le terrain.} Un tireur qui s'abrite derrière un arbre, par exemple, est plus difficile à toucher qu'un homme qui se trouve à découvert. Le meneur de jeu peut donner un malus à la compétence (voir Bonus et malus, plus haut).

\subsection*{Réussite critique et maladresse en combat}

Pour presque toutes les compétences de combat, un résultat compris entre 01 et 05 signifie que les dommages infligés par l'attaque sont plus importants que prévu. Lancez deux fois les dés de dommages de l'arme, additionnez les résultats obtenus et soustrayez le résultat aux points de vie de la cible. De plus, les attaques portées avec des armes tranchantes (épée, dague, etc.) ou des armes à feu arrivent à trouver le défaut de l'armure. La protection de cette dernière est annulée. Les conséquences d'une maladresse sont potentiellement plus variées. En voici une petite liste non limitative :~\\
$\bullet$ \textbf{Bagarre :} la cible s'écarte à la dernière seconde. Le personnage fonce dans le mur et se blesse.~\\
$\bullet$ \textbf{Toutes les compétences d'armes :} l'arme se brise, elle reste coincée, le personnage perd l'équilibre (et ne pourra pas attaquer au prochain round), l'attaque touche un ami au lieu de la cible...~\\
$\bullet$ \textbf{Compétences d'armes à feu :} l'arme s'enraye, elle est inutilisable pour ld6 rounds, un ricochet touche un ami au lieu de la cible (dommages divisés par deux), les détonations attirent l'attention du voisinage qui prévient la police...~\\

\subsection*{Combat au corps à corps}

On peut distinguer deux cas : celui où les combattants ne sont pas armés, et celui où ils le sont.

$\bullet$ \textbf{Combat à mains nues :} Le combat à mains nues dépend de deux compétences : Bagarre et Lutte.~\\
$\bullet$ \emph{Bagarre.} La Bagarre consiste à cogner à coups de pied, de tête ou de poing, de manière instinctive. N'importe qui peut le faire avec des chances de succès raisonnables, mais ce n'est pas une attaque très efficace. Comme vous pouvez le voir sur la feuille de personnage, elle n'inflige qu'1d3 points de dommages (plus, bien sûr, le bonus aux dommages de l'attaquant). On ne peut utiliser cette compétence que pour parer une attaque de Bagarre, pour des raisons évidentes: essayez de parer un coup de hache avec votre main, vous m'en direz des nouvelles !~\\
Pourcentage de base: 50 \%.~\\
Rappel : Le pourcentage de base est le score d'un individu non entraîné (plus de précisions au chapitre Les compétences).~\\
$\bullet$ \emph{Lutte.} Les personnages qui ont investi dans la compétence
Lutte sont mieux entraînés que les simples bagarreurs, et savent comment être efficaces. Si, au cours d'un combat, le personnage réussit un jet de Bagarre qui est également inférieur à son score en Lutte, il inflige 2d3 points de dommages. La compétence Lutte a d'autres intérêts. Utilisée seule, elle permet d'immobiliser un adversaire (grâce à un jet FOR/FOR sur la table de résistance, à renouveler à chaque round); de le jeter à terre (réussite automatique); de l'étrangler (appliquez les règles sur l'asphyxie. La victime peut faire un jet de FOR/FOR à chaque round pour se dégager).~\\
Pourcentage de base: 20 \%.~\\

$\bullet$ \textbf{Combat aux armes blanches :} Le combat au corps à corps dépend de deux compétences : armes de mêlée et armes d'hast.~\\
Le combat au corps à corps dépend de deux compétences : armes de mêlée et armes d'hast. Toutes deux peuvent servir à parer.~\\
$\bullet$ \emph{Armes de mêlée} recouvre la plupart des armes tranchantes ou contondantes qui s'utilisent à une main. Cela va du gourdin à l'épée longue en passant par la dague ou la hache.~\\
Pourcentage de base: 25\%.~\\
Le fouet est un cas particulier, même s'il dépend aussi de cette compétence. Une attaque réussie avec un fouet n'inflige qu'1d3 points de dommages, mais la mèche du fouet est enroulée autour de l'adversaire. Il est possible de le faire tomber en réussissant un jet de FOR/TAI ou de le désarmer en réussissant un jet de DEX x3 \%. Si le résultat du jet d'attaque est compris entre 01 et 05, l'attaquant peut appliquer directement l'un de ces effets, sans avoir à faire de deuxième jet. La victime peut tenter de se dégager à chaque round avec un jet de FOR/FOR (elle arrache le fouet des mains de l'attaquant) ou de DEX/FOR (elle se dégage en souplesse).~\\
$\bullet$ \emph{Armes d'hast} concerne toutes les armes longues, qui s'utilisent généralement à deux mains, comme les hallebardes, les piques, les lances, etc.~\\
Pourcentage de base 20\%.~\\

~\columnbreak

\subsection*{Assommer un adversaire}

Pour assommer un adversaire, un personnage doit annoncer son intention en début de round, puis réussir une attaque de Bagarre, de Lutte ou d'une arme de mêlée non tranchante (un gourdin par exemple). Les dommages sont soustraits aus points de vie de la victime, puis on les oppose aux points de vie restants à cette dernière sur la table de résistance. Si le jet est un succès, la victime perd connaissance pour sa CON x heures. A son réveil, elle aura une grosse bosse, mais n'aura perdu qu'1/3 des points de vie de l'attaque.

\subsection*{Combat à distance}

Là encore, le combat à distance peut être divisé en deux grandes catégories : les armes de jet et les armes à feu.

$\bullet$ \textbf{Armes de jet :} Les armes de jet ne peuvent être utilisées qu'une fois par round. Lors du premier round de combat, elles ne tirent qu'à la troisième et dernière passe (le temps d'armer); aux rounds suivants, elles tirent lors de la première. Les deux compétences en armes de jet sont :~\\
$\bullet$ \emph{Les armes de tir.} Cette compétence permet d'utiliser arcs, arbalètes, frondes, sarbacanes... Bref, toutes les armes "balistiques".~\\
Pourcentage de base: 25 \%.~\\
$\bullet$ \emph{Les armes de lancer.} Cette compétence concerne les couteaux de lancer, les javelots, mais aussi les armes improvisées, comme un bête caillou ramassé au hasard d'un chemin.~\\
Pourcentage de base: 20 \%.~\\
$\bullet$ \textbf{Armes à feu :} Comme les armes de jet, les armes à feu tirent à la première ou à la troisième passe de chaque round. Mais contrairement aux arcs et autres arbalètes, la plupart sont assez rapides pour être utilisées à la première et à la troisième passe du round. Contrairement à la plupart des autres types d'armes, les armes à feu recouvrent un grand nombre de compétences. Leurs noms mêmes expliquent assez clairement à quoi elles correspondent.~\\
$\bullet$ \emph{Armes de poing} permet d'utiliser les pistolets et les revolvers, quel que soit leur calibre. Il est impossible de parer avec une arme de poing.~\\
Pourcentage de base: 20 \%.~\\
$\bullet$ \emph{Fusils} est utilisé pour tout ce qui est carabines ou fusils de guerre non automatiques.~\\
Pourcentage de base: 15 \%.~\\
$\bullet$ \emph{Fusils de chasse} sert pour les armes qui tirent des plombs ou des chevrotines, et non des balles. Particularité : les dommages dépendent de la distance; ces armes sont terribles à courte portée, et beaucoup moins efficaces si la cible est loin. Ce sont les seules armes à feu pour lesquelles la compétence de tir n'est pas doublée à bout portant. De plus, elles ne doublent pas les dommages sur un succès critique.~\\
Pourcentagede base: 20 \%.~\\
$\bullet$ \emph{Mitraillettes} regroupe toutes les armes automatiques capables de tirer en rafale, que ce soit les mitrailleuses lourdes ou les Uzis.~\\
Pourcentage de base: 10 \%.

\subsection*{Tir en rafale}

Le tir en rafale a ses propres règles. On ne peut tirer une rafale qu'à la troisième passe d'un round. Une "rafale" comprend de 6 à 20 balles. Chaque balle tirée ajoute 5 % à la compétence du tireur, à concurrence du double de sa compétence. Si le jet sous la compétence est réussi, lancez un dé ayant autant de faces qu'il y a de balles dans la rafale, soit ld6 pour une rafale de 6 balles, 1d10 pour une rafale de 10 balles, 2d10 pour une rafale de 20 balles...

Le résultat indique le nombre de balles qui touchent leur cible. 

Calculez ensuite les dommages infligés à la cible, balle par balle, en soustrayant l'armure à chaque fois. Si le résultat du jet d'attaque était compris entre 01 et 05, la première balle inflige double dommage. Si le tireur vise plusieurs cibles, ses chances toucher ne sont pas modifiées. Une fois la rafale tirée, on lance le ou les dés pour savoir combien de balles ont touché, et le tireur les répartit entre les différentes cibles. Si le résultat du jet d'attaque était compris entre 01 et 05, l'une des balles inflige double dommage, et le tireur décide alors quelle cible est concernée.

\subsection*{Recharger}

La plupart des armes modernes sont des automatiques, avec des chargeurs. Il faut un round entier pour expulser le chargeur vide et le remplacer par un neuf. Au round suivant, le personnage tirera une fois, à la troisième passe, puis tout rentrera dans l'ordre. Dans le cas des revolvers, il faut un round entier pour glisser deux balles dans le barillet.

\subsection*{Tirer à bout portant}

Lorsque le tireur est très proche de la cible, ses chances de la toucher sont doublées. Le "bout portant" est une distance égale à la DEXtérité du tireur exprimée en mètres. Ce bonus ne vaut que pour la compétence, pas pour les dommages. Cette règle ne s'applique pas au tir en rafale. Si un personnage vide un chargeur de mitraillette à bout portant sur quelqu'un, ce n'est pas vraiment la peine de lancer les dés. Considérez que ce quelqu'un est mort, ou bon pour un très long séjour à l'hôpital.

\end{multicols}

%% \clearpage

\begin{center}
	\begin{tabular}{ l c }
		\multicolumn{2}{c}{\textbf{TABLE DES ARMURES}}	\\
		\hline
		\textbf{Type}							& \textbf{Protection} \\
		%% \hline
		Cuir souple (blouson) 					& 1 \\
		Cuir rigide (armure) 					& 2 \\
		Cuir et métal 							& 4 \\
		Cotte de mailles (gilet pare-balles) 	& 6 \\
		Armure de plaques* 						& 8 \\
		\hline
	\end{tabular}
\end{center}

* Ce type d'armure, très lourde, inflige un malus de 10% aux compétences de combat et ne permet pas d'Esquiver.

\begin{center}
	\begin{tabular}{ l l c }
		\multicolumn{3}{c}{\textbf{TABLE DES ARMES D'HAST ET DE MÊLÉE}}	\\
		\hline
		\textbf{Arme} 							& \textbf{Compétence} 	& \textbf{Dommages} \\
		%% \hline
		Dague / poignard 				& Mêlée 		& 1d3+2 \\
		Épée / rapière 					& Mêlée 		& 1d6+2 \\
		Épieu / pique 					& Hast 			& 2d6 \\
		Espadon* 						& Mêlée 		& 2d6 \\
		Gourdin 						& Mêlée 		& 1d6 \\
		Hache de bataille 				& Mêlée 		& 2d6 \\
		Hachette						& Mêlée 		& 1d6+1 \\
		Hallebarde* 					& Hast 			& 3d6 \\
		Javelot / lance courte 			& Hast 			& 1d6+1 \\
		Masse / fléau d'armes 			& Mêlée 		& 1d6+2 \\
		Coup de poing, de tête, etc. 	& Bagarre 		& 1d3 \\
		Bouclier 						& Celle de l'arme principale 	& - \\
		\hline
	\end{tabular}
\end{center}

* Ces armes s'utilisent à deux mains et sont trop encombrantes pour que l'on puisse parer efficacement avec.

\begin{center}
	\begin{tabular}{ l l c c c }
		\multicolumn{5}{c}{\textbf{TABLE DES ARMES D'HAST ET DE MÊLÉE}}	\\
		\hline
		\textbf{Arme} 		& \textbf{Compétence}	& \textbf{Portée efficace} 	& \textbf{Portée maximale} 	& \textbf{Dommages} \\
		%% \hline
		Arbalète 	& Tir 			& 20 m 				& 50 m 				& 2d6 \\
		Arc 		& Tir 			& 50 m 				& 150 m 			& 1d6+2 \\
		Dague 		& Lancer 		& FOR en m 			& FOR x 2 m 		& 1d3+2 \\
		Fronde 		& Tir 			& 50 m 				& 100 m 			& 1d6 \\
		Javelot 	& Lancer 		& FOR en m 			& FOR x 3 m 		& 1d6+1 \\
		\hline
	\end{tabular}
\end{center}

* 1 tir par round à la 3e passe.

\clearpage

\begin{center}
	\begin{tabular}{ l c c c c }
		\multicolumn{5}{c}{\textbf{TABLE DES ARMES À FEU}}	\\
		\hline
		\textbf{Arme} 			& \textbf{Dommages} 	& \textbf{Portée} 	& \textbf{Tirs} 	& \textbf{Munitions} \\
		\hline
		\multicolumn{5}{ l }{\textbf{Armes de poing*}} \\
		\hline
		Calibre 22 				& 1d6 		& 10 m 		& 2 	& 6 \\
		Calibre 32 				& 1d6+2 	& 15 m 		& 2 	& 8 \\
		Calibre 38 				& 1d10 		& 15 m 		& 2 	& 8 \\
		Calibre 44 Magnum 		& 2d6+2 	& 20 m 		& 1 	& 6 \\
		Calibre 45 				& 1d10+2 	& 15 m 		& 1 	& 6 \\
		\hline
		\multicolumn{5}{ l }{\textbf{Fusils}} \\
		\hline
		Carabine 22 long rifle 	& 1d6+2 	& 30 m 		& 1 	& 6 \\
		Carabine 30 			& 2d6 		& 50 m 		& 1 	& 6 \\
		Fusil 30-06 			& 2d6+4 	& 100 m 	& 1/2 	& 5 \\
		\hline
		\multicolumn{5}{ l }{\textbf{Fusil de chasse}} \\
		\hline
		Calibre 12 				& 4d6 / 2d6 / 1d6 	& 10 / 20 / 50 m 	& 1 & 5 \\
		Calibre 12 (canon scié) & 4d6 / 1d6 		& 5 / 10 m 			& 1 & 5 \\
		\hline
		\multicolumn{5}{ l }{\textbf{Mitraillettes}} \\
		\hline
		Kalachnikov 			& 2d6+1 	& 100 m 	& 2 ou rafale 	& 30 \\
		M-16 					& 2d6 		& 120 m 	& 2 ou rafale 	& 30 \\
		Uzi 					& 1d10 		& 50 m 		& 2 ou rafale 	& 30 \\
		\hline
		\multicolumn{5}{ l }{\textbf{Armes anciennes}} \\
		\hline
		Arquebuse**				& 1d10 		& 15 m 		& 1 / 1d6+3 	& 1 \\
		Mousquet 				& 1d6+3 	& 15 m 		& 1 / 1d6+3 	& 1 \\
		Pistolet 				& 1d6+2 	& 10 m 		& 1 / 1d6+3 	& 1 \\
		\hline
	\end{tabular}
\end{center}

* Toutes ces armes existent en version automatique et revolver.~\\
$**$ Impossible à manier seul. Nécessite d'être fixée au sol ou à l'harnachement d'un cheval.~\\

\textbf{Notes}~\\
\emph{Dommages} : nombre de dés à lancer si le jet d'attaque est réussi.~\\
\emph{Portée} : distance en mètres à laquelle l'arme est efficace. Au delà elle est moins précise et la compétence du tireur/lanceur est divisée par deux.~\\
\emph{Tirs par round} : nombre de fois que l'arme peut tirer à chaque round.~\\
\emph{Munitions} : nombre de balles dans le magasin. N'oubliez pas de les décompter. Arrivé à 0, il faut recharger.~\\

\end{document}
