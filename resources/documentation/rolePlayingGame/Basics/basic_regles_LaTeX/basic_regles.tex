\documentclass[10pt,twoside,a4paper]{article}
\usepackage[utf8]{inputenc}
\usepackage[french]{babel}
\usepackage{geometry}
\geometry{a4paper, margin=1.5cm}
\usepackage{array}
%% \usepackage{amsfonts,amssymb,amsmath,amsthm}

%% \title{BaSIC : Le jeu de rôle de base}
%% \author{BaSIC \copyright Casus Belli/Multisim sous licence Chaosium}

\usepackage{multicol} % pour {\'e}crire dans certaines zones en colonnes : \begin{multicols}{nb colonnes}...\end{multicols}

\begin{document}

\setlength\parindent{0pt} % \noindent for all document

%% \maketitle
\Large{\textbf{BaSIC : Le jeu de rôle de base}}~\\
\large{BaSIC \copyright Casus Belli/Multisim sous licence Chaosium}~\\

\begin{multicols}{2}

\section*{INTRODUCTION}

S'il y a un système de jeu de rôle dont on connaît les qualités et les défauts, les possibilités et les limites, c'est bien le Basic Role Playing, le système de l'éditeur américain Chaosium. Il fut mis au point à l'aube du jeu de rôle par Steve Perrin et Lynn Willis pour le jeu RuneQuest, puis adapté pour les autres jeux de Chaosium (Elric, Hawkinson, L'appel de Cthulhu, Ellquest) et plus tard Multisim en acquit les droits en France pour son jeu Nephilim. BaSIC est la version la plus épurée du système Chaosium. Un système qui n'est pas forcément le plus réaliste, mais en tout cas le plus intuitif, le plus facile à intégrer, à jouer et à faire jouer, et le plus couramment pratiqué par un vaste nombre de joueurs !

Cette version a été coordonnée par Tristan Lhomme. Introduction extraite d'un préambule de Didier Guiserix paru dans le hors-série Casus Belli n$^{\circ}$19 "BaSIC, le jeu de rôle de base" (1997).

\section*{CRÉER UN PERSONNAGE}

\subsection*{Définitions}

Un personnage est défini par deux séries de chiffres : ses caractéristiques et ses compétences. Les caractéristiques définissent ce qu'est le personnage, et les compétences ce qu'il sait. Votre personnage sera plus ou moins fort, plus ou moins intelligent, plus ou moins adroit, et cela dépend largement du hasard. En revanche, c'est vous qui choisirez si, au cours de sa vie, il a appris à se battre ou s'il a préféré se tourner vers l'érudition, la médecine...

\subsection*{Les caractéristiques}

Elles sont au nombre de sept. Pour un être normal, leur valeur varie entre 3 et 18. Les personnages des joueurs étant des héros, donc par définition un peu meilleurs que la moyenne, leurs caractéristiques iront de 8 à 18.

\textbf{FORce} : C'est une mesure de la puissance musculaire du personnage. Cette caractéristique donne une idée de ce qu'il peut soulever, porter, pousser ou tirer. À titre indicatif, un personnage avec une FORce de 8 est un quidam moyen, qui n'a plus fait de sport après le lycée. Avec une FORce de 18, c'est un athlète qui joue dans la même cour qu'Arnold Schwarzenegger.~\\

\textbf{CONstitution} : Cette caractéristique rend compte de la vitalité et de la santé du personnage. Plus sa CONstitution est élevée, mieux le personnage résistera à la fatigue, aux coups, à la maladie, etc. Avec une CONstitution de 8, c'est un individu normal, peut-être un peu plus sujet aux rhumes que la moyenne. Avec une CONstitution de 18, il n'a probablement jamais été malade de sa vie.~\\

\textbf{TAIlle} : En fait, cette caractéristique englobe la taille et le poids. Être grand et massif vous permet de faire plus mal lors des combats, mais peut poser problème dans d'autres circonstances, par exemple lorsqu'il faut se faufiler par un passage étroit. Pour un personnage masculin, une TAIlle de 8 est légèrement inférieure à la moyenne (dans les 1,65 m pour 60 kg). Une TAIlle de 18 peut indiquer un géant très maigre, ou un individu de taille moyenne mais avec un embonpoint.~\\

\textbf{DEXtérité} : Comme son nom l'indique, cette caractéristique mesure la vivacité et la rapidité physique du personnage. Elle joue un rôle primordial dans les combats. En effet, les personnages dotés d'une DEXtérité élevée agissent plus souvent en premier. Avec une DEXtérité de 8, un personnage est très moyen, et sera bien avisé de ne pas se risquer, par exemple, dans une bagarre de café. Avec une DEXtérité de 18, il est extrêmement rapide et, potentiellement, pourrait jongler avec cinq couteaux et autant de torches enflammées.~\\

\textbf{APParence} : Votre personnage est-il beau ou pas ? S'habille-t-il avec élégance ? Sait-il faire bonne impression au premier contact ? L'APParence mesure tous ces facteurs. Un personnage doté d'une APParence de 8 est le genre d'individu que peu de gens remarquent au milieu d'une foule (ce qui peut, parfois, être un avantage). En revanche, avec une APParence de 18, il attirera l'attention partout où il passe.~\\

\textbf{INTelligence} : Elle décrit les capacités d'apprentissage, de mémorisation et d'analyse de votre personnage. Elle ne remplace pas l'intelligence du joueur, mais gère, à sa place, un certain nombre de tâches fastidieuses. Si votre personnage se retrouve en possession d'un message codé, par exemple, c'est son INTelligence qui servira à déterminer s'il parvient à le déchiffrer. Avec une INTelligence de 8, un personnage n'est pas très malin. À 18, c'est un génie.~\\

\textbf{POUvoir} : Cette caractéristique mesure la volonté du personnage, sa force d'âme et, dans certains univers de jeu, sa capacité à utiliser d'éventuels pouvoirs mentaux (magie ou pouvoirs psioniques). Il sert également à calculer sa Chance (voir plus bas). Avec un POUvoir de 8, votre personnage sera quelqu'un de relativement effacé, sans grandes dispositions pour le paranormal. Avec un POUvoir de 18, au contraire, il sera très impressionnant, et/ou aura un grand potentiel magique.~\\

\subsection*{Les compétences}

Les compétences sont des domaines de connaissance que votre personnage a eu l'occasion d'étudier au cours de sa vie. Elles sont différentes d'un univers de jeu à l'autre. Exemples: se servir d'une épée, parler une langue étrangère, conduire une automobile... Les compétences se mesurent sur 100.

\section*{LA CRÉATION PAS À PAS}

\subsection*{Première étape : le héros}

Demandez au meneur de jeu quel "genre de héros" serait à l'aise dans l'histoire qu'il va vous faire jouer. Cela vous donnera une idée des personnages dont il a besoin.

\subsection*{Deuxième étape : les caractéristiques}

Munissez-vous d'une feuille de personnage, d'un bout de brouillon, d'un crayon et d'une gomme. Lancez sept fois de suite 2d6+6. Notez les sept résultats sur votre brouillon, puis répartissez-les à votre guise entre les caractéristiques.

\subsection*{Troisième étape : les valeurs dérivées}

Les caractéristiques servent à calculer cinq valeurs dérivées. Suivez les instructions ci-dessous et reportez les résultats sur votre feuille de personnage.

\textbf{Points de vie (PV)} : Ils mesurent l'état physique du personnage. Lorsqu'il est blessé, il perd des points de vie. S'il tombe à 0 point de vie, il est mort! Faites la somme de la TAIlle et de la CONstitution, divisez le résultat par deux (en arrondissant à l'entier supérieur, si besoin est).~\\

\textbf{Points d'énergie (PE)} : Ils servent, si l'univers de jeu le permet, à faire de la magie (ou à utiliser des pouvoirs psioniques). Votre capital de points d'énergie est égal à votre POUvoir.~\\

\textbf{Bonus aux dommages} : Plus un personnage est grand et fort, plus il fera mal lors des combats au corps à corps. Additionnez la TAIlle et la FORce de votre personnage, et reportez-vous à la table des bonus aux dommages.~\\

\textbf{Le jet d'Idée} : Etes-vous sûr qu'il soit opportun d'engager votre personnage dans ce tunnel empli de fumée rougeâtre, d'où sortent des ricanements démoniaques ? Le jet d'Idée est un jet de dés qui, s'il est réussi, permet au meneur de jeu de vous souffler une information oubliée ou de vous suggérer un plan d'action. Pour connaître sa valeur, multipliez l'INTelligence du personnage par 5.~\\

\textbf{Le jet de Chance} : Votre personnage est en équilibre précaire au bord d'une falaise. Va-t-il tomber ? Cela dépend s'il a ou non de la chance. Cette valeur est souvent très utile pour simuler le "mouvement du monde" autour des personnages. Comme le jet d'Idée, le jet de Chance est un pourcentage, noté sur 100. Il se calcule en multipliant le POUvoir par 5.~\\

\begin{center}
	\begin{tabular}{ c c }
		%% \multicolumn{2}{c}{\textbf{TABLE DES BONUS AUX DOMMAGES}}	\\
		\multicolumn{2}{c}{\textbf{TABLE DES BONUS}}	\\
		\multicolumn{2}{c}{\textbf{AUX DOMMAGES}}	\\
		\hline
		\textbf{FORce + TAIlle} & \textbf{Bonus} \\
		02 à 24 & Aucun \\
		25 à 32 & +1d3 \\
		33 à 40 & +1d6 \\
		41 à 60 & +2d6 \\
		\hline
	\end{tabular}
\end{center}

\subsection*{Quatrième étape : les compétences}

Regardez la liste des professions disponibles dans votre univers de jeu et choisissez-en une. Vous avez 300 points à répartir dans les compétences qui en découlent (sans obligation de les prendre toutes). Ensuite, vous pourrez répartir 150 points parmi toutes les autres compétences de la feuille de personnage, à votre guise.

\section*{Liste des professions avec compétences}

\subsection*{Exemples pour un univers médiéval-fantastique}

\begin{itemize}
	\setlength{\parskip}{0pt}
	\setlength{\itemsep}{0pt plus 1pt}
    \item[$\bullet$] \textbf{Chasseur} : Athlétisme, Armes d'hast, Armes de jet, Artisanat, Connaissance de la nature, Orientation, Secourisme, Survie.
    \item[$\bullet$] \textbf{Chevalier} : Athlétisme, Equitation, Droit et usages, Armes de mêlée, Armes d'hast, Esquive, Vigilance, Lire et écrire.
    \item[$\bullet$] \textbf{Erudit} : Alchimie, Connaissance de la nature, Connaissance des peuples, Droit et usages, Légendes, Lire et écrire, Potions et herbes, Sagacité.
    \item[$\bullet$] \textbf{Fermier} : Armes d'hast, Armes de lancer, Artisanat, Commerce, Connaissance de la nature, Orientation, Potions et herbes, Secourisme.
    \item[$\bullet$] \textbf{Magicien} : Alchimie, Droit et usages, Légendes, Lire et écrire, Persuasion, Sagacité, Vigilance, 3 sortilèges au choix.
    \item[$\bullet$] \textbf{Soldat} : Athlétisme, Armes de mêlée, Armes d'hast, Armes de tir, Esquive, Discrétion, Vigilance, Secourisme.
    \item[$\bullet$] \textbf{Troubadour} : Artisanat (musique), Connaissance des peuples ou des religions, Culture générale, Droit et usages, Légendes, Lire et écrire, Persuasion, Sagacité.
    \item[$\bullet$] \textbf{Voleur} : Armes de lancer, Armes de mêlée, Cascade, Discrétion, Esquive, Persuasion, Sagacité, Serrurerie.
\end{itemize}

\subsection*{Exemples pour un univers contemporain}

\begin{itemize}
	\setlength{\parskip}{0pt}
	\setlength{\itemsep}{0pt plus 1pt}
    \item[$\bullet$] \textbf{Agent de renseignement} : Cascade, Déguisement, Droit et usages, Histoire et géographie, Informatique, Lutte, Renseignement, Sciences appliquées, Serrurerie.
    \item[$\bullet$] \textbf{Journaliste} : Bibliothèque, Conduire (auto ou moto), Connaissance de la rue, Culture générale, Langue étrangère, Persuasion, Sagacité, Vigilance.
    \item[$\bullet$] \textbf{Homme d'affaire} : Commerce, Comptabilité, Droit, Informatique, Langue étrangère, Persuasion, Sagacité, (plus une autre au choix).
    \item[$\bullet$] \textbf{Militaire} : Cascade, Conduire (au choix), Leadership, Navigation, Pilotage (au choix), Sabotage, Lutte, Armes de poing, Fusil.
    \item[$\bullet$] \textbf{Spécialiste en sciences humaines} : Bibliothèque, Culture générale, Droit et usages, Langue étrangère, Histoire et géographie, Orientation, Sciences sociales, Survie.
\end{itemize}

\section*{Cinquième étape : les finitions}

Le jeu de rôle est un moyen de raconter des histoires, comme les romans ou le cinéma, et les histoires sont toujours meilleures lorsqu'elles ont des héros crédibles. Pour l'instant, votre personnage est un tas de chiffres gribouillés au crayon sur un formulaire abscons. Pour lui insuffler un peu de vie, la première chose à faire est de remplir son "état civil".

\begin{itemize}
	\setlength{\parskip}{0pt}
	\setlength{\itemsep}{0pt plus 1pt}
    \item[$\bullet$] \textbf{La description} : il existe une foule de détails qu'aucun jet de dés ne définira à votre place. Quelle est la couleur des yeux de votre personnage ? Celle de ses cheveux ? Est-il droitier ou gaucher ? Comment s'habille-t-il ?
    \item[$\bullet$] \textbf{Le comportement} : Pensez à votre personnage comme à une vraie personne. A-t-il des goûts particuliers ? Des choses dont il a horreur ?
    \item[$\bullet$] \textbf{L'histoire personnelle} : Maintenant que vous avez une assez bonne image de lui dans le présent, remontez un peu dans son passé.
    \item[$\bullet$] \textbf{La motivation} : Idéalement, il faudrait que votre personnage ait une bonne raison de se jeter dans l'aventure.
\end{itemize}

\section*{LES RÈGLES DE BASE}

\section*{Question de bon sens}

\begin{itemize}
	\setlength{\parskip}{0pt}
	\setlength{\itemsep}{0pt plus 1pt}
    \item \textbf{Première règle} : Une action impossible à rater réussit toujours.
    \item \textbf{Deuxième règle} : Une action impossible à réussir échoue toujours.
    \item \textbf{Troisième règle} : Pour tous les autres cas, faites un jet de compétence.
    \item \textbf{Quatrième règle} : Lorsqu'il est impossible de faire un jet de compétence, utilisez un jet de caractéristique ou un jet d'opposition.
\end{itemize}

\section*{Action immanquables}

Les actions physiques et intellectuelles tentées dans des situations banales réussissent toujours.

\section*{Actions impossibles}

Les actions absurdes et irréalisables échouent toujours.

\section*{Jets de compétence}

La plupart des situations à résoudre se situent entre ces deux extrêmes, dans des circonstances où l'échec et la réussite sont tous deux possibles.

\begin{itemize}
	\setlength{\parskip}{0pt}
	\setlength{\itemsep}{0pt plus 1pt}
    \item \textbf{Maladresse} : Lorsque le résultat est compris entre 96 et 00, non seulement l'action est manquée, mais en plus, l'échec est particulièrement grave.
    \item \textbf{Réussite critique} : En revanche, avec un résultat compris entre 01 et 05, l'action est réussie de manière particulièrement brillante.
\end{itemize}

\section*{Bonus et malus}

\begin{center}
	\begin{tabular}{ c c c }
		\multicolumn{3}{c}{\textbf{TABLE DE CIRCONSTANCES}}	\\
		\hline
		\textbf{Modif} & \textbf{Circonstance} & \textbf{Exemple} \\
		-20\% & Très difficile & A \\
		-10\% & Difficile & B \\
		+10\% & Facile & C \\
		+20\% & Plus que facile & D \\
		\hline
	\end{tabular}
\end{center}

\begin{itemize}
	\setlength{\parskip}{0pt}
	\setlength{\itemsep}{0pt plus 1pt}
	\item[A] Tirer de nuit sur cible mobile, ou désamorcer une bombe qui va sauter d'une seconde à l'autre.
	\item[B] Tirer sur cible mobile ou de petite taille, crocheter une serrure particulièrement bien conçue.
	\item[C] Tirer sur cible immobile et de grande taille, escalader un mur avec de nombreuses prises.
	\item[D] Suivre un flâneur qui ne se méfie pas, se souvenir que la bastille a été prise un 14 juillet. 
\end{itemize}

\section*{Expérience}

Les sept caractéristiques ne peuvent pas augmenter, pas plus que leurs dérivées. Les scores de Chance ou Idée ne bougeront jamais, pas plus que le total des points de vie ou le bonus aux dommages. En revanche, les compétences peuvent augmenter.

\section*{Astuces}

\subsection*{Les jets de caractéristique}

Il existe parfois des situations où aucune compétence ne s'applique. Dans ce cas, la caractéristique concernée est multipliée par un nombre entre 1 et 5.

\subsection*{Les jets en opposition}

Parfois, vous aurez besoin de savoir ce qui se passe lorsqu'un personnage lutte contre quelque chose qui lui résiste.

\end{multicols}

\begin{center}
\begin{tabular}{ c c c c c c c c c c c c c c c c c c c c c c }
		\multicolumn{21}{c}{\textbf{TABLE DE RÉSISTANCE}}		\\
		\hline
 		\multicolumn{21}{c}{\textbf{Caractéristique active}}	\\
		   & 01 & 02 & 03 & 04 & 05 & 06 & 07 & 08 & 09 & 10 & 11 & 12 & 13 & 14 & 15 & 16 & 17 & 18 & 19 & 20 & 21 \\
		01 & 50 & 55 & 60 & 65 & 70 & 75 & 80 & 85 & 90 & 95 & - & - & - & - & - & - & - & - & - & - & - \\
		02 & 45 & 50 & 55 & 60 & 65 & 70 & 75 & 80 & 85 & 90 & 95 & - & - & - & - & - & - & - & - & - & - \\
		03 & 40 & 45 & 50 & 55 & 60 & 65 & 70 & 75 & 80 & 85 & 90 & 95 & - & - & - & - & - & - & - & - & - \\
		04 & 35 & 40 & 45 & 50 & 55 & 60 & 65 & 70 & 75 & 80 & 85 & 90 & 95 & - & - & - & - & - & - & - & - \\
		05 & 30 & 35 & 40 & 45 & 50 & 55 & 60 & 65 & 70 & 75 & 80 & 85 & 90 & 95 & - & - & - & - & - & - & - \\
		06 & 25 & 30 & 35 & 40 & 45 & 50 & 55 & 60 & 65 & 70 & 75 & 80 & 85 & 90 & 95 & - & - & - & - & - & - \\
		07 & 20 & 25 & 30 & 35 & 40 & 45 & 50 & 55 & 60 & 65 & 70 & 75 & 80 & 85 & 90 & 95 & - & - & - & - & - \\
		08 & 15 & 20 & 25 & 30 & 35 & 40 & 45 & 50 & 55 & 60 & 65 & 70 & 75 & 80 & 85 & 90 & 95 & - & - & - & - \\
		09 & 10 & 15 & 20 & 25 & 30 & 35 & 40 & 45 & 50 & 55 & 60 & 65 & 70 & 75 & 80 & 85 & 90 & 95 & - & - & - \\
		10 & 05 & 10 & 15 & 20 & 25 & 30 & 35 & 40 & 45 & 50 & 55 & 60 & 65 & 70 & 75 & 80 & 85 & 90 & 95 & - & - \\
		11 & - & 05 & 10 & 15 & 20 & 25 & 30 & 35 & 40 & 45 & 50 & 55 & 60 & 65 & 70 & 75 & 80 & 85 & 90 & 95 & - \\
		12 & - & - & 05 & 10 & 15 & 20 & 25 & 30 & 35 & 40 & 45 & 50 & 55 & 60 & 65 & 70 & 75 & 80 & 85 & 90 & 95 \\
		13 & - & - & - & 05 & 10 & 15 & 20 & 25 & 30 & 35 & 40 & 45 & 50 & 55 & 60 & 65 & 70 & 75 & 80 & 85 & 90 \\
		14 & - & - & - & - & 05 & 10 & 15 & 20 & 25 & 30 & 35 & 40 & 45 & 50 & 55 & 60 & 65 & 70 & 75 & 80 & 85 \\
		15 & - & - & - & - & - & 05 & 10 & 15 & 20 & 25 & 30 & 35 & 40 & 45 & 50 & 55 & 60 & 65 & 70 & 75 & 80 \\
		16 & - & - & - & - & - & - & 05 & 10 & 15 & 20 & 25 & 30 & 35 & 40 & 45 & 50 & 55 & 60 & 65 & 70 & 75 \\
		17 & - & - & - & - & - & - & - & 05 & 10 & 15 & 20 & 25 & 30 & 35 & 40 & 45 & 50 & 55 & 60 & 65 & 70 \\
		18 & - & - & - & - & - & - & - & - & 05 & 10 & 15 & 20 & 25 & 30 & 35 & 40 & 45 & 50 & 55 & 60 & 65 \\
		19 & - & - & - & - & - & - & - & - & - & 05 & 10 & 15 & 20 & 25 & 30 & 35 & 40 & 45 & 50 & 55 & 60 \\
		20 & - & - & - & - & - & - & - & - & - & - & 05 & 10 & 15 & 20 & 25 & 30 & 35 & 40 & 45 & 50 & 55 \\
		21 & - & - & - & - & - & - & - & - & - & - & - & 05 & 10 & 15 & 20 & 25 & 30 & 35 & 40 & 45 & 50 \\
		\hline
	\end{tabular}
\end{center}

\section*{Un peu de calcul}

Si vous devez confronter des valeurs qui ne se trouvent pas sur la table de résistance, utilisez la formule suivante : Le pourcentage de chance de base est de 50\% - (carac. passive x 5) + (carac. active x 5).

\section*{Temps de jeu}

Le temps de jeu et le temps réel sont deux notions différentes, qui ont peu de rapport l'une avec l'autre. Un personnage peut passer plusieurs jours penché sur un grimoire ardu à déchiffrer, mais il ne faudra que quelques secondes au meneur de jeu pour résumer aux joueurs ce qu'il contient.

\section*{Mouvement et poursuites}

Les personnages (et tous les autres êtres humains) se déplacent en moyenne de 8 mètres par round. C'est leur valeur de Mouvement.

\section*{Blessures}

Les points de vie mesurent l'état physique du personnage. Lorsqu'il est blessé, il subit des dommages.

\begin{itemize}
	\setlength{\parskip}{0pt}
	\setlength{\itemsep}{0pt plus 1pt}
    \item \textbf{Blessures graves} : Si un personnage perd en une seule fois la moitié du nombre actuel de ses points de vie, il est blessé grièvement.
    \item \textbf{Agonie} : Un personnage qui se trouve à 1, 2 ou 3 points de vie doit faire un jet de CON x 3 par round.
    \item \textbf{Mort} : Un personnage qui tombe à 0 point de vie ou moins meurt instantanément.
\end{itemize}

\section*{Causes de blessures}

\begin{itemize}
	\setlength{\parskip}{0pt}
	\setlength{\itemsep}{0pt plus 1pt}
    \item \textbf{Chutes} : Retirez 1d6 points de vie par tranche de 3 mètres de chute.
    \item \textbf{Feux} : Si l'on s'en sert comme arme, une torche enflammée inflige 1d6 points de dommages par round.
    \item \textbf{Asphyxie/noyade} : Lorsqu'un personnage est exposé à des gaz toxiques, immergé dans un liquide, ou étranglé par un individu mal intentionné, on commence à découper le temps en rounds.
    \item \textbf{Explosions} : Dans les univers où ils existent, les explosifs infligent des dommages dans un certain rayon.
    \item \textbf{Poison} : Les poisons en tout genre sont définis par une seule caractéristique : leur VIRulence.
    \item \textbf{Maladie} : Les règles sur les maladies sont très proches de celles sur les poisons.
\end{itemize}

\section*{Guérison}

Bien entendu, un personnage blessé finira par guérir. Ce n'est qu'une question de temps...

\section*{LES COMPÉTENCES}

Les compétences représentent les domaines de connaissance que le personnage a appris. Elles peuvent varier d'un univers de jeu à l'autre et sont sujettes à amélioration.

\section*{Liste universelle}

\begin{itemize}
	\setlength{\parskip}{0pt}
	\setlength{\itemsep}{0pt plus 1pt}
    \item \textbf{Art/Artisanat (05\%)} : Il s'agit en fait d'une famille de compétences, permettant de fabriquer des objets de première nécessité et/ou des œuvres d'art.
    \item \textbf{Athlétisme (15\%)} : Cette compétence regroupe toutes les activités physiques : course, nage, saut, escalade.
    \item \textbf{Bricolage (10\%)} : Le contenu de cette compétence change selon les univers de jeu.
    \item \textbf{Cascade (10\%)} : Cette compétence combine l'agilité et la souplesse du personnage.
    \item \textbf{Chercher* (20\%)} : On utilise Chercher pour fouiller un endroit.
    \item \textbf{Culture générale (20\%)} : Cette compétence simule une connaissance superficielle d'un grand nombre de sujets historiques, culturels ou scientifiques.
    \item \textbf{Commerce (20\%)} : La version médiévale permet de savoir où acheter et vendre quelles marchandises, et de marchander de manière efficace.
    \item \textbf{Connaissance de la rue (10\%)} : Cette compétence permet de savoir où trouver des contacts : indics, tenanciers de bars louches, fabricants de faux papiers, etc.
    \item \textbf{Déguisement (10\%)} : Grâce à Déguisement, un personnage peut modifier son apparence.
    \item \textbf{Discrétion (15\%)} : Demandez un jet de Discrétion lorsque les personnages ont besoin de se déplacer silencieusement.
    \item \textbf{Droit, administration, usages (10\%)} : Cette compétence est un mélange de politesse, de questions posées aux bonnes personnes et de connaissances livresques.
    \item \textbf{Équitation (20\%)} : Cette compétence permet d'utiliser les animaux de monte.
    \item \textbf{Esquiver (25\%)} : L'Esquive est une compétence précieuse en combat.
    \item \textbf{Langue natale (80\%)} : Dans les univers où l'instruction est obligatoire, cette compétence recouvre la lecture, l'écriture et la communication orale.
    \item \textbf{Langue étrangère (00\%)} : Chaque langue étrangère est une compétence distincte.
    \item \textbf{Leadership (15\%)} : Leadership permet de commander un groupe, de l'organiser, de le motiver.
    \item \textbf{Navigation (00\%)} : La Navigation est l'art de s'orienter sur l'eau, de manœuvrer un navire.
    \item \textbf{Orientation* (15\%)} : L'Orientation sert à ne pas se perdre dans un environnement peu familier.
    \item \textbf{Persuasion (15\%)} : Cette compétence sert à convaincre autrui du bien-fondé de ses arguments.
    \item \textbf{Sagacité (20\%)} : Cette compétence permet au personnage qui l'utilise d'avoir une idée de l'humeur et des motivations d'un personnage non-joueur.
    \item \textbf{Secourisme (30\%)} : Grâce à Secourisme, on peut ranimer les personnages inconscients, et surtout soigner les blessés.
    \item \textbf{Survie (10\%)} : Faites faire des jets de Survie lorsqu'un personnage se trouve dans un environnement hostile.
    \item \textbf{Vigilance (20\%)} : Utilisez cette compétence lorsqu'un personnage file un suspect, essaye d'écouter une conversation ou de remarquer un indice.
\end{itemize}

\section*{Liste spécifique (fantastique-contemporain)}

\begin{itemize}
	\setlength{\parskip}{0pt}
	\setlength{\itemsep}{0pt plus 1pt}
    \item \textbf{Bibliothèque (25\%)} : Cette compétence permet d'utiliser une bibliothèque publique.
    \item \textbf{Comptabilité (00\%)} : La compétence Comptabilité permet de se repérer dans les comptes d'un particulier ou d'une entreprise.
    \item \textbf{Conduire... (20\%)} : La compétence Conduire est, en fait, triple.
    \item \textbf{Histoire et géographie (10\%)} : Cette compétence fournit des informations dans ces deux domaines.
    \item \textbf{Informatique (20\%)} : De nos jours, n'importe qui peut allumer un ordinateur et apprendre à se servir d'un logiciel.
    \item \textbf{Médecine (00\%)} : Un jet réussi dans cette compétence permet de redonner 1d6 points de vie à un blessé.
    \item \textbf{Paranormal (20\%)} : Cette compétence regroupe des informations sur une foule de sujets.
    \item \textbf{Piloter... (00\%)} : Comme la conduite, le pilotage est en fait un groupe de compétences.
    \item \textbf{Plongée (00\%)} : Cette compétence permet de savoir se débrouiller avec des bouteilles.
    \item \textbf{Renseignements (20\%)} : Cette compétence concerne tout ce qui a trait au monde de l'espionnage et du crime organisé.
    \item \textbf{Sabotage (05\%)} : Le Sabotage permet de manier des explosifs, fabriquer des cocktails Molotov.
    \item \textbf{Science appliquée (00\%)} : Permet d'identifier et d'utiliser du matériel de haute technologie.
    \item \textbf{Science pure (00\%)} : Cette compétence regroupe la plupart des sciences "dures".
    \item \textbf{Sciences sociales (00\%)} : Cette compétence regroupe tout ce qui concerne les sciences humaines.
    \item \textbf{Serrurerie (15\%)} : C'est l'art et la manière d'ouvrir une serrure sans laisser de traces.
\end{itemize}

\section*{Liste spécifique (médiéval-fantastique)}

\begin{itemize}
	\setlength{\parskip}{0pt}
	\setlength{\itemsep}{0pt plus 1pt}
    \item \textbf{Alchimie (00\%)} : Cette compétence correspond à une connaissance primitive de la chimie.
    \item \textbf{Art de la guerre (00\%)} : Cette compétence permet de dresser un plan de bataille.
    \item \textbf{Civilisations anciennes (10\%)} : Cette compétence donne des informations sur les peuples qui ont précédé les cultures actuelles.
    \item \textbf{Conduite d'attelage (15\%)} : L'art et la manière d'atteler un chariot, un carrosse.
    \item \textbf{Connaissance de la nature (10\%)} : Cette compétence regroupe des connaissances de base sur les animaux, les plantes, la météo.
    \item \textbf{Connaissance des peuples (20\%)} : La Connaissance des peuples donne des informations plus ou moins précises sur "l'étranger".
    \item \textbf{Connaissance des religions (10\%)} : Comme la précédente, cette compétence fournit des informations, mais sur un domaine plus restreint.
    \item \textbf{Démolition/sape (00\%)} : Démolition est l'art et la manière de creuser des galeries sous un rempart.
    \item \textbf{Dressage (15\%)} : Le Dressage permet de mater un animal sauvage ou de l'apprivoiser.
    \item \textbf{Héraldique (00\%)} : Cette compétence permet de reconnaître les blasons des familles nobles.
    \item \textbf{Légendes* (05\%)} : La possession de cette compétence indique que le personnage connaît de nombreuses légendes.
    \item \textbf{Lire et écrire (00\%)} : Dans les mondes où la plupart des gens n'ont pas accès à l'instruction, la lecture et l'écriture sont une compétence à part entière.
    \item \textbf{Potions et herbes (10\%)} : En dehors de son nom, Potions et herbes est exactement identique à Médecine.
\end{itemize}

\section*{LE COMBAT}

Les règles sur le combat peuvent paraître compliquées. Elles reposent sur des principes simples, mais à la première lecture, vous risquez d'être noyé sous le flot de petits détails qui les rendent plus réalistes, mais moins intelligibles.

\section*{Round}

Le round sert à découper le combat de manière à ce qu'il soit jouable. C'est une unité de temps au cours de laquelle tous les participants du combat ont l'occasion d'agir au moins une fois.

\section*{Initiative}

Au début de chaque round, chaque participant au combat lance 1d6 et y ajoute la valeur de sa DEXtérité, puis annonce le résultat.

\section*{Déclaration d'intention}

Le meneur de jeu procède à un tour de table, demandant à chaque joueur ce qu'il compte faire pendant ce round.

\section*{Ordre des attaques}

À l'intérieur du round, on distingue trois passes, qui se succèdent toujours dans le même ordre.

\begin{itemize}
    \item Première passe : les personnages (PJ et PNJ) qui ont des armes à feu ou des armes à projectiles prêtes à servir agissent en premier.
    \item Deuxième passe : une fois que ces premiers tirs ont eu lieu, on prend les personnages qui n'ont pas encore agi.
    \item Troisième passe : les personnages qui ont des armes à feu pouvant tirer deux fois par round ont droit à leur deuxième tir.
\end{itemize}

\section*{Esquiver et parer}

En plus de l'attaque et du déplacement, les personnages ont deux autres possibilités : l'esquive et la parade.

\section*{Blesser l'adversaire}

Il suffit de réussir un jet sous la compétence qui régit l'arme utilisée.

\section*{Protections}

Un combattant désireux de vivre vieux ne se lance pas dans la mêlée sans protections.

\begin{itemize}
	\setlength{\parskip}{0pt}
	\setlength{\itemsep}{0pt plus 1pt}
    \item Les armures.
    \item Les boucliers.
    \item Le terrain.
\end{itemize}

\section*{Réussite critique et maladresse en combat}

Pour presque toutes les compétences de combat, un résultat compris entre 01 et 05 signifie que les dommages infligés par l'attaque sont plus importants que prévu.

\section*{Combat au corps à corps}

On peut distinguer deux cas : celui où les combattants ne sont pas armés, et celui où ils le sont.

\begin{itemize}
	\setlength{\parskip}{0pt}
	\setlength{\itemsep}{0pt plus 1pt}
    \item \textbf{Combat à mains nues} : Le combat à mains nues dépend de deux compétences : Bagarre et Lutte.
    \item \textbf{Combat aux armes blanches} : Le combat au corps à corps dépend de deux compétences : armes de mêlée et armes d'hast.
\end{itemize}

\section*{Combat à distance}

Là encore, le combat à distance peut être divisé en deux grandes catégories : les armes de jet et les armes à feu.

\begin{itemize}
	\setlength{\parskip}{0pt}
	\setlength{\itemsep}{0pt plus 1pt}
    \item \textbf{Armes de jet} : Les armes de jet ne peuvent être utilisées qu'une fois par round.
    \item \textbf{Armes à feu} : Comme les armes de jet, les armes à feu tirent à la première ou à la troisième passe de chaque round.
\end{itemize}

\section*{TABLE DES ARMURES}

\begin{center}
\begin{tabular}{|l|c|}
\hline
Type & Protection \\
\hline
Cuir souple (blouson) & 1 \\
Cuir rigide (armure) & 2 \\
Cuir et métal & 4 \\
Cotte de mailles (gilet pare-balles) & 6 \\
Armure de plaques* & 8 \\
\hline
\end{tabular}
\end{center}

\section*{TABLE DES ARMES D'HAST ET DE MÊLÉE}

\begin{center}
\begin{tabular}{|l|l|c|}
\hline
Arme & Compétence & Dommages \\
\hline
Dague / poignard & Mêlée & 1d3+2 \\
Épée / rapière & Mêlée & 1d6+2 \\
Épieu / pique & Hast & 2d6 \\
Espadon* & Mêlée & 2d6 \\
Gourdin & Mêlée & 1d6 \\
Hache de bataille & Mêlée & 2d6 \\
Hachette & Mêlée & 1d6+1 \\
Hallebarde* & Hast & 3d6 \\
Javelot / lance courte & Hast & 1d6+1 \\
Masse / fléau d'armes & Mêlée & 1d6+2 \\
Coup de poing, de tête, etc. & Bagarre & 1d3 \\
Bouclier & Celle de l'arme principale & - \\
\hline
\end{tabular}
\end{center}

\section*{TABLE DES ARMES DE TIR ET DE LANCER}

\begin{center}
\begin{tabular}{|l|l|c|c|c|}
\hline
Arme & Compétence & Portée efficace & Portée maximale & Dommages \\
\hline
Arbalète & Tir & 20 m & 50 m & 2d6 \\
Arc & Tir & 50 m & 150 m & 1d6+2 \\
Dague & Lancer & FOR en m & FOR x 2 m & 1d3+2 \\
Fronde & Tir & 50 m & 100 m & 1d6 \\
Javelot & Lancer & FOR en m & FOR x 3 m & 1d6+1 \\
\hline
\end{tabular}
\end{center}

\section*{TABLE DES ARMES À FEU}

\begin{center}
\begin{tabular}{|l|c|c|c|c|}
\hline
Arme & Dommages & Portée & Tirs & Munitions \\
\hline
\multicolumn{5}{|c|}{Armes de poing} \\
\hline
Calibre 22 & 1d6 & 10 m & 2 & 6 \\
Calibre 32 & 1d6+2 & 15 m & 2 & 8 \\
Calibre 38 & 1d10 & 15 m & 2 & 8 \\
Calibre 44 Magnum & 2d6+2 & 20 m & 1 & 6 \\
Calibre 45 & 1d10+2 & 15 m & 1 & 6 \\
\hline
\multicolumn{5}{|c|}{Fusils} \\
\hline
Carabine 22 long rifle & 1d6+2 & 30 m & 1 & 6 \\
Carabine 30 & 2d6 & 50 m & 1 & 6 \\
Fusil 30-06 & 2d6+4 & 100 m & 1/2 & 5 \\
\hline
\multicolumn{5}{|c|}{Fusil de chasse} \\
\hline
Calibre 12 & 4d6 / 2d6 / 1d6 & 10 / 20 / 50 m & 1 & 5 \\
Calibre 12 (canon scié) & 4d6 / 1d6 & 5 / 10 m & 1 & 5 \\
\hline
\multicolumn{5}{|c|}{Mitraillettes} \\
\hline
Kalachnikov & 2d6+1 & 100 m & 2 ou rafale & 30 \\
M-16 & 2d6 & 120 m & 2 ou rafale & 30 \\
Uzi & 1d10 & 50 m & 2 ou rafale & 30 \\
\hline
\multicolumn{5}{|c|}{Armes anciennes} \\
\hline
Arquebuse & 1d10 & 15 m & 1 / 1d6+3 & 1 \\
Mousquet & 1d6+3 & 15 m & 1 / 1d6+3 & 1 \\
Pistolet & 1d6+2 & 10 m & 1 / 1d6+3 & 1 \\
\hline
\end{tabular}
\end{center}

\end{document}
