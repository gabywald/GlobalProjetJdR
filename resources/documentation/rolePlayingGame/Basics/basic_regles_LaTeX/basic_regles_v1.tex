\documentclass{article}
\usepackage[utf8]{inputenc}
\usepackage[french]{babel}
\usepackage{geometry}
\geometry{a4paper, margin=1.5cm}
\usepackage{longtable}
\usepackage{array}

\title{BaSIC : Le jeu de rôle de base}
\author{Tristan Lhomme, Didier Guiserix}
\date{1997}

\usepackage{multicol} % pour {\'e}crire dans certaines zones en colonnes : \begin{multicols}{nb colonnes}...\end{multicols}

\begin{document}

\setlength\parindent{0pt} % \noindent for all document

\maketitle

\begin{multicols}{2}

\section*{INTRODUCTION}

S'il y a un système de jeu de rôle dont on connaît les qualités et les défauts, les possibilités et les limites, c'est bien le Basic Role Playing, le système de l'éditeur américain Chaosium. Il fut mis au point à l'aube du jeu de rôle par Steve Perrin et Lynn Willis pour le jeu RuneQuest, puis adapté pour les autres jeux de Chaosium (Elric, Hawkinson, L'appel de Cthulhu, Ellquest) et plus tard Multisim en acquit les droits en France pour son jeu Nephilim. BaSIC est la version la plus épurée du système Chaosium. Un système qui n'est pas forcément le plus réaliste, mais en tout cas le plus intuitif, le plus facile à intégrer, à jouer et à faire jouer, et le plus couramment pratiqué par un vaste nombre de joueurs !

Cette version a été coordonnée par Tristan Lhomme. Introduction extraite d'un préambule de Didier Guiserix paru dans le hors-série Casus Belli n$^{\circ}$19 "BaSIC, le jeu de rôle de base" (1997).

\section*{CRÉER UN PERSONNAGE}

\subsection*{Définitions}

Un personnage est défini par deux séries de chiffres : ses caractéristiques et ses compétences. Les caractéristiques définissent ce qu'est le personnage, et les compétences ce qu'il sait. Votre personnage sera plus ou moins fort, plus ou moins intelligent, plus ou moins adroit, et cela dépend largement du hasard. En revanche, c'est vous qui choisirez si, au cours de sa vie, il a appris à se battre ou s'il a préféré se tourner vers l'érudition, la médecine...

\subsection*{Les caractéristiques}

Elles sont au nombre de sept. Pour un être normal, leur valeur varie entre 3 et 18. Les personnages des joueurs étant des héros, donc par définition un peu meilleurs que la moyenne, leurs caractéristiques iront de 8 à 18.

\begin{itemize}
    \item \textbf{FORce} : Mesure de la puissance musculaire du personnage.
    \item \textbf{CONstitution} : Rend compte de la vitalité et de la santé du personnage.
    \item \textbf{TAIlle} : Englobe la taille et le poids.
    \item \textbf{DEXtérité} : Mesure la vivacité et la rapidité physique.
    \item \textbf{APParence} : Mesure l'apparence physique et l'impression faite aux autres.
    \item \textbf{INTelligence} : Décrit les capacités d'apprentissage, de mémorisation et d'analyse.
    \item \textbf{POUvoir} : Mesure la volonté du personnage, sa force d'âme.
\end{itemize}

\subsection*{Les compétences}

Les compétences sont des domaines de connaissance que votre personnage a eu l'occasion d'étudier au cours de sa vie. Elles sont différentes d'un univers de jeu à l'autre.

\section*{LA CRÉATION PAS À PAS}

\subsection*{Première étape : le héros}

Demandez au meneur de jeu quel "genre de héros" serait à l'aise dans l'histoire qu'il va vous faire jouer.

\subsection*{Deuxième étape : les caractéristiques}

Lancez sept fois de suite 2d6+6. Notez les sept résultats sur votre brouillon, puis répartissez-les à votre guise entre les caractéristiques.

\subsection*{Troisième étape : les valeurs dérivées}

\begin{itemize}
    \item \textbf{Points de vie (PV)} : Faites la somme de la TAIlle et de la CONstitution, divisez le résultat par deux.
    \item \textbf{Points d'énergie (PE)} : Égaux à votre POUvoir.
    \item \textbf{Bonus aux dommages} : Additionnez la TAIlle et la FORce, reportez-vous à la table des bonus aux dommages.
\end{itemize}

\section*{TABLE DES BONUS AUX DOMMAGES}

\begin{center}
\begin{tabular}{|c|c|}
\hline
FORce + TAIlle & Bonus \\
\hline
02 à 24 & Aucun \\
25 à 32 & +1d3 \\
33 à 40 & +1d6 \\
41 à 60 & +2d6 \\
\hline
\end{tabular}
\end{center}

\section*{LES COMPÉTENCES}

Les compétences représentent les domaines de connaissance que le personnage a appris. Elles peuvent varier d'un univers de jeu à l'autre.

\section*{LISTE DES PROFESSIONS AVEC COMPÉTENCES}

\subsection*{Exemples pour un univers médiéval-fantastique}

\begin{itemize}
    \item \textbf{Chasseur} : Athlétisme, Armes d'hast, Armes de jet, Artisanat, Connaissance de la nature, Orientation, Secourisme, Survie.
    \item \textbf{Chevalier} : Athlétisme, Equitation, Droit et usages, Armes de mêlée, Armes d'hast, Esquive, Vigilance, Lire et écrire.
    \item \textbf{Erudit} : Alchimie, Connaissance de la nature, Connaissance des peuples, Droit et usages, Légendes, Lire et écrire, Potions et herbes, Sagacité.
    \item \textbf{Fermier} : Armes d'hast, Armes de lancer, Artisanat, Commerce, Connaissance de la nature, Orientation, Potions et herbes, Secourisme.
    \item \textbf{Magicien} : Alchimie, Droit et usages, Légendes, Lire et écrire, Persuasion, Sagacité, Vigilance, 3 sortilèges au choix.
    \item \textbf{Soldat} : Athlétisme, Armes de mêlée, Armes d'hast, Armes de tir, Esquive, Discrétion, Vigilance, Secourisme.
    \item \textbf{Troubadour} : Artisanat (musique), Connaissance des peuples ou des religions, Culture générale, Droit et usages, Légendes, Lire et écrire, Persuasion, Sagacité.
    \item \textbf{Voleur} : Armes de lancer, Armes de mêlée, Cascade, Discrétion, Esquive, Persuasion, Sagacité, Serrurerie.
\end{itemize}

\section*{LES RÈGLES DE BASE}

\section*{QUESTION DE BON SENS}

\begin{itemize}
    \item Une action impossible à rater réussit toujours.
    \item Une action impossible à réussir échoue toujours.
    \item Pour tous les autres cas, faites un jet de compétence.
    \item Lorsqu'il est impossible de faire un jet de compétence, utilisez un jet de caractéristique ou un jet d'opposition.
\end{itemize}

\section*{JETS DE COMPÉTENCE}

La plupart des situations à résoudre se situent entre ces deux extrêmes, dans des circonstances où l'échec et la réussite sont tous deux possibles.

\section*{BONUS ET MALUS}

\begin{center}
\begin{tabular}{|c|c|c|}
\hline
Modif & Circonstance & Exemple \\
\hline
-20\% & Très difficile & A \\
-10\% & Difficile & B \\
+10\% & Facile & C \\
+20\% & Plus que facile & D \\
\hline
\end{tabular}
\end{center}

\section*{EXPÉRIENCE}

Les sept caractéristiques ne peuvent pas augmenter, pas plus que leurs dérivées. Les compétences peuvent augmenter, selon le principe : c'est en forgeant qu'on devient forgeron...

\section*{LES JETS DE CARACTÉRISTIQUE}

Il existe parfois des situations où aucune compétence ne s'applique. Dans ce cas, la caractéristique concernée est multipliée par un nombre entre 1 et 5.

\section*{LES JETS EN OPPOSITION}

Parfois, vous aurez besoin de savoir ce qui se passe lorsqu'un personnage lutte contre quelque chose qui lui résiste.

\end{multicols}

\section*{TABLE DE RÉSISTANCE}

\begin{center}
\begin{tabular}{|c|c|c|c|c|c|c|c|c|c|c|c|c|c|c|c|c|c|c|c|c|c|}
\hline
 & \multicolumn{20}{c|}{Caractéristique active} \\
\cline{2-21}
 & 01 & 02 & 03 & 04 & 05 & 06 & 07 & 08 & 09 & 10 & 11 & 12 & 13 & 14 & 15 & 16 & 17 & 18 & 19 & 20 & 21 \\
\hline
01 & 50 & 55 & 60 & 65 & 70 & 75 & 80 & 85 & 90 & 95 & - & - & - & - & - & - & - & - & - & - & - \\
02 & 45 & 50 & 55 & 60 & 65 & 70 & 75 & 80 & 85 & 90 & 95 & - & - & - & - & - & - & - & - & - & - \\
03 & 40 & 45 & 50 & 55 & 60 & 65 & 70 & 75 & 80 & 85 & 90 & 95 & - & - & - & - & - & - & - & - & - \\
04 & 35 & 40 & 45 & 50 & 55 & 60 & 65 & 70 & 75 & 80 & 85 & 90 & 95 & - & - & - & - & - & - & - & - \\
05 & 30 & 35 & 40 & 45 & 50 & 55 & 60 & 65 & 70 & 75 & 80 & 85 & 90 & 95 & - & - & - & - & - & - & - \\
06 & 25 & 30 & 35 & 40 & 45 & 50 & 55 & 60 & 65 & 70 & 75 & 80 & 85 & 90 & 95 & - & - & - & - & - & - \\
07 & 20 & 25 & 30 & 35 & 40 & 45 & 50 & 55 & 60 & 65 & 70 & 75 & 80 & 85 & 90 & 95 & - & - & - & - & - \\
08 & 15 & 20 & 25 & 30 & 35 & 40 & 45 & 50 & 55 & 60 & 65 & 70 & 75 & 80 & 85 & 90 & 95 & - & - & - & - \\
09 & 10 & 15 & 20 & 25 & 30 & 35 & 40 & 45 & 50 & 55 & 60 & 65 & 70 & 75 & 80 & 85 & 90 & 95 & - & - & - \\
10 & 05 & 10 & 15 & 20 & 25 & 30 & 35 & 40 & 45 & 50 & 55 & 60 & 65 & 70 & 75 & 80 & 85 & 90 & 95 & - & - \\
11 & - & 05 & 10 & 15 & 20 & 25 & 30 & 35 & 40 & 45 & 50 & 55 & 60 & 65 & 70 & 75 & 80 & 85 & 90 & 95 & - \\
12 & - & - & 05 & 10 & 15 & 20 & 25 & 30 & 35 & 40 & 45 & 50 & 55 & 60 & 65 & 70 & 75 & 80 & 85 & 90 & 95 \\
13 & - & - & - & 05 & 10 & 15 & 20 & 25 & 30 & 35 & 40 & 45 & 50 & 55 & 60 & 65 & 70 & 75 & 80 & 85 & 90 \\
14 & - & - & - & - & 05 & 10 & 15 & 20 & 25 & 30 & 35 & 40 & 45 & 50 & 55 & 60 & 65 & 70 & 75 & 80 & 85 \\
15 & - & - & - & - & - & 05 & 10 & 15 & 20 & 25 & 30 & 35 & 40 & 45 & 50 & 55 & 60 & 65 & 70 & 75 & 80 \\
16 & - & - & - & - & - & - & 05 & 10 & 15 & 20 & 25 & 30 & 35 & 40 & 45 & 50 & 55 & 60 & 65 & 70 & 75 \\
17 & - & - & - & - & - & - & - & 05 & 10 & 15 & 20 & 25 & 30 & 35 & 40 & 45 & 50 & 55 & 60 & 65 & 70 \\
18 & - & - & - & - & - & - & - & - & 05 & 10 & 15 & 20 & 25 & 30 & 35 & 40 & 45 & 50 & 55 & 60 & 65 \\
19 & - & - & - & - & - & - & - & - & - & 05 & 10 & 15 & 20 & 25 & 30 & 35 & 40 & 45 & 50 & 55 & 60 \\
20 & - & - & - & - & - & - & - & - & - & - & 05 & 10 & 15 & 20 & 25 & 30 & 35 & 40 & 45 & 50 & 55 \\
21 & - & - & - & - & - & - & - & - & - & - & - & 05 & 10 & 15 & 20 & 25 & 30 & 35 & 40 & 45 & 50 \\
\hline
\end{tabular}
\end{center}

\section*{LE COMBAT}

\section*{ROUND}

Le round sert à découper le combat de manière à ce qu'il soit jouable.

\section*{INITIATIVE}

Au début de chaque round, chaque participant au combat lance 1d6 et y ajoute la valeur de sa DEXtérité.

\section*{ORDRE DES ATTAQUES}

\begin{itemize}
    \item Première passe : les personnages (PJ et PNJ) qui ont des armes à feu ou des armes à projectiles prêtes à servir agissent en premier.
    \item Deuxième passe : les personnages qui n'ont pas encore agi.
    \item Troisième passe : les personnages qui ont des armes à feu pouvant tirer deux fois par round.
\end{itemize}

\section*{ESQUIVER ET PARER}

\begin{itemize}
    \item L'Esquive est une compétence à part entière.
    \item La parade n'est pas vraiment une compétence, c'est seulement une autre manière d'utiliser les compétences d'attaque.
\end{itemize}

\section*{BLESSER L'ADVERSAIRE}

Il suffit de réussir un jet sous la compétence qui régit l'arme utilisée.

\section*{PROTECTIONS}

\begin{itemize}
    \item Les armures.
    \item Les boucliers.
    \item Le terrain.
\end{itemize}

\section*{RÉUSSITE CRITIQUE ET MALADRESSE EN COMBAT}

\begin{itemize}
    \item Réussite critique : les dommages infligés par l'attaque sont plus importants que prévu.
    \item Maladresse : les conséquences sont variées.
\end{itemize}

\section*{COMBAT AU CORPS À CORPS}

\begin{itemize}
    \item Combat à mains nues : dépend de deux compétences : Bagarre et Lutte.
    \item Combat aux armes blanches : dépend de deux compétences : armes de mêlée et armes d'hast.
\end{itemize}

\section*{COMBAT À DISTANCE}

\begin{itemize}
    \item Armes de jet.
    \item Armes à feu.
\end{itemize}

\section*{TABLE DES ARMURES}

\begin{center}
\begin{longtable}{|l|c|}
\hline
Type & Protection \\
\hline
Cuir souple (blouson) & 1 \\
Cuir rigide (armure) & 2 \\
Cuir et métal & 4 \\
Cotte de mailles (gilet pare-balles) & 6 \\
Armure de plaques & 8 \\
\hline
\end{longtable}
\end{center}

\section*{TABLE DES ARMES D'HAST ET DE MÊLÉE}

\begin{center}
\begin{longtable}{|l|l|c|}
\hline
Arme & Compétence & Dommages \\
\hline
Dague / poignard & Mêlée & 1d3+2 \\
Épée / rapière & Mêlée & 1d6+2 \\
Épieu / pique & Hast & 2d6 \\
Espadon & Mêlée & 2d6 \\
Gourdin & Mêlée & 1d6 \\
Hache de bataille & Mêlée & 2d6 \\
Hachette & Mêlée & 1d6+1 \\
Hallebarde & Hast & 3d6 \\
Javelot / lance courte & Hast & 1d6+1 \\
Masse / fléau d'armes & Mêlée & 1d6+2 \\
Coup de poing, de tête, etc. & Bagarre & 1d3 \\
Bouclier & Celle de l'arme principale & - \\
\hline
\end{longtable}
\end{center}

\section*{TABLE DES ARMES DE TIR ET DE LANCER}

\begin{center}
\begin{longtable}{|l|l|c|c|c|}
\hline
Arme & Compétence & Portée efficace & Portée maximale & Dommages \\
\hline
Arbalète & Tir & 20 m & 50 m & 2d6 \\
Arc & Tir & 50 m & 150 m & 1d6+2 \\
Dague & Lancer & FOR en m & FOR x 2 m & 1d3+2 \\
Fronde & Tir & 50 m & 100 m & 1d6 \\
Javelot & Lancer & FOR en m & FOR x 3 m & 1d6+1 \\
\hline
\end{longtable}
\end{center}

\end{document}

