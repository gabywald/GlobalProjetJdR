\documentclass[11pt,twoside,a4paper]{article}
% http://www-h.eng.cam.ac.uk/help/tpl/textprocessing/latex_maths+pix/node6.html symboles de math
% http://fr.wikibooks.org/wiki/Programmation_LaTeX Programmation latex (wikibook)
%=========================== En-Tete =================================
%--- Insertion de paquetages (optionnel) ---
\usepackage[french]{babel}   % pour dire que le texte est en fran{\'e}ais
\usepackage{a4}	             % pour la taille   
\usepackage[T1]{fontenc}     % pour les font postscript
\usepackage{epsfig}          % pour gerer les images
%\usepackage{psfig}
\usepackage{amsmath, amsthm} % tres bon mode mathematique
\usepackage{amsfonts,amssymb}% permet la definition des ensembles
\usepackage{float}           % pour le placement des figure
\usepackage{verbatim}

\usepackage{longtable} % pour les tableaux de plusieurs pages

\usepackage[table]{xcolor} % couleur de fond des cellules de tableaux

\usepackage{lastpage}

\usepackage{multirow}

\usepackage{multicol} % pour {\'e}crire dans certaines zones en colonnes : \begin{multicols}{nb colonnes}...\end{multicols} 

% \usepackage[top=1.5cm, bottom=1.5cm, left=1.5cm, right=1.5cm]{geometry}
% gauche, haut, droite, bas, entete, ente2txt, pied, txt2pied
\usepackage{vmargin}
\setmarginsrb{1.00cm}{1.00cm}{1.00cm}{1.00cm}{15pt}{3pt}{50pt}{20pt}

\usepackage{lscape} % changement orientation page
%\usepackage{frbib} % enlever pour obtenir references en anglais
% --- style de page (pour les en-tete) ---
\pagestyle{empty}

\def\txtTITLE{Le Killer SuperGang} %%%%% !! TITRE !! %%%%%
\def\imgCORNER{\includegraphics[width=0.25cm]{../../../../../imgGraphics/logos/glider/logo-glider.png}}

%--- Definitions de nouvelles couleurs ---
\definecolor{verylightgrey}{rgb}{0.8,0.8,0.8}
\definecolor{verylightgray}{gray}{0.80}
\definecolor{lightgrey}{rgb}{0.6,0.6,0.6}
\definecolor{lightgray}{gray}{0.6}

% % % en-tete et pieds de page configurables : fancyhdr.sty

% http://www.trustonme.net/didactels/250.html

% http://ww3.ac-poitiers.fr/math/tex/pratique/entete/entete.htm
% http://www.ctan.org/tex-archive/macros/latex/contrib/fancyhdr/fancyhdr.pdf
%% \usepackage{fancyhdr}
%% \pagestyle{fancy}
%% % \newcommand{\chaptermark}[1]{\markboth{#1}{}}
%% % \newcommand{\sectionmark}[1]{\markright{\thesection\ #1}}
%% \fancyhf{}
%% \fancyhead[LE,RO]{\bfseries\thepage}
%% \fancyhead[LO]{\bfseries\rightmark}
%% \fancyhead[RE]{\bfseries\leftmark}
%% \fancyfoot[LE]{\thepage /\pageref{LastPage} \hfill
%% 	\scriptsize{\txtTITLE} % TITLE
%% \hfill \imgCORNER }
%% \fancyfoot[RO]{\imgCORNER \hfill
%% 	\scriptsize{\txtTITLE} % TITLE
%% \hfill \thepage /\pageref{LastPage}}
%% \renewcommand{\headrulewidth}{0.5pt}
%% \renewcommand{\footrulewidth}{0.5pt}
%% \addtolength{\headheight}{0.5pt}
%% % \fancypagestyle{plain}{
%% 	% \fancyhead{}
%% 	% \renewcommand{\headrulewidth}{0pt}
%% % }

\usepackage{lettrine}
\usepackage{fancybox}

\title{\txtTITLE}
\date{ --- }

%============================= Corps =================================
\begin{document}

\setlength\parindent{0pt} % \noindent for all document

\begin{center}
	\textbf{\huge \txtTITLE}
\end{center}

%% \textbf{\Large Le Killer SuperGang}~\\

\textbf{\scriptsize Voici un sc{\'e}nario pour \emph{Killer}, qui d{\'e}veloppe davantage le c{\^o}t{\'e} r{\^o}le et ne fait pas de l'assassinat l'unique m{\'e}thode d'action. Paradoxal me direz-vous ? Peut-{\^e}tre, mais le killer SuperGang s'est jou{\'e} il y a quelques ann{\'e}es dans diff{\'e}rentes universit{\'e}s parisiennes et a eu un franc succ{\`e}s. } %% ~\\
\texttt{\scriptsize{(Originellement Publi{\'e} dans : Casus Belli Hors s{\'e}rie n 4 -- Le Jeu de R{\^o}le Grandeur Nature (GN), 1992 -- Excelsior Publications)}}~\\

\begin{multicols*}{2}
	\small
	
Le contexte est inspir{\'e} du jeu de plateau SuperGang (\emph{Ludod{\'e}lire}) : des gangs de mafiosi veulent prendre  le contr{\^o}le de la ville et envoient les divers membres  de leur gang {\`a} l'action. Chaque {\'e}quipe doit comprendre (en plus des hommes de main) un parrain, deux lieutenants, et un avocat (un arbitre) qui sont d{\'e}termin{\'e}s comme suit. Les membres du gang choisissent parmi eux trois lieutenants. Ces derniers s'isolent et d{\'e}terminent qui des trois sera le parrain. De cette fa\c{c}on, les autres membres du gang ne savent pas qui est le v{\'e}ritable chef du gang (pour {\'e}viter son {\'e}limination trop rapide en cas de << vampage >>... nous y reviendrons). Buen s{\^u}r, ils mettent l'avocat de la famille au courant de leur d{\'e}cision (c'est l'un des arbitres du jeu apr{\`e}s tout). Le parrain prend possession d'un capital de d{\'e}part de 2500 \$ et d{\'e}cide ensuite des d{\'e}penses du gang, pour lesquelles il est seul d{\'e}cisionnaire. Il suivra les conseils de ses lieutenants, dont il a tout int{\'e}r{\^e}t {\`a} m{\'e}nager la susceptibilit{\'e} car eux connaissent sa v{\'e}ritable fonction qui doit rester secr{\`e}te. Au yeux de tous, il doit {\^e}tre un lieutenant comme les autres. Il ach{\`e}te donc des << fonctions >> pour les membres de son gang (ainsi que pour lui-m{\^e}me et ses lieutenants s'il le d{\'e}sire). Il en existe de trois sortes : 
\begin{itemize}
	\item[$\bullet$] \textbf{Tueur} (co{\^u}t d'achat : 400 \$). C'est le seul {\`a} avoir droit de tuer, quelle que soit la m{\'e}thode utilis{\'e}e. 
	\item[$\bullet$] \textbf{Vamp} (co{\^u}t d'achat : 200 \$). Elle (ou il) s{\'e}duit les membres de sexe oppos{\'e} des autres gangs. 
	\item[$\bullet$] \textbf{Trafiquant} (co{\^u}t d'achat : 200 \$). Il rapporte de l'argent {\`a} la famille. 
\end{itemize}~\\

\textbf{\large Les << r{\^o}les >>}~\\

Le \textbf{\emph{tueur}} dispose de tous les moyens propres au killer pour occire ses ennemis, mais le pistolet {\`a} fl{\'e}chettes reste une valeur traditionnelle, une << valeur >> dont il se s{\'e}pare rarement. ~\\

La \textbf{\emph{vamp}} s{\'e}duit les membres des autres gangs en jouant {\`a} une esp{\`e}ce de << ni oui, ni non >>. Elle [ou il] doit faire prononcer un mot pr{\'e}cis {\`a} sa victime (choisissez des mots relativement difficiles, comme << compteur {\`a} gaz >> ou << encyclop{\'e}die Diderot >>). L'arme de chaque vamp est un mot diff{\'e}rent, inscrit sur la carte de fonction que lui a remise un lieutenant. Quand elle [ou il] r{\'e}ussit {\`a} le faire prononcer au membre d'un autre gang, celui-ci tombe << amoureux >> et ob{\'e}it {\`a} toutes ses instructions pendant trois jours. Il met tous les pouvoirs de sa fonction {\`a} sa disposition (meurtre pour un tueur, gain des transactions pour un trafiquant) et se comporte comme un tra{\^i}tre au sein de sa propre famille. ~\\

\vfill
\columnbreak

Le \textbf{\emph{trafiquant}} doit rencontrer les avocats des autres familles qui jouent le r{\^o}le de courtiers {\`a} la bourse du crime. Pour une partie o{\`u} s'affrontent cinq gangs, mettez une dizaine de marchandises en jeu. Chaque avocat ach{\`e}te une marchandise {\`a} un pris donn{\'e} et le revend {\`a} un autre prix. Les trafiquants doivent donc deviner quel avocat ach{\`e}te et vend au meilleur prix. Les marchandises sont repr{\'e}sent{\'e}es par des cartes ou des objets... Si vous voulez compliquer la partie, les cours de la bourse peuvent changer toutes les semaines, tous les quinze jours ou tous les mois. Pr{\'e}parez-les {\`a} l'avance pour qu'ils soient {\'e}quilibr{\'e}s entre les diff{\'e}rents avocats. ~\\

Les membres d'un gang ne peuvent pas cumuler deux fonctions (attention, lieutenant et parrain sont pas des fonctions mais des positions au sein de la famille). Ils peuvent n'en avoir aucune et continuer {\`a} jouer sans pouvoir particulier. Cela arrive souvent quand le gang est trop appauvri. ~\\

\textbf{\large D{\'e}roulement de la partie}~\\

L'avocat fournit au parrain les cartes de fonction qu'il a achet{\'e}es, pour qu'il les remette aux membres de son gang (lui-m{\^e}me ou par l'interm{\'e}diaire d'un lieutenant). Les joueurs pourront ainsi prouver la validit{\'e} de leurs actions aux autres participants. L'avocat est le seul {\`a} pouvoir valider les achats de fonctions du parrain, qui lui remettra l'argent correspondant. C'est lui aussi qui joue le courtier en bourse pour les trafiquants des autres gangs (jamais pour ceux de la famille dont il est l'arbitre, ce serait trop facile). ~\\

Quand un personnage est {\'e}limin{\'e}, son assassin r{\'e}cup{\`e}re son argent et sa carte de fonction (plus {\'e}ventuellement sa carte de lieutenant)). Si c'est un trafiquant, il s'empare aussi de ses cartes de marchandises... Il rend la carte de fonction {\`a} l'avocat de la famille mais il peut disposer librement des autres {\'e}l{\'e}ments : les remettre aux lieutenants de son gang ou les garder pour lui. ~\\

Quand le parrain est {\'e}limin{\'e}, le gang se r{\'e}unit et {\'e}lit un nouveau lieutenant. Les trois lieutenants choisissent ensuite secr{\`e}tement le parrain parmi eux. [Quand un lieutenant est {\'e}limin{\'e}, l'{\'e}lection d'un nouveau lieutenant peut {\^e}tre faite. ]~\\

La fin du jeu peut {\^e}tre d{\'e}termin{\'e}e de plusieurs fa\c{c}ons : accumulation d'une somme donn{\'e}e, {\'e}limination des autres gangs, gang le plus riche {\`a} la fin d'une dur{\'e}e d{\'e}termin{\'e}e... Vous pouvez aussi d{\'e}terminer des victoires individuelles plut{\^o}t que des victoires par gang. ~\\

\end{multicols*}

%% \begin{center} \rule{0.80\textwidth}{0.05cm} \end{center}

\end{document}
