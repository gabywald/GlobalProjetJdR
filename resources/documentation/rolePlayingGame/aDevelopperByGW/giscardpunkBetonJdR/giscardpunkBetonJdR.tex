\documentclass[11pt,twoside,a4paper]{article}
% http://www-h.eng.cam.ac.uk/help/tpl/textprocessing/latex_maths+pix/node6.html symboles de math
% http://fr.wikibooks.org/wiki/Programmation_LaTeX Programmation latex (wikibook)
%=========================== En-Tete =================================
%--- Insertion de paquetages (optionnel) ---
\usepackage[french]{babel}   % pour dire que le texte est en fran{\'e}ais
\usepackage{a4}	             % pour la taille   
\usepackage[T1]{fontenc}     % pour les font postscript
\usepackage{epsfig}          % pour gerer les images

\usepackage{amsmath, amsthm} % tres bon mode mathematique
\usepackage{amsfonts,amssymb}% permet la definition des ensembles
\usepackage{float}           % pour le placement des figure
\usepackage{verbatim}

\usepackage{longtable} % pour les tableaux de plusieurs pages

\usepackage[table]{xcolor} % couleur de fond des cellules de tableaux

\usepackage{tikz}

\usepackage{lastpage}

% \usepackage[top=1.5cm, bottom=1.5cm, left=1.5cm, right=1.5cm]{geometry}
% gauche, haut, droite, bas, entete, ente2txt, pied, txt2pied
\usepackage{vmargin}
\setmarginsrb{1.0cm}{1.0cm}{1.0cm}{1.0cm}{15pt}{3pt}{60pt}{3pt}

\usepackage{lscape} % changement orientation page
%\usepackage{frbib} % enlever pour obtenir references en anglais
% --- style de page (pour les en-tete) ---
\pagestyle{headings}

% % % en-tete et pieds de page configurables : fancyhdr.sty

% http://www.trustonme.net/didactels/250.html

% http://ww3.ac-poitiers.fr/math/tex/pratique/entete/entete.htm
% http://www.ctan.org/tex-archive/macros/latex/contrib/fancyhdr/fancyhdr.pdf
\usepackage{fancyhdr}
\pagestyle{fancy}
% \newcommand{\chaptermark}[1]{\markboth{#1}{}}
% \newcommand{\sectionmark}[1]{\markright{\thesection\ #1}}
\fancyhf{}
\fancyhead[LE,RO]{\bfseries\thepage}
\fancyhead[LO]{\bfseries\rightmark}
\fancyhead[RE]{\bfseries\leftmark}
\fancyfoot[LE]{\thepage /\pageref{LastPage} \hfill
	GiscardPunk -- \emph{Un JdR de B{\'e}ton}
\hfill \includegraphics[width=0.5cm]{../../../../imgGraphics/logos/glider/logo-glider.png} }
\fancyfoot[RO]{\includegraphics[width=0.5cm]{../../../../imgGraphics/logos/glider/logo-glider.png} \hfill
	\emph{Un JdR de B{\'e}ton} -- GiscardPunk
\hfill \thepage /\pageref{LastPage}}
\renewcommand{\headrulewidth}{0.5pt}
\renewcommand{\footrulewidth}{0.5pt}
\addtolength{\headheight}{0.5pt}
\fancypagestyle{plain}{
	\fancyhead{}
	\renewcommand{\headrulewidth}{0pt}
}

\renewcommand{\headrulewidth}{0.25pt}
\renewcommand{\footrulewidth}{0.5pt}
%% \setlength{\headheight}{85pt}
% \addtolength{\headheight}{0.5pt}
% \fancypagestyle{plain}{
% 	\fancyhead{}
% 	\fancyfoot{}
% 	\renewcommand{\headrulewidth}{0pt}
% }

%--- Definitions de nouvelles commandes ---
\newcommand{\N}{\mathbb{N}} % les entiers naturels


%--- Pour le titre ---
\def\maketitle{%
	\begin{center}
		\begin{tabular}[c]{c|c}
			\textsc{\textbf{...}}~\\[\baselineskip]~\\[\baselineskip]
			\emph{\textbf{Version \today}}~\\[\baselineskip]~\\[\baselineskip]
			\emph{\textbf{JdR inspir{\'e} des Villes nouvelles Futuristes des ann{\'e}es 1970's et 1980's}}~\\[\baselineskip]~\\[\baselineskip]
			\textsc{Gaby Wald \& Amael Assour}~\\[\baselineskip]~\\[\baselineskip]
			& 
			\includegraphics[width=3cm]{../../../../imgGraphics/logos/glider/logo-glider.png}~\\[\baselineskip]
		\end{tabular}
		% \\ \hline
		 	% % if more than one logo
			% \includegraphics[width=5cm]{../../../../imgGraphics/logos/glider/logo-glider.png}
		% \\ \hline
		% \end{tabular}
			~\\[\baselineskip]~\\[\baselineskip]
			\Huge{1980's GiscardPunk}~\\[\baselineskip]
			\Large{Un JdR de B{\'e}ton}~\\[\baselineskip]
		
		~\\[\baselineskip]
		~\\[\baselineskip]
	%% \large{
	%% 	\textsc{\textbf{...}}
	%% 	~\\[\baselineskip]
	%% 	<<titre personne>> : \texttt{Anne ONYME}~\\[\baselineskip]
	%% 	<<titre personne>> : \texttt{Jocelyn CONNU}~\\[\baselineskip]
	%% 	~\\[\baselineskip]
	%% 	\textit{Pr{\'e}cisions du contexte de r{\'e}daction de l'article}
	%% }

	\end{center}

}%

%--- Pour le glossaire --- a defaut de \makeglossary ou d'utilisation d'index latex

\definecolor{verylightgray}{rgb}{0.8,0.8,0.8}
\def\makeglossaire{%
	\begin{center}

	\begin{tabular}{|>{\columncolor{verylightgray}} p{0.20\textwidth}|p{0.70\textwidth}|}

		\hline

		\textbf{Terme} & 
			\begin{tabular}{p{0.68\textwidth}}
				D{\'e}finition du terme
			 \end{tabular} \\
		\hline


	\end{tabular}

\end{center}

}%

%============================= Corps =================================
\begin{document}
%ecrire le titre...
\maketitle
\setcounter{page}{0}
\thispagestyle{empty}
\clearpage

\setcounter{page}{0}
\thispagestyle{empty}

~\\

\clearpage
\setcounter{page}{0}
\thispagestyle{empty}
% ecrire la table des mati{\'e}res...
\tableofcontents

% \clearpage

% \setcounter{page}{0}
% \thispagestyle{empty}

% ecrire la table des figures et celle des tableaux

%% \setcounter{page}{0}
%% \thispagestyle{empty}
%% ~\\ \rule{10cm}{1mm}~\\
%% \listoffigures
%% ~\\ \rule{10cm}{1mm}~\\
%% \listoftables
\clearpage

\setcounter{page}{1}

\section*{Introduction\markboth{Introduction}{Introduction}}

\addcontentsline{toc}{section}{Introduction}


Tout d'abord, quelques images d'illustration :~\\

\begin{minipage}{0.45\linewidth}
	%% \begin{figure}[ht]
	\includegraphics[width=0.95\textwidth]{../../../../imgGraphics/giscardpunk/planVoisinAlainBublex/2-e1596900022719.png}~\\
	\emph{La maquette de Paris par le Corbusier}
	%% \caption{La maquette de Paris par le Corbusier}
	%% \end{figure}
\end{minipage}
\begin{minipage}{0.20\linewidth}\end{minipage}
\begin{minipage}{0.45\linewidth}
	%% \begin{figure}[ht]
	\includegraphics[width=0.95\textwidth]{../../../../imgGraphics/giscardpunk/planVoisinAlainBublex/8.png}~\\
	\emph{Une Vision Glorieuse d'un Pass{\'e} R{\'e}volu ?}
	%% \caption{Une Vision Glorieuse d'un Pass{\'e} R{\'e}volu ?}
	%% \end{figure}
\end{minipage}~\\~\\

\rule{0.33\linewidth}{0.5mm}~\\

\begin{minipage}{0.30\linewidth}
	%% \begin{figure}[ht]
	\includegraphics[width=0.95\textwidth]{../../../../imgGraphics/giscardpunk/planVoisinAlainBublex/c-a_hk681_barbes-870x652.jpg}~\\
	%% \caption{Ruelle 1 : Barb{\`e}s}
	%% \end{figure}
\end{minipage}
\begin{minipage}{0.01\linewidth}\end{minipage}
\begin{minipage}{0.30\linewidth}
	%% \begin{figure}[ht]
	\includegraphics[width=0.95\textwidth]{../../../../imgGraphics/giscardpunk/planVoisinAlainBublex/c-a_hk826_montaigne-870x652.jpg}~\\
	%% \caption{Ruelle 2 : Avenue Montaigne}
	%% \end{figure}
\end{minipage}
\begin{minipage}{0.01\linewidth}\end{minipage}
\begin{minipage}{0.30\linewidth}
	%% \begin{figure}[ht]
	\includegraphics[width=0.95\textwidth]{../../../../imgGraphics/giscardpunk/planVoisinAlainBublex/c-a_hk831_6e_b-870x652.jpg}~\\
	%% \caption{Ruelle 3 : o{\`u} ?}
	%% \end{figure}
\end{minipage}~\\

\rule{0.66\linewidth}{0.5mm}~\\

%% \begin{itemize}
%% 	\item[$\checkmark$] This will give a checkmark bullet.
%% 	\item[$\square$] This will give a hollow square bullet.
%% 	\item[$\blacksquare$] This will give a filled square bullet.
%% 	\item[$\bigstar$] This will give you a bigstar bullet.
%% \end{itemize}

\begin{itemize}
	\item[$\bigstar$] Ruelles et avenues de b{\'e}ton {\'e}clair{\'e}es par des n{\'e}ons clignotants ?
	\item[$\bigstar$] Des publicit{\'e}s pour des produits polluants (voitures, {\'e}lectrom{\'e}nager, cosm{\'e}tiques...) ?
	\item[$\bigstar$] Une technologie omnipr{\'e}sente d{\'e}pass{\'e}e, tout de m{\^e}me r{\'e}trofuturiste ?
\end{itemize}~\\

Ce n'est pas CyberPunk 2020 ni m{\^e}me CyberPunk 2077 : c'est du \textbf{GiscardPunk} !~\\

Bienvenue dans un R{\'e}troPunk qui fonctionne au Diesel, au Minitel et {\`a} la poussi{\`e}re de B{\'e}ton !~\\

\rule{1.00\linewidth}{0.5mm}~\\

\clearpage


\section{Dans le B{\'e}ton}

\subsection{C'{\'e}tait le Futur}

<<La France de Demain !>>~\\

C'{\'e}tait avant tout dans la continuit{\'e} des plans de construction centrales nucl{\'e}aires, la construction des villes nouvelles en b{\'e}ton ({\'E}vry-Courcouronnes, Cr{\'e}teil, Champs-Sur-Marne...) aux alentours de Paris et centr{\'e}e sur l'automobile roulant au Diesel, des autoroutes pour les relier, des voies ferr{\'e}es rapides avec le TGV (<<Train Grande Vitesse>>) et ses airs de Concorde Orange restant au sol, le RER (<<R{\'e}seau Express R{\'e}gional>>) et ses tunnels carrel{\'e}s aux tonalit{\'e}s color{\'e}es blanches-oranges-rouges de la d{\'e}cennie 1970, des services t{\'e}l{\'e}matiques et de t{\'e}l{\'e}texte qui pointent leur nez avec le Minitel : vous pouvez r{\'e}server vos billets AirFrance ou SNCF, ou acheter "en ligne" avec des services comme LaRedoute ou Les3Suisses et vous faire livrer chez vous !~\\

Du CyberPunk 50 ans trop t{\^o}t : avec des technologies d'avance comme le TGV, le Minitel, les villes nouvelles !~\\ 

\c{C}a, c'{\'e}tait l'avenir, c'{\'e}tait avant !

\subsection{C'est le Pr{\'e}sent}

Le b{\'e}ton a un peu vieillit, l'essence co{\^u}te plus cher, les routes ne sont pas entretenues, les constructions non plus. La publicit{\'e} recouvre tout ceci afin de maintenir l'illusion d'une {\'e}conomie de march{\'e} florissante !~\\

Pas ou peu d'innovations ici : c'est une "soci{\'e}t{\'e} glorieuse" qui continue dans sa lanc{\'e}e ! L'administration et les entreprises s'occupent de vos emplois et de besoins en biens de consommation !~\\

Qui se soucie r{\'e}ellement d'aller au-del{\`a} de la maintenance ? <<Cela a toujours fonctionn{\'e} ainsi !>>

\subsection{Ce sera le Pass{\'e}}

Des personnes s'organisent pour que cela change, {\`a} leur niveau !~\\

\begin{itemize}
	\item[$\bigstar$] De la Magye ?
	\item[$\bigstar$] Des Nouvelles Technologies ?
	\item[$\bigstar$] Des Organisations Alternatives ?
\end{itemize}~\\

{\`A} vous de voir !

\section{Sc{\'e}narios}

\begin{itemize}
	\item[$\bigstar$] S.O.S. Bonheur Saison 1 : la Compagnie d'Analyse G{\'e}n{\'e}rale recrute ! Une occasion en opr de sortir du ch{\^o}mage ! Mais que fait r{\'e}ellement cette soci{\'e}t{\'e} dont personne n'a jamais vu les dirigeants, {\`a} part le Directeur des Ressources Humaines; Qui donc occupe les {\'e}tages sup{\'e}rieurs au num{\'e}ro 11 ?
	\item[$\bigstar$] S.O.S. Bonheur Saison 1 : des cr{\'e}ations non homologu{\'e}es se distribuent sous le manteau, au d{\'e}triment des v{\'e}ritables artistes reconnus par l'{\'E}tat ; qui diffuse ces \oe uvres, et pourquoi ?  
	\item[$\bigstar$] S.O.S. Bonheur Saison 1 : ...
\end{itemize}~\\

\clearpage

\section{Annexes}

\subsection{Glossaire}

%% voir en haut : sous la definition du titre : glossaire

\makeglossaire

\subsection{Bibliographie\markboth{Bibliographie}{Bibliographie}}

\addcontentsline{toc}{section}{Bibliographie}
\nocite{*}
%toutes references biblio : 6 lettres + 2 chiffres
\bibliography{giscardpunkbetonJdR}
\bibliographystyle{frplain} % plain or frplain
\end{document}
