\documentclass[11pt,twoside,a4paper]{article}
% http://www-h.eng.cam.ac.uk/help/tpl/textprocessing/latex_maths+pix/node6.html symboles de math
% http://fr.wikibooks.org/wiki/Programmation_LaTeX Programmation latex (wikibook)
%=========================== En-Tete =================================
%--- Insertion de paquetages (optionnel) ---
\usepackage[french]{babel}   % pour dire que le texte est en fran{\'e}ais
\usepackage{a4}	             % pour la taille   
\usepackage[T1]{fontenc}     % pour les font postscript
\usepackage{epsfig}          % pour gerer les images

\usepackage{amsmath, amsthm} % tres bon mode mathematique
\usepackage{amsfonts,amssymb}% permet la definition des ensembles
\usepackage{float}           % pour le placement des figure
\usepackage{verbatim}

\usepackage{longtable} % pour les tableaux de plusieurs pages

\usepackage[table]{xcolor} % couleur de fond des cellules de tableaux

\usepackage{tikz}
\usepackage{geometry}

\usepackage{lastpage}

% \usepackage[top=1.5cm, bottom=1.5cm, left=1.5cm, right=1.5cm]{geometry}
% gauche, haut, droite, bas, entete, ente2txt, pied, txt2pied
\usepackage{vmargin}
\setmarginsrb{1.0cm}{1.0cm}{1.0cm}{1.0cm}{15pt}{3pt}{60pt}{3pt}

\usepackage{lscape} % changement orientation page
%\usepackage{frbib} % enlever pour obtenir references en anglais
% --- style de page (pour les en-tete) ---
\pagestyle{headings}

% % % en-tete et pieds de page configurables : fancyhdr.sty

% http://www.trustonme.net/didactels/250.html

% http://ww3.ac-poitiers.fr/math/tex/pratique/entete/entete.htm
% http://www.ctan.org/tex-archive/macros/latex/contrib/fancyhdr/fancyhdr.pdf
\usepackage{fancyhdr}
\pagestyle{fancy}
% \newcommand{\chaptermark}[1]{\markboth{#1}{}}
% \newcommand{\sectionmark}[1]{\markright{\thesection\ #1}}
\fancyhf{}
\fancyhead[LE,RO]{\bfseries\thepage}
\fancyhead[LO]{\bfseries\rightmark}
\fancyhead[RE]{\bfseries\leftmark}
\fancyfoot[LE]{\thepage /\pageref{LastPage} \hfill
	Agence Townsend -- \emph{Un JdR Girly}
\hfill \includegraphics[width=0.5cm]{../../../../imgGraphics/logos/glider/logo-glider.png} }
\fancyfoot[RO]{\includegraphics[width=0.5cm]{../../../../imgGraphics/logos/glider/logo-glider.png} \hfill
	\emph{Un JdR Girly} -- Agence Townsend
\hfill \thepage /\pageref{LastPage}}
\renewcommand{\headrulewidth}{0.5pt}
\renewcommand{\footrulewidth}{0.5pt}
\addtolength{\headheight}{0.5pt}
\fancypagestyle{plain}{
	\fancyhead{}
	\renewcommand{\headrulewidth}{0pt}
}

\renewcommand{\headrulewidth}{0.25pt}
\renewcommand{\footrulewidth}{0.5pt}
%% \setlength{\headheight}{85pt}
% \addtolength{\headheight}{0.5pt}
% \fancypagestyle{plain}{
% 	\fancyhead{}
% 	\fancyfoot{}
% 	\renewcommand{\headrulewidth}{0pt}
% }

%--- Definitions de nouvelles commandes ---
\newcommand{\N}{\mathbb{N}} % les entiers naturels


\def\makeLogoAndTXT{%
	\begin{tikzpicture}[scale=0.5]
		%% color 		white, black, red, green, blue, cyan, magenta, yellow 
		%% thickness 	ultra thin, very thin, thin, thick, very thick, ultra thick 
		
		%% Wings && Lines within wings
		% \draw [very thin, gray] ( 0, 0) grid ( 6, 6);
		\draw[thick] ( 1, 0) -- ( 6, 5) -- ( 2, 5);
		\draw[thick] ( 1, 0) -- ( 1, 4) -- ( 5, 4);
		\draw[thick] ( 1, 2) -- ( 4, 5);
		\draw[thick] ( 1, 2) -- ( 2, 1);
		\draw[thick] ( 2, 1) -- ( 2, 3) -- ( 4, 3);
		\draw[thick] ( 2, 3) -- ( 3, 2);
		
		\draw[thick] (-1, 0) -- (-6, 5) -- (-2, 5);
		\draw[thick] (-1, 0) -- (-1, 4) -- (-5, 4);
		\draw[thick] (-1, 2) -- (-4, 5);
		\draw[thick] (-1, 2) -- (-2, 1);
		\draw[thick] (-2, 1) -- (-2, 3) -- (-4, 3);
		\draw[thick] (-2, 3) -- (-3, 2);
	
		%% TXT
		\node at ( 0,-2) {\textbf{TOWNSEND AGENCY}};
		\node at ( 0,-3) {\large LOS ANGELES};
		\node at ( 0,-4) {\small BEIJING \textbullet\ BERLIN \textbullet\ LONDON};
		\node at ( 0,-5) {\small MUMBAI \textbullet\ NAIROBI \textbullet\ SAO PAULO};
	\end{tikzpicture}
}%


%--- Pour le titre ---
\def\maketitle{%
	\begin{center}
		\begin{tabular}[c]{c|c}
			\textsc{\textbf{...}}~\\[\baselineskip]~\\[\baselineskip]
			\emph{\textbf{Version \today}}~\\[\baselineskip]~\\[\baselineskip]
			\emph{\textbf{JdR inspir{\'e} des films \emph{Charlie's Angels}}}~\\[\baselineskip]~\\[\baselineskip]
			\textsc{Gaby Wald \& Amael Assour}~\\[\baselineskip]~\\[\baselineskip]
			& 
			\makeLogoAndTXT~\\[\baselineskip]
		\end{tabular}
		% \\ \hline
		 	% % if more than one logo
			% \includegraphics[width=5cm]{../../../../imgGraphics/logos/glider/logo-glider.png}
		% \\ \hline
		% \end{tabular}
			~\\[\baselineskip]~\\[\baselineskip]
			\Huge{Agence Townsend}~\\[\baselineskip]
			\Large{Un JdR Girly}~\\[\baselineskip]
		
		~\\[\baselineskip]
		~\\[\baselineskip]
	%% \large{
	%% 	\textsc{\textbf{...}}
	%% 	~\\[\baselineskip]
	%% 	<<titre personne>> : \texttt{Anne ONYME}~\\[\baselineskip]
	%% 	<<titre personne>> : \texttt{Jocelyn CONNU}~\\[\baselineskip]
	%% 	~\\[\baselineskip]
	%% 	\textit{Pr{\'e}cisions du contexte de r{\'e}daction de l'article}
	%% }

	\end{center}

}%

%--- Pour le glossaire --- a defaut de \makeglossary ou d'utilisation d'index latex

\definecolor{verylightgray}{rgb}{0.8,0.8,0.8}
\def\makeglossaire{%
	\begin{center}

	\begin{tabular}{|>{\columncolor{verylightgray}} p{0.20\textwidth}|p{0.70\textwidth}|}

		\hline

		\textbf{Terme} & 
			\begin{tabular}{p{0.68\textwidth}}
				Définition du terme
			 \end{tabular} \\
		\hline


	\end{tabular}

\end{center}

}%

%============================= Corps =================================
\begin{document}
%ecrire le titre...
\maketitle
\setcounter{page}{0}
\thispagestyle{empty}
\clearpage

\setcounter{page}{0}
\thispagestyle{empty}

~\\

\clearpage
\setcounter{page}{0}
\thispagestyle{empty}
% ecrire la table des mati{\'e}res...
\tableofcontents

% \clearpage

% \setcounter{page}{0}
% \thispagestyle{empty}

% ecrire la table des figures et celle des tableaux

%% \setcounter{page}{0}
%% \thispagestyle{empty}
%% ~\\ \rule{10cm}{1mm}~\\
%% \listoffigures
%% ~\\ \rule{10cm}{1mm}~\\
%% \listoftables
\clearpage

\setcounter{page}{1}

\section*{Introduction\markboth{Introduction}{Introduction}}

\addcontentsline{toc}{section}{Introduction}


[...]~\\

\rule{10cm}{0.5mm}~\\

\begin{minipage}{0.5\linewidth}

	\makeLogoAndTXT
	
\end{minipage}\hfill\begin{minipage}{0.5\linewidth}

	\begin{tikzpicture}[scale=0.5]
		%% TXT
		\node at ( 0, 0 ) {
			\begin{tabular}{c}
				BEIJING \\ 
				BERLIN \\ 
				LONDON \\ 
				LOS ANGELES \\ 
				MUMBAI \\ 
				NAIROBI \\ 
				SAO PAULO \\ 
			\end{tabular}
		};
		\node at ( 0,-5) {\large TOWNSEND};
		\node at ( 0,-6) {\small AGENCY};
		
	\end{tikzpicture}

\end{minipage}


\clearpage

%% \section{Section}
%% \subsection{Sous-section}
%% \subsubsection{sous sous sous section}
%% [...]~\\
%% \subsubsection{sous sous sous section}
%% \subsection*{SousSectionNonNumerot}
%% \addcontentsline{toc}{section}{SousSectionNonNumerot}
%% \begin{table}[ht]
	%% \begin{center}
		%% \begin{tabular}{|p{0.1\textwidth}|p{0.7\textwidth}|}
		%% \hline
		%% \includegraphics[width=1cm]{img/logo_glider.png}
		%% & 
		%% \textbf{GLIDER} est le logo des hacker, symbole repris du jeu de la vie (cavalier). \\
		%% \hline
		%% \end{tabular}
	%% \end{center}
	%% \caption{Un tableau r{\'e}f{\'e}renc{\'e}}
	%% \label{tab:TabReference01}
%% \end{table}~\\
%% \clearpage
%% \subsection{Encore une sous-section}
%% \begin{figure}[H]
	%% \centerline {\epsfig {file=img/logo_glider.png,width=0.5\textwidth}}
	%% \caption{Une belle image}
	%% \label{fig:FigReference01}
%% \end{figure}
%% \clearpage

\section{Annexes}

\subsection{Glossaire}

%% voir en haut : sous la definition du titre : glossaire

\makeglossaire

\clearpage

\subsection{Bibliographie\markboth{Bibliographie}{Bibliographie}}

\addcontentsline{toc}{section}{Bibliographie}
\nocite{*}
%toutes references biblio : 6 lettres + 2 chiffres
\bibliography{agenceTownsendJdR}
\bibliographystyle{frplain} % plain or frplain
\end{document}
