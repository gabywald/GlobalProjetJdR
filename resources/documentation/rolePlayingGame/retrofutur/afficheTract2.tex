\documentclass[11pt,twoside,a4paper]{article}
% http://www-h.eng.cam.ac.uk/help/tpl/textprocessing/latex_maths+pix/node6.html symboles de math
% http://fr.wikibooks.org/wiki/Programmation_LaTeX Programmation latex (wikibook)
%=========================== En-Tete =================================
%--- Insertion de paquetages (optionnel) ---
\usepackage[french]{babel}   % pour dire que le texte est en francais
\usepackage{a4}	             % pour la taille   
\usepackage[T1]{fontenc}     % pour les font postscript
\usepackage{epsfig}          % pour gerer les images
%\usepackage{psfig}
\usepackage{amsmath, amsthm} % tres bon mode mathematique
\usepackage{amsfonts,amssymb}% permet la definition des ensembles
\usepackage{float}           % pour le placement des figure
\usepackage{verbatim}

\usepackage{longtable} % pour les tableaux de plusieurs pages

\usepackage[table]{xcolor} % couleur de fond des cellules de tableaux

\usepackage{lscape} % changement orientation page

\usepackage{lastpage}

% \usepackage[top=1.5cm, bottom=1.5cm, left=1.5cm, right=1.5cm]{geometry}
% gauche, haut, droite, bas, entete, ente2txt, pied, txt2pied
\usepackage{vmargin}
\setmarginsrb{1.0cm}{1.0cm}{1.0cm}{1.0cm}{15pt}{1pt}{15pt}{1pt}

%\usepackage{frbib} % enlever pour obtenir references en anglais
% --- style de page (pour les en-tete) ---
\pagestyle{empty}

%% \usepackage[all,center]{background}
%% \SetBgContents{Confidential}
%% \SetBgAngle{45}
%% \SetBgScale{15}
%% \SetBgOpacity{0.1}

%% \usepackage[printwatermark=true,allpages=true,
%% 	fontfamily=pag,color=gray!25,textmark=DRAFT,
%% 	angle=45,scale=0.8,xcoord=0,ycoord=0]{xwatermark}

%--- Definitions de nouvelles commandes ---
\newcommand{\N}{\mathbb{N}} % les entiers naturels

%--- Definitions de nouvelles couleurs ---
\definecolor{verylightgrey}{rgb}{0.8,0.8,0.8}
\definecolor{verylightgray}{gray}{0.80}
\definecolor{lightgrey}{rgb}{0.6,0.6,0.6}
\definecolor{lightgray}{gray}{0.6}

\usepackage{multicol}


%============================= Corps =================================
\begin{document}

\pagecolor{lightgray}

\begin{center}
{\Huge CITOYENS DU MONDE ENTIER}~\\~\\
%% {\Huge CITOYENS}~\\~\\
%% {\huge DU MONDE ENTIER ! }~\\~\\
{\large \textsc{Depuis quatre-vingts ans, nous vivons sous le r{\`e}gne du mensonge. }}~\\~\\
\end{center}

\begin{multicols}{2}

\setlength{\parindent}{0pt} % \noindent

Depuis le jour du Contact, tout ce qui fait la grandeur de l'humanit{\'e} a {\'e}t{\'e} mis entre parenth{\`e}ses pour atteindre "le plus noble des objectifs", nous pr{\'e}parer {\`a} rencontrer les {\'E}trangers. ~\\

Nous nous sommes pr{\'e}par{\'e}s. ~\\
Nous avons attendu. ~\\
Et attendu encore. ~\\

Petit {\`a} petit, alors que nous avions les yeux fix{\'e}s sur l'avenir, tout nous a {\'e}t{\'e} arrach{\'e}. ~\\ 
Notre libert{\'e} d'action a {\'e}t{\'e} r{\'e}duite. ~\\
Notre libert{\'e} de pens{\'e}e a {\'e}t{\'e} rogn{\'e}e, limit{\'e}e, supprim{\'e}e. ~\\ %% ~\\~\\

Et nous attendions toujours. ~\\

Jusqu'{\`a} ce jour de 1948 o{\`u} tout a bascul{\'e} {\`a} nouveau. Lors de sa fameuse conf{\'e}rence de Princeton, le professeur Einstein a propos{\'e} un nouveau mod{\`e}le math{\'e}matique de l'univers, un mod{\`e}le qui contredit le dogme officiel. ~\\ 

L'attente est termin{\'e}e. ~\\

Les {\'E}trangers n'existent pas. ~\\
Nous sommes seuls dans l'univers. ~\\
Seuls face aux agences. ~\\
Face {\`a} un syst{\`e}me qui n'a aucune l{\'e}gitimit{\'e}. ~\\ 
Un syst{\`e}me inhumain au sens premier du mot. ~\\

Pour les agences, l'Humain est un moyen et non plus une finalit{\'e}. Elles g{\`e}rent l'humanit{\'e} comme une fourmili{\`e}re, et ont oubli{\'e} ce que c'est que d'{\^e}tre un individu. ~\\ 

La barbarie des agences est une r{\'e}alit{\'e} tangible, mesurable, d{\'e}montrable. ~\\
Tout homme qui ne se laisse pas poser des oeill{\`e}res peut s'en rendre compte. ~\\
O{\`u} sont pass{\'e}s les milliers de "contestataires" arr{\'e}t{\'e}s lors des Grandes Gr{\`e}ves ? ~\\ 
{\`A} l'{\'e}poque, il a {\'e}t{\'e} question de "mesures de r{\'e}habilitation transitoires", "d'internement temporaire". ~\\ 
O{\`u} sont-ils ? Ils ont disparus. Effac{\'e}s comme s'ils n'avaient jamais exist{\'e}. ~\\ 

Bien s{\^u}r, ouvrir les yeux exige un effort. ~\\
Contre la r{\'e}alit{\'e}, les agences alignent des arm{\'e}es de propagandistes, charg{\'e}es de repeindre en rose un r{\`e}ve qui tourne de plus en plus rapidement au cauchemar. ~\\ 
Faites cet effort. ~\\ 
Regardez la r{\'e}alit{\'e} en face. ~\\

Le syst{\`e}me nous montre les plus excentriques des {\'e}gar{\'e}s et nous dit "c'est nous ou eux". ~\\
Il nous montre les plus cruels des terroristes et nous dit "nous sommes la seule alternative {\`a} ces monstres". ~\\
Et m{\^e}me les hommes de bonne volont{\'e} se laissent prendre, parce qu'il "n'existe pas d'autre choix". ~\\

Cr{\'e}ez cette alternative. ~\\
Changez le monde. ~\\

Nous sommes des millions {\`a} refuser la tyrannie des agences sans oser agir. ~\\
Si nous nous organisons, rien ne pourra nous arr{\^e}ter. ~\\

Organisons la r{\'e}sistance. ~\\
Ouvrons les yeux de la population. ~\\
Apprenons-en le plus possible sur les plans des agences, et d{\'e}jouons-les. ~\\
Nous devons mener la lutte par tous les moyens possibles. ~\\

Toutefois, je demande {\`a} tous ceux qui nous rejoignent de ne jamais perdre de vue que la violence est un moyen. Elle n'est ni une fin en soi, ni un id{\'e}al. ~\\ 

Nous luttons contre un syst{\`e}me. Pour cela, il nous faut porter des coups {\`a} ses repr{\'e}sentants. Mais nous nous interdisons d'avoir recours {\`a} la terreur et de frapper aveugl{\'e}ment des objectifs civils. Nous sommes des soldats sans uniformes, pas des bouchers. ~\\ 

Commencez le combat. ~\\
Nous ne sommes pas loin. ~\\
Nous vous trouverons. ~\\
Nous vous aiderons. ~\\
Ensemble, nous arracherons la victoire. ~\\

Chaque instant hors de la tutelle des agences est un succ{\`e}s. ~\\
Chaque acte qui ne rentre pas dans leurs sch{\'e}ma est un triomphe. ~\\
Chaque pens{\'e}e libre est une victoire. ~\\

Catacombes de Paris, ~\\
20 novembre 1950 ~\\~\\

{\large COMIT{\'E} DE LIB{\'E}RATION MONDIAL}~\\

\end{multicols} 

\end{document}
