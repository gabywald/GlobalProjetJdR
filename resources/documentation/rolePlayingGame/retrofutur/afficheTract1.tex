\documentclass[11pt,twoside,a4paper]{article}
% http://www-h.eng.cam.ac.uk/help/tpl/textprocessing/latex_maths+pix/node6.html symboles de math
% http://fr.wikibooks.org/wiki/Programmation_LaTeX Programmation latex (wikibook)
%=========================== En-Tete =================================
%--- Insertion de paquetages (optionnel) ---
\usepackage[french]{babel}   % pour dire que le texte est en francais
\usepackage{a4}	             % pour la taille   
\usepackage[T1]{fontenc}     % pour les font postscript
\usepackage{epsfig}          % pour gerer les images
%\usepackage{psfig}
\usepackage{amsmath, amsthm} % tres bon mode mathematique
\usepackage{amsfonts,amssymb}% permet la definition des ensembles
\usepackage{float}           % pour le placement des figure
\usepackage{verbatim}

\usepackage{longtable} % pour les tableaux de plusieurs pages

\usepackage[table]{xcolor} % couleur de fond des cellules de tableaux

\usepackage{lscape} % changement orientation page

\usepackage{lastpage}

% \usepackage[top=1.5cm, bottom=1.5cm, left=1.5cm, right=1.5cm]{geometry}
% gauche, haut, droite, bas, entete, ente2txt, pied, txt2pied
\usepackage{vmargin}
\setmarginsrb{1.5cm}{1.5cm}{1.5cm}{1.5cm}{64pt}{1pt}{15pt}{1pt}

%\usepackage{frbib} % enlever pour obtenir references en anglais
% --- style de page (pour les en-tete) ---
\pagestyle{empty}

%% \usepackage[all,center]{background}
%% \SetBgContents{Confidential}
%% \SetBgAngle{45}
%% \SetBgScale{15}
%% \SetBgOpacity{0.1}

%% \usepackage[printwatermark=true,allpages=true,
%% 	fontfamily=pag,color=gray!25,textmark=DRAFT,
%% 	angle=45,scale=0.8,xcoord=0,ycoord=0]{xwatermark}

%--- Definitions de nouvelles commandes ---
\newcommand{\N}{\mathbb{N}} % les entiers naturels

%--- Definitions de nouvelles couleurs ---
\definecolor{verylightgrey}{rgb}{0.8,0.8,0.8}
\definecolor{verylightgray}{gray}{0.80}
\definecolor{lightgrey}{rgb}{0.6,0.6,0.6}
\definecolor{lightgray}{gray}{0.6}

\definecolor{yellowPassed}{RGB}{237, 230, 212}

\definecolor{titlered}{RGB}{110, 17, 10}


%============================= Corps =================================
\begin{document}

\begin{center} \begin{bfseries} \begin{scshape}

\pagecolor{yellowPassed}


\textcolor{titlered}{\LARGE Citoyens du monde entier, }
\textcolor{titlered}{\LARGE depuis quatre-vingt ans, }
\textcolor{titlered}{\LARGE nous vivons sous le r{\`e}gne du mensonge ! }~\\

~\\

Depuis le jour du contact, tout ce qui fait la grandeur de l'humanit{\'e}~\\
a {\'e}t{\'e} mis entre parenth{\`e}ses pour atteindre <<le plus noble des objectifs>> :~\\
la rencontre des {\'E}trangers.~\\

~\\

Nous nous sommes pr{\'e}par{\'e}s, ~\\nous avons construit les titanopoles, ~\\
nous avons attendus.~\\

~\\

Mais petit {\`a} petit, tout nous a {\'e}t{\'e} arrach{\'e}, ~\\
notre libert{\'e} d'action et de penser.~\\ 

~\\

Puis il y a deux ans tout a bascul{\'e}. Lors de la fameuse conf{\'e}rence de Princeton, ~\\
le Pr. Einstein a pr{\'e}sent{\'e} un nouveau mod{\`e}le math{\'e}matique de l'univers, ~\\
qui contredit le dogme officiel.~\\ 

~\\

Un mod{\`e}le qui prouve l'inexistence des {\'E}trangers ! ~\\
Nous sommes seuls dans l'univers, seuls face aux agences, ~\\
face {\`a} un syst{\`e}me sans l{\'e}gitimit{\'e}.~\\

~\\

\textcolor{titlered}{\LARGE REGARDEZ LA R{\'E}ALIT{\'E} EN FACE ~\\
REJOIGNEZ LA R{\'E}SISTANCE ! }~\\

~\\

\includegraphics{img/resistance_logo_greys_yellow.jpg}~\\

~\\

Vous {\^e}tes des millions {\`a} {\^e}tre conscients de la tyrannie ~\\
des agences sans oser agir. ~\\
Allions nous et rien ne pourra nous arr{\^e}ter.~\\

~\\

R{\'e}sistez, nous ne sommes pas loin. ~\\
Nous vous trouverons et ensemble nous arracherons la victoire.~\\

~\\

<<\textcolor{titlered}{Chaque instant hors de la tutelle des agences est un succ{\`e}s. ~\\
Chaque acte qui ne rentre pas dans leurs sch{\'e}mas est un triomphe. ~\\
Chaque pens{\'e}e libre est une victoire. }>>~\\

~\\

Catacombes de Paris, 20 Novembre 1950, ~\\
Jean Moulin~\\


\end{scshape} \end{bfseries} \end{center}

\end{document}
