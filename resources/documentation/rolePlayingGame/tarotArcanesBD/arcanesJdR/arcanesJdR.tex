\documentclass[11pt,twoside,a4paper]{article}
% http://www-h.eng.cam.ac.uk/help/tpl/textprocessing/latex_maths+pix/node6.html symboles de math
% http://fr.wikibooks.org/wiki/Programmation_LaTeX Programmation latex (wikibook)
%=========================== En-Tete =================================
%--- Insertion de paquetages (optionnel) ---
\usepackage[french]{babel}   % pour dire que le texte est en fran{\'e}ais
\usepackage{a4}	             % pour la taille   
\usepackage[T1]{fontenc}     % pour les font postscript
\usepackage{epsfig}          % pour gerer les images

\usepackage{amsmath, amsthm} % tres bon mode mathematique
\usepackage{amsfonts,amssymb}% permet la definition des ensembles
\usepackage{float}           % pour le placement des figure
\usepackage{verbatim}

\usepackage{longtable} % pour les tableaux de plusieurs pages

\usepackage[table]{xcolor} % couleur de fond des cellules de tableaux

\usepackage{tikz}

\usepackage{lastpage}

% \usepackage[top=1.5cm, bottom=1.5cm, left=1.5cm, right=1.5cm]{geometry}
% gauche, haut, droite, bas, entete, ente2txt, pied, txt2pied
\usepackage{vmargin}
\setmarginsrb{1.0cm}{1.0cm}{1.0cm}{1.0cm}{15pt}{3pt}{60pt}{3pt}

\usepackage{lscape} % changement orientation page
%\usepackage{frbib} % enlever pour obtenir references en anglais
% --- style de page (pour les en-tete) ---
\pagestyle{headings}

% % % en-tete et pieds de page configurables : fancyhdr.sty

% http://www.trustonme.net/didactels/250.html

% http://ww3.ac-poitiers.fr/math/tex/pratique/entete/entete.htm
% http://www.ctan.org/tex-archive/macros/latex/contrib/fancyhdr/fancyhdr.pdf
\usepackage{fancyhdr}
\pagestyle{fancy}
% \newcommand{\chaptermark}[1]{\markboth{#1}{}}
% \newcommand{\sectionmark}[1]{\markright{\thesection\ #1}}
\fancyhf{}
\fancyhead[LE,RO]{\bfseries\thepage}
\fancyhead[LO]{\bfseries\rightmark}
\fancyhead[RE]{\bfseries\leftmark}
\fancyfoot[LE]{\thepage /\pageref{LastPage} \hfill
	Arcanes -- \emph{Un JdR de Tarot}
\hfill \includegraphics[width=0.5cm]{../../../../imgGraphics/logos/glider/logo-glider.png} }
\fancyfoot[RO]{\includegraphics[width=0.5cm]{../../../../imgGraphics/logos/glider/logo-glider.png} \hfill
	\emph{Un JdR de Tarot} -- Arcanes
\hfill \thepage /\pageref{LastPage}}
\renewcommand{\headrulewidth}{0.5pt}
\renewcommand{\footrulewidth}{0.5pt}
\addtolength{\headheight}{0.5pt}
\fancypagestyle{plain}{
	\fancyhead{}
	\renewcommand{\headrulewidth}{0pt}
}

\renewcommand{\headrulewidth}{0.25pt}
\renewcommand{\footrulewidth}{0.5pt}
%% \setlength{\headheight}{85pt}
% \addtolength{\headheight}{0.5pt}
% \fancypagestyle{plain}{
% 	\fancyhead{}
% 	\fancyfoot{}
% 	\renewcommand{\headrulewidth}{0pt}
% }

%--- Definitions de nouvelles commandes ---
\newcommand{\N}{\mathbb{N}} % les entiers naturels


%--- Pour le titre ---
\def\maketitle{%
	\begin{center}
		\begin{tabular}[c]{c|c}
			\textsc{\textbf{...}}~\\[\baselineskip]~\\[\baselineskip]
			\emph{\textbf{Version \today}}~\\[\baselineskip]~\\[\baselineskip]
			\emph{\textbf{JdR inspir{\'e} des BD \emph{Arcanes} / \emph{Arcane Majeure} / \emph{L'Histoire Secr{\`e}te}}}~\\[\baselineskip]~\\[\baselineskip]
			\textsc{Gaby Wald \& Amael Assour}~\\[\baselineskip]~\\[\baselineskip]
			& 
			\includegraphics[width=3cm]{../../../../imgGraphics/logos/glider/logo-glider.png}~\\[\baselineskip]
		\end{tabular}
		% \\ \hline
		 	% % if more than one logo
			% \includegraphics[width=5cm]{../../../../imgGraphics/logos/glider/logo-glider.png}
		% \\ \hline
		% \end{tabular}
			~\\[\baselineskip]~\\[\baselineskip]
			\Huge{Arcanes}~\\[\baselineskip]
			\Large{Un JdR de tarot}~\\[\baselineskip]
		
		~\\[\baselineskip]
		~\\[\baselineskip]
	%% \large{
	%% 	\textsc{\textbf{...}}
	%% 	~\\[\baselineskip]
	%% 	<<titre personne>> : \texttt{Anne ONYME}~\\[\baselineskip]
	%% 	<<titre personne>> : \texttt{Jocelyn CONNU}~\\[\baselineskip]
	%% 	~\\[\baselineskip]
	%% 	\textit{Pr{\'e}cisions du contexte de r{\'e}daction de l'article}
	%% }

	\end{center}

}%

%--- Pour le glossaire --- a defaut de \makeglossary ou d'utilisation d'index latex

\definecolor{verylightgray}{rgb}{0.8,0.8,0.8}
\def\makeglossaire{%
	\begin{center}

	\begin{tabular}{|>{\columncolor{verylightgray}} p{0.20\textwidth}|p{0.70\textwidth}|}

		\hline

		\textbf{Carte RS} & 
			\begin{tabular}{p{0.68\textwidth}}
				Au d{\'e}but des ann{\'e}es 80, on trouva un catalyseur de l'effet RS en l'esp{\`e}ce de curieuses cartes, dont on ignore {\`a} peu pr{\`e}s tout. les premi{\`e}res cartes firent leur apparition dans la r{\'e}gion de Seattle, USA. On cherche encore {\`a} percer l'identit{\'e} du ou des cr{\'e}ateurs des premiers paquets. 
				La C.I.A., comprenant rapidement l'immense potentiel tactique que recelaient les cartes, cr{\'e}a Stargate, un groupe de scientifiques charg{\'e} d'anakyser le ph{\'e}nom{\`e}ne et de former des agents capables de le contr{\^o}ler : les agents RS.
			 \end{tabular} \\
		\hline
		\textbf{Agent RS} & 
			\begin{tabular}{p{0.68\textwidth}}
				D{\`e}s le d{\'e}but, les premiers agents RS remport{\`e}rent des succ{\`e}s impresionnants, accomplissant des missions alors que toutes les chances de r{\'e}ussites {\'e}taient contre eux. 
				Les diff{\'e}rentes op{\'e}rations Stargate sont encore class{\'e}es "secret d{\'e}fense" mais les chercheurs pointent le doigt sur un certain nombre d'{\'e}v{\`e}nements, comme la mort du Pr{\'e}sident panam{\'e}en en 82 ou le putsch rat{\'e} des g{\'e}n{\'e}raux russes. 
				Certains affirment que les cartes existent de toute {\'e}ternit{\'e}, et que nos cartes {\`a} jouer en seraient des formes simplifi{\'e}es. 
			\end{tabular} \\
		\hline
		\textbf{Grade Stargate} & 
			\begin{tabular}{p{0.68\textwidth}}
				On a pris l'habitude de classer les agents Stargate {\`a} partir des figures des cartes {\`a} jouer classiques : 6, 8, 9, 10, Valet, Cavalier, etc. Ainsi, un 6 n'est sans doute capable que d'influer sur de petits {\'e}v{\`e}nements ayant peu de possibilit{\'e}s statistiques, comme le lancement d'une pi{\`e}ce de monnaie par exemple. Les grades sup{\'e}rieurs peuvent agir beaucoup plus en profondeur sur la trame du hasard.
			\end{tabular} \\
		\hline
%% 		\textbf{DEA (Drug Enforcement Administration)} & 
%% 			\begin{tabular}{p{0.68\textwidth}}
%% 				Agence gouvernementale am{\'e}ricaine charg{\'e}e de la lutte anti-drogue dans le monde.
%% 			\end{tabular} \\
%% 		\hline  
%% 		\textbf{Nagual} & 
%% 			\begin{tabular}{p{0.68\textwidth}}
%% 				Esprit, double de l'esprit humain.
%% 			\end{tabular} \\
%% 		\hline 
		\textbf{Comit{\'e} Magic} & 
			\begin{tabular}{p{0.68\textwidth}}
				Organisme Ultra-secret mis en place par les Alli{\'e}s durant la Seconde Guerre Mondiale pour {\'e}tudier, utiliser les cartes et manipuler le hasard. Il n'avait de compte {\`a} rendre qu'{\`a} Churchill et Roosevelt. Magic fut dissous en 1946. 
				Magic travailla en {\'e}troite collaboration avec Ultra, le programma de d{\'e}codage des messages secrets allemands [Enigma] puis avec Manhattan, le programme de r{\'e}alisation de la premi{\`e}re bombe atomique am{\'e}ricaine {\`a} Los Alamos. 
				Les bras arm{\'e}s de Magic {\'e}taient, du c{\^o}t{\'e} anglais : le SOE (Special Operation Executive), et c{\^o}t{\'e} am{\'e}ricain l'OSS (anc{\^e}tre de la CIA), comme Magic, ces deux entit{\'e}s furent dissoutes au lendemain de la guerre, du moins officiellement.
			\end{tabular} \\
		\hline 
		\textbf{Fondation Vane} & 
			\begin{tabular}{p{0.68\textwidth}}
				Agence gouvernementale am{\'e}ricaine qui prit la suite de Magic. Les anglais refus{\`e}rent d'en faire partie. Au d{\'e}part, il s'agissait d'un simple programme d{\'e}tude des ph{\'e}nom{\`e}nes RS.
			\end{tabular} \\
		\hline 
		\textbf{Stargate} & 
			\begin{tabular}{p{0.68\textwidth}}
				Successeur de la Fondation Vane dans les ann{\'e}es 60. Dot{\'e} d'un budget beaucoup plus important, et plus tourn{\'e} vers l'utilisation des cartes que vers leur {\'e}tude.
				Programme de la C.I.A. qui cherche depuis plusieurs ann{\'e}es {\`a} percer le secret des cartes RS et qui, dans le m{\^e}me temps, les utilise pour former des agents RS capables de manipuler le hasard. Ces agents, tr{\`e}s rares et tr{\`e}s fragiles psychologiquement ont un immense potentiel tactique. Ils peuvent, {\`a} la lettre, accomplir des miracles, gagner contre toutes les probailit{\'e}s, mlais qui sont-ils vraiment, et en derni{\`e}re analyse, qui les contr{\^o}le ?
			\end{tabular} \\
		\hline 
		\textbf{R{\'e}tro-Synchronicit{\'e} (RS)} & 
			\begin{tabular}{p{0.68\textwidth}}
				Th{\'e}orie selon laquelle le hasard n'existe pas. Elle suppose qu'on peut orienter les {\'e}v{\`e}nements et le futur {\`a} condition de disposer d'un catalyseur qui d{\'e}clenche et acc{\'e}l{\`e}re la r{\'e}action. ces catalyseurs seraient en l'occurrence des cartes qui seraient les anc{\^e}tres de nos jeux de cartes.
			\end{tabular} \\
		\hline
		\textbf{Singularit{\'e} ou effet RS} & 
			\begin{tabular}{p{0.68\textwidth}}
				Moment o{\`u} l'action des cartes se d{\'e}clenche. Le sujet est alors devant les possibilit{\'e}s offertes par le futur, {\`a} lui de choisir les plus int{\'e}ressantes. Plus une singularit{\'e} est profonde, plus les possibilit{\'e}s sont nombreuses et plongent plus loin dans le futur. 
			\end{tabular} \\
		\hline

	\end{tabular}

\end{center}

}%

%============================= Corps =================================
\begin{document}
%ecrire le titre...
\maketitle
\setcounter{page}{0}
\thispagestyle{empty}
\clearpage

\setcounter{page}{0}
\thispagestyle{empty}

~\\

\clearpage
\setcounter{page}{0}
\thispagestyle{empty}
% ecrire la table des mati{\'e}res...
\tableofcontents

% \clearpage

% \setcounter{page}{0}
% \thispagestyle{empty}

% ecrire la table des figures et celle des tableaux

%% \setcounter{page}{0}
%% \thispagestyle{empty}
%% ~\\ \rule{10cm}{1mm}~\\
%% \listoffigures
%% ~\\ \rule{10cm}{1mm}~\\
%% \listoftables
\clearpage

\setcounter{page}{1}

\section*{Introduction\markboth{Introduction}{Introduction}}

\addcontentsline{toc}{section}{Introduction}


[...]~\\

\rule{10cm}{0.5mm}~\\


\clearpage

\section{{\'E}l{\'e}ments de base, id{\'e}es...}

\subsection{R{\^o}le des cartes}

{ \setlength\parindent{0pt} \small
\begin{tabular}[c]{c c c c c c}
	\rowcolor{verylightgray}
	Famille			&	Symbole			&	{\'E}l{\'e}ment	& 	Type / Groupe... ou Id{\'e}al (?)		& Famille "CyberPunk" (?)		& "Pouvoirs"			\\
	Arcanes			&	Nb et Nom		&	-- selon --		&	Magie (?) ...							& Magie (?) ; Hasard ... 		& ... 					\\
	B{\^a}tons			&	Carreaux		&	Air				&	Sociabilit{\'e}, Charisme ; Sant{\'e}	& Gangs, Nomades...				& Social / Relation 	\\
	Coupes			&	C\oe ur			&	Eau				& 	Vie, Gu{\'e}rison ; Ing{\'e}nierie		& Tech, MedTech... 				& Mat{\'e}riel			\\
	Deniers			&	Piques			&	Terre			& 	Protection, R{\'e}flexion ; Fortune		& Solos, Fixers, Hackers...		& Virtuel / Intellect	\\
	{\'E}p{\'e}es	&	Tr{\`e}fles		&	Feu				& 	Pouvoir, Unification ; Action			& Corporations, {\'E}tats...	& Temporel / Religieux	\\
	
	%% Erlin => Deniers => Tr{\`e}fles
	%% Dyo => Coupes => Coeurs
	%% Reka => B{\^a}tons => Piques
	%% Aker => {\'E}p{\'e}es => Carreaux
\end{tabular} }~\\

%% Tirage des PJ : 5 {\`a} 10 cartes classiques ; d{\'e}termination puissance dans chaque caract{\'e}ristique, si t{\^e}te : cela permet de d{\'e}terminer une appartenance {\`a} une maison (ou deux ?) et un niveau au sein ce celle(s)-ci. ~\\

Pour le tirage des personnages : une carte pour d{\'e}finir l'arch{\'e}type (parmi les atouts / arcanes majeures), et une autre pour d{\'e}finir son statut et son appartenance (dans les arcanes mineures, {\`a} partir du 8 inclu) ; ceci est not{\'e} sur la fiche de personnage, visible ou pas des autres joueurs. Ces cartes sont remises dans le jeu. Cette carte d'arcane mineure est jouable constamment en plus par le joueur (porteur permanent) au prix du d{\'e}p{\^o}t de sa main compl{\`e}te (acte h{\'e}ro{\"i}que).~\\  

Pendant une partie, main de 5 cartes pour chaque joueur : utilisation d'un jeu de tarot complet (une campagne ?) ou un jeu classique de 54 cartes (ou deux, selon les besoins ou la variante utilis{\'e}e), symbolisant jeu(x) construit(s) "dans le jeu" (intra-di{\'e}g{\'e}tique) "{\`a} la demande", ou apparus voire d{\'e}couverts selon les circonstances... %% ~\\

\subsection{R{\'e}solution des actions des PJ / Phase d'Action}

\textbf{Rappel / R{\`e}gle d'or : une action triviale r{\'e}ussit toujours !} {\`A} moins qu'une action d'un autre joueur / personnage ne rende cela pas trivial. L'initiative est r{\'e}solue par le tirage d'une carte dans un jeu sp{\'e}cifique m{\'e}lang{\'e} ({\`a} moins que le premier intervenant ne prenne les autres par surprise) : cela peut {\^e}tre dans la pile de cartes restantes ou une r{\'e}serve de cartes du MJ ; en cas d'{\'e}aglit{\'e} entre valeurs faciales, respecter l'ordre classique ou le niveau du personnage, ou un nouveau tirage.~\\

Chaque joueur dispose d'une main de cinq (5) cartes, renouvelable au fur et {\`a} mesure de son utilisation apr{\`e}s chaque usage d'une ou pluseurs cartes dans une pile de tirage d{\'e}di{\'e}e, en compl{\'e}tion {\`a} 5 (sauf sp{\'e}cificit{\'e}s).~\\
Une action se r{\'e}sout par le d{\'e}p{\^o}t d'une ou plusieurs cartes par chaque joueur pour son PJ, le d{\'e}compte du "score" s'effectue de la fa\c{c}on suivante (hors Atouts / Arcanes) : 
\begin{itemize}
	\item Valeur faciale des cartes li{\'e}es au type ou au groupe utilis{\'e} (de l'as qui vaut 1 au 14 pour la plus haute t{\^e}te) ; 
	\item Un demi point par autres cartes non li{\'e}es au type ou au groupe utilis{\'e} ; 
	\item Un Joker (ou Excuse) vaut toujours 7 ; 
	\item Dans les combinaisons sp{\'e}ciales (paire, brelan, suite...), multipliez le score pr{\'e}c{\'e}demment obtenu par le nombre de cartes de la combinatoire ; 
	\item \emph{Optionnel : }Un Atout ou une Arcane (ou un Joker) permet une action sp{\'e}ciale (contr{\'e} par un Atout de valeur sup{\'e}rieure) : 
	\begin{itemize}
		\item Tirage de cartes suppl{\'e}mentaires (valeur de l'Atout divis{\'e} par trois (3), arrondi au sup{\'e}rieur ; ou 1d6) ; 
		\item Rejouer imm{\'e}diatement (apr{\`e}s compl{\'e}tion de la main) ; 
		\item Action h{\'e}ro{\"i}que (dont la r{\'e}ussite d{\'e}pend du reste des cartes jou{\'e}es avec celle-ci). 
	\end{itemize}
	\item \emph{Optionnel : }Utilisation d'une Carte Sp{\'e}ciale d{\'e}finissant le PJ : {\`a} d{\'e}finir {\`a} l'avance entre le MJ et le joueur (valeur de la carte), mise en jeu de la vie du PJ et : ou du pouvoir associ{\'e} {\`a} cette carte. 
\end{itemize} %% ~\\

Les cartes jou{\'e}es rejoignent une d{\'e}fausse en attendant d'{\^e}tre rem{\'e}lang{\'e}es pour cr{\'e}er une nouvelle pile de tirage quand celle-ci est vide. %% ~\\

\subsection{"Effet R{\'e}troSynchronicit{\'e}" (RS)} 

En dehors d'un Phase d'Action, on peut utiliser un "Effet R{\'e}troSynchronicit{\'e}" (RS) : un joueur pose une carte sur la table en pr{\'e}vision d'une utilisation future (avec une annotation ou un post-it) ; cela permet d'avoir un {\'e}l{\'e}ment utile par la suite et que le joueur peut ainsi faire appara{\^i}tre par la suite (v{\'e}hicule, {\'e}l{\'e}ment de d{\'e}cor...). Cette carte est ensuite int{\'e}gr{\'e}e {\`a} la pile de tirage : quand cette carte est tir{\'e}e, elle est forc{\'e}ment en faveur du joueur l'ayant d{\'e}pos{\'e}e (int{\'e}gration {\`a} sa main courante). Cela peut {\^e}tre utile selon puissance et typologie de la carte concern{\'e}e, et l'objet concern{\'e}.~\\

\clearpage

Une variante de cet "Effet RS" peut {\^e}tre utilis{\'e} de la fa\c{c}on suivante : selon des {\'e}l{\'e}ments pr{\'e}vus par le MJ ou propos{\'e}s par les joueurs (fixes : lieux et leurs propri{\'e}t{\'e}s ; mobiles : v{\'e}hicules) : l'utilisationn de la carte immobilis{\'e}e devant soi fait intervenir cet {\'e}l{\'e}ment au b{\'e}n{\'e}fice de la carte la plus puissante, sous r{\'e}serve que le lieu soit {\`a} nouveau parcouru ou que l'{\'e}l{\'e}ment mobile puisse arriver {\`a} l'endroit de l'action. Une fois cet {\'e}l{\'e}ment utilis{\'e}, la carte peut {\^e}tre d{\'e}fauss{\'e}e ou {\^e}tre laiss{\'e}e associ{\'e} {\`a} l'{\'e}l{\'e}ment. Lorsqu'elle est associ{\'e}e {\`a} un {\'e}l{\'e}ment, la carte est pos{\'e}e devant le joueur, annot{\'e}e (post-it) et une nouvelle carte est tir{\'e}e dans sa main. %%~\\

\subsection{Ressorts pour le MJ}

Quelques {\'e}l{\'e}ments {\`a} utiliser par le MJ comme ressorts d'action, {\`a} communiquer ou {\`a} faire d{\'e}couvrir : 
\begin{itemize}
	\item Les cartes, leurs utilisations et les porteurs de cartes sont rep{\'e}rables {\`a} distance par d'autres (adversaires ou non) : bruit blanc / color{\'e} perceptible, cartes qui chauffent (pour les joueurs)... ; 
	\item Certains mat{\'e}riaux peuvent servir d'isolants : eau, tissus (soie ou autres mat{\'e}riaux sp{\'e}cifiques)...
	\item Il est possible de marquer des lieux par des glyphes, de m{\^e}me pour les gens (tatouages) ou certains objets : cela peut servir de protection, dans un sens comme dans l'autre (bloquer {\`a} l'int{\'e}rieur ou {\`a} l'ext{\'e}rieur) ; 
	\item Lois physiques modifi{\'e}es en certains lieux (g{\'e}om{\'e}trie non euclidienne : on entre dans un b{\^a}timent, la sortie aboutie ailleurs...) ; 
	\item ... 
\end{itemize}%%~\\

\subsection{Variantes et modes de jeu (jeu en campagne)}

Ces variantes peuvent compl{\'e}ter ou remplacer les r{\`e}gles d{\'e}crites pr{\'e}c{\'e}dentes.~\\

Variante (jeu en campagne) : chaque joueur dispose au d{\'e}part de la campagne d'un jeu partiel ou complet (54 ou 78 cartes) et peut utiliser ses cartes comme bon lui semble ; les cartes tir{\'e}es pour d{\'e}finir son personnage sont consid{\'e}r{\'e}es comme maitresses et peuvent n'{\^e}tre actives qu'en certaines circonstances.~\\

Destructions de cartes ? Annotations sp{\'e}cifiques ? Par exemple les cartes du tirage du personnage du joueur ou tir{\'e}es pendant la partie ; si n{\'e}cessaire on subtitue par un carton solide (carte bristol) avec annotations. %% ~\\

\subsection{Modalit{\'e}s particuli{\`e}res : {\'e}poques, cultures...}

Modalit{\'e}s CyberPunk : magie limit{\'e} au monde virtuel (matrice / cyberespace) et classiquement reli{\'e} {\`a} la mythologie vaudoue (at autres {\'e}l{\'e}ments cultures Cara{\"i}bes). 
Relier {\`a} une autre mythologie dans un autre univers (de jeu) : romain, grec, sum{\'e}rien, {\'e}gyptien, nordique, inca... (NOTE : tableau de correspondances {\`a} faire !)~\\

Selon les {\'e}poques, les symboliques, les cultures... {\`A} d{\'e}finir {\'e}ventuellement par le groupe. ~\\

Renommage des Arcanes majeures ? Renommage des Figures des familles ?~\\

Antiquit{\'e} et Moyen-{\^A}ge (et {\'e}quivalents dans diverses cultures) : Aau, Air, Terre, Feu (voire d'autres {\'e}l{\'e}ments : Bois, M{\'e}tal...) d'o{\`u} l'id{\'e}e de diff{\'e}rentes familles et diff{\'e}rents regroupement d'arcanes. ~\\

Combinatoire avec Magie / Magye ou Technologie : comme le vaudou dans l'informatique (dans les suites de \emph{Neuromancien} notamment) ; {\'e}mergence d'une autre symbolique au choix du (de la) MJ et des joueurs / joueuses ? Comme des domaiesn {\`a} conqu{\'e}rir ou {\`a} se r{\'e}partir ? <<Terre, Mer, ciel, Cyberespace>> ?~\\

Diff{\'e}rents jeux possibles en m{\^e}me temps ?~\\

\subsection{Symboles et design / dessin des cartes}

N'importe quel jeu de cartes ou de tarot peut faire l'affaire ; ceux {\`a} petit prix peuvent faire l'affaire (selon le mode de jeu), ou de votre propre dessin. L'essentiel est de pouvoir les utiliser facilement pendant la ou les parties.~\\

\begin{minipage}[ht]{0.45\textwidth}
	FEU / AIR / EAU / TERRE
\end{minipage} \hfill \begin{minipage}[ht]{0.45\textwidth}
	\def\triangle{--++(120:1)--++(240:1)--cycle}
	\def\triangleReverse{--++(240:1)--cycle}
	\begin{tikzpicture}
		\tikz\draw[red,thick](0,0)\triangle; %% FEU
		
		\tikz\draw[blue,thick](2,0)\triangle (1,0.40)--(2,0.40); %% AIR
		
		\tikz\draw[orange,thick](3,1)--(4,1)\triangleReverse; %% EAU
		
		\tikz\draw[green,thick](5,1)--(6,1)\triangleReverse (5,0.60)--(6,0.60); %% TERRE
	\end{tikzpicture}
\end{minipage}

\clearpage

\section{Quelques notes et id{\'e}es}

Cette partie est l{\`a} pour vous donner une id{\'e}e de la structure d'un jeu de tarot et des cartes pr{\'e}sentes, leurs d{\'e}nominations, les concepts associ{\'e}s aux arcanes et lames. Beaucoup de jeux de tarots d{\'e}coratifs ou jouables sont en g{\'e}n{\'e}ral disponibles dans des magasins dits "{\'e}sot{\'e}riques" ou des magasins sp{\'e}cialis{\'e}s (jeux de cartes), avec des illustrations vari{\'e}es. L'id{\'e}e ici est d'en faire usage pour jouer d'une fa\c{c}on diff{\'e}rente et raconter une histoire interactive entre amis autour d'une table. %%~\\ 

\subsection{Tarot "classique"}

Les 78 cartes sont r{\'e}parties r{\'e}parties comme suit (trois fois sept plus un et quatre fois deux fois sept) :
\begin{minipage}[ht]{0.45\textwidth}
 	\begin{itemize}
	\item 22 arcanes majeures, num{\'e}rot{\'e}es de 0 {\`a} 21 inclus :~\newline
		\begin{tabular}[c]{ p{4.5cm} p{4.5cm} }
			0	Mat	(Excuse)			&	11	La Force			\\
			1	Le Bateleur	(Le Mage)	&	12	Le Pendu			\\
			2	La Papesse				&	13	La Mort				\\
			3	L'Imp{\'e}ratrice		&	14	La Temp{\'e}rance	\\
			4	L'Empereur				&	15	Le Diable			\\
			5	Le Pape					&	16	La Maison-Dieu		\\
			6	L'Amoureux				&	17	L'{\'E}toile		\\
			7	Le Chariot				&	18	La Lune				\\
			8	La Justice				&	19	Le Soleil			\\
			9	L'Ermite (L'Hermite)	&	20	Le Jugement			\\
			10	La Roue de Fortune		&	21	Le Monde			\\
		\end{tabular}
	\end{itemize}
\end{minipage}
\begin{minipage}[ht]{0.45\textwidth}
	\begin{itemize}
	\item 56 arcanes mineures, r{\'e}parties en quatres s{\'e}ries ({\'E}p{\'e}es, Coupes, Deniers, B{\^a}tons) comme suit : % ~\newline
		\begin{itemize}
			\item As, 2, 3, 4, 5, 6, 7, 8, 9, 10, Valet, Cavalier, Dame, Roi
			\item As a pour valeur 1 ; Valet, 11 ; Cavalier 12 ; Dame 13 ; Roi 14. 
			\item Dans certains cas Cavalier sup{\'e}rieur au Roi (au lieu d'{\^e}tre entre Valet et Dame). 
		\end{itemize}
		\item[] 
		\item[] Arcanes majeures non associ{\'e}es {\`a} des arcanes mineures : 0;5;10;15;20;21
		\item[] 
		\item[] 
	\end{itemize}
\end{minipage}

	\begin{itemize}
	\item Associations et "significations" des s{\'e}ries, et arcanes majeures associ{\'e}es :~\newline
		\begin{tabular}[c]{ c c c c }
			B{\^a}tons		&	Air		&	Carreaux	&	1;2;3;4		\\
			{\'E}p{\'e}es	&	Feu		&	Tr{\^e}fles	&	6;7;8;9		\\
			Coupes			&	Eau		&	Coeurs		&	11;12;13;14	\\
		 	Deniers			&	Terre	&	Piques		&	16;17;18;19	\\
		 \end{tabular}
	\end{itemize}

\subsection{Tarot du JdR Mage}

\begin{minipage}[ht]{13cm}
	{ \setlength\parindent{0pt} \small
	\begin{tabular}[c]{ p{6.5cm} p{6.5cm} }
			0.	Le Fou - Les Possibilit{\'e}s				& 11.	La Justice - La Justice			\\
			1.	Le Mage - La Volont{\'e}					& 12. Le Pendu - Les Perspectives		\\
			2.	La Grande Pr{\'e}tresse - L'Illumination	& 13. La Mort - Le Renouveau			\\
			3.	L'Imp{\'e}ratrice - La Fertilit{\'e}		& 14. La Temp{\'e}rance - L'Adaptation	\\
			4.	L'Empereur - Le Gouvernement				& 15. Le Diable - Les Entraves			\\
			5.	L'Hi{\'e}rophante - La Morale				& 16. La Tour - La Purification			\\
			6.	Les Amoureux - L'Attirance					& 17. L'{\'E}toile - L'Inspiration		\\
			7.	Le Chariot - Le Triomphe					& 18. Luna - L'Intuition				\\
			8.	La Force - La Puissance						& 19. Le Soleil - La Lib{\'e}ration		\\
			9.	L'Ermite - Le Conseil						& 20. Le Jugement - La R{\'e}union		\\
			10.	La Roue de Fortune - Le Destin				& 21. Gaia - L'Accomplissement			\\
	\end{tabular} }~\\
\end{minipage}
\begin{minipage}[ht]{5.0cm}
	\footnotesize
	Les quatres familles : 
		\begin{itemize}
			\item Questivisme - Traditions - Feu - B{\^a}tons
			\item Primordialisme - Nephandi - Eau - Coupes
			\item Dynamisme - Maraudeurs - Air - {\'E}p{\'e}es
			\item Motivalisme - Technocratie - Terre - Deniers (Pentacles)
		\end{itemize}~\\
\end{minipage}
	
\subsection{Tarot "{\'E}gyptien" (non Etteila)}

{ \setlength\parindent{0pt} \small
\begin{tabular}[c]{ p{6.5cm} p{6.5cm} }	
		0.	Le Fou - Le Ch{\^a}timent			& 11.	La Force - La Puissance			\\
		1.	Le Bateleur - La Volont{\'e}		& 12. Le Pendu - L'Expiation			\\
		2.	La Papesse - La Science				& 13. La Mort - La Transformation		\\
		3.	L'Imp{\'e}ratrice - L'Action		& 14. La Temp{\'e}rance - L'initiative	\\
		4.	L'Empereur - La R{\'e}alisation		& 15. Le Diable - La Fatalit{\'e}		\\
		5.	Le Pape - L'Inspiration				& 16. La Pyramide - La Ruine			\\
		6.	Les Amoureux - L'{\'E}preuve		& 17. L'{\'E}toile - L'Espoir			\\
		7.	Le Chariot - La Victoire			& 18. La Lune - La D{\'e}ception		\\
		8.	La Justice - L'{\'E}quilibre		& 19. Le Soleil - Le Bonheur			\\
		9.	L'Ermite - La Prudence				& 20. Le Jugement - Le Renouvellement	\\
		10.	La Roue de Fortune - Le Destin		& 21. Le Monde - La R{\'e}compense		\\
\end{tabular} }~\\

%% \subsection{Autres Jeux de tarot}



%% utilisation packages latex tikz / tikzpeople pour figures et dessins de fonds
%% tikz-3dplot ?
%% tikzsymbols ?
%% tikz-network ?
%% ... 
%% find /usr/share/ -name "tikz*"
%% find /usr/share/tex* -name "*card*"
%% find /usr/share/tex* -name "*game*"
%% find /usr/share/tex* -name "*chess*"
%% %% /usr/share/texlive/texmf-dist/tex/generic/pgf/frontendlayer/tikz/libraries/tikzlibraryshapes.symbols.code.tex
%% chercher dans : 
%% %% /usr/share/doc/texlive-doc/latex
%% %% /usr/share/texlive/texmf-dist/makeindex/latex
%% %% /usr/share/texlive/texmf-dist/source/latex
%% %% /usr/share/texlive/texmf-dist/tex4ht/ht-fonts/alias/latex
%% %% /usr/share/texlive/texmf-dist/tex4ht/ht-fonts/unicode/latex
%% %% /usr/share/texlive/texmf-dist/tex/latex
%% %% /usr/share/texmf/doc/latex
%% %% /usr/share/texmf/tex/latex

\clearpage

%% \section{Section}
%% \subsection{Sous-section}
%% \subsubsection{sous sous sous section}
%% [...]~\\
%% \subsubsection{sous sous sous section}
%% \subsection*{SousSectionNonNumerot}
%% \addcontentsline{toc}{section}{SousSectionNonNumerot}
%% \begin{table}[ht]
	%% \begin{center}
		%% \begin{tabular}{|p{0.1\textwidth}|p{0.7\textwidth}|}
		%% \hline
		%% \includegraphics[width=1cm]{img/logo_glider.png}
		%% & 
		%% \textbf{GLIDER} est le logo des hacker, symbole repris du jeu de la vie (cavalier). \\
		%% \hline
		%% \end{tabular}
	%% \end{center}
	%% \caption{Un tableau r{\'e}f{\'e}renc{\'e}}
	%% \label{tab:TabReference01}
%% \end{table}~\\
%% \clearpage
%% \subsection{Encore une sous-section}
%% \begin{figure}[H]
	%% \centerline {\epsfig {file=img/logo_glider.png,width=0.5\textwidth}}
	%% \caption{Une belle image}
	%% \label{fig:FigReference01}
%% \end{figure}
%% \clearpage

\section{Annexes}

\subsection{Glossaire}

%% voir en haut : sous la definition du titre : glossaire

\makeglossaire

\clearpage

\subsection{Jeu de r{\^o}le g{\'e}n{\'e}rique de Philippe Tromeur : Arcanes}

Source : \texttt{http://philippe.tromeur.free.fr/53/arcanes.pdf}, publi{\'e} le mardi 10 juin 2008.

\subsubsection{Le mat{\'e}riel et les personnages}

Ce jeu de r{\^o}le se joue avec un jeu de tarots normal, mais vous pouvez {\'e}galement utiliser un jeu de tarot divinatoire, si \c{c}a vous amuse.
Durant la partie, chaque joueur a une main de cinq cartes, renouvell{\'e}es au fur et {\`a} mesure du jeu. Les autres cartes sont dispos{\'e}es dans une pioche tourn{\'e}e contre la table.
Papier, crayons et victuailles sont utiles...~\\

Les personnages sont d{\'e}finis par 5 Couleurs not{\'e}es A, B, C, D et E : Arcane, B{\^a}tons, Coupes, Deniers et {\'E}p{\'e}es.
Le joueur peut r{\'e}partir 22 points entre ses 5 caract{\'e}ristiques, sachant que l'Arcane co{\^u}te deux fois plus cher que les autres Couleurs. 
Pour chaque Couleur, choisir une sp{\'e}cialit{\'e}. ~\\

\begin{minipage}[ht]{0.40\textwidth}

	\subsubsection{Tirage de cartes}
	
	Pour effectuer une action, utilisez une carte de votre main ou tirez-en une de la pioche.
	Chaque action est associ{\'e}e {\`a} une Couleur :
	\begin{itemize}
		\item[Arcanes] (ou Atouts) pour la magie
		\item[B{\^a}tons] (ou Carreau) pour la sant{\'e}
		\item[Coupes] (ou Coeurs) pour le charme
		\item[Deniers] (ou Tr{\`e}fles) pour la fortune
		\item[Ep{\'e}es] (ou Pique) pour l'action
	\end{itemize}

\end{minipage} \hfill \begin{minipage}[ht]{0.54\textwidth}

	\subsubsection{Personnages joueurs (PJ) et Personnages non joueurs (PNJ)}
	
	Le R{\'e}sultat est la valeur de la carte pos{\'e}e, ajout{\'e}e au score du personnage dans la Couleur de l'action. Si la carte est de la Couleur de l'action, le score du personnage compte double.
	L'Excuse / le Mat compte pour 0 ; les Valets 11, les Cavaliers 12, les Dames 13, les Rois 14. 
	Par exemple : un personnage chante, avec un score de 5 en Coeur. Il pose le Valet de Coeur : le R{\'e}sultat est donc de (5x2)+11=21

\end{minipage}~\\

Le ma{\^i}tre du jeu a une seule main utilis{\'e}e pour l'ensemble des personnages non joueurs.

\subsubsection{Combat}

Un combat consiste en des oppositions en Ep{\'e}e : le vainqueur a bless{\'e} l'adversaire. Ce dernier doit faire une opposition contre une difficult{\'e} d{\'e}pendant de l'arme (et souvent du score de B{\^a}tons de l'attaquant) :~\\
\begin{tabular}[c]{c c c c}
	Poings		&	2xB{\^a}tons	&	Pistolet		&	20	\\
	Couteau		&	5+B{\^a}tons	&	Fusil			&	30	\\
	Ep{\'e}e	&	10+B{\^a}tons	&	Lance-Pierre	& 	5	\\
\end{tabular}~\\
En cas d'{\'e}chec (r{\'e}sultat inf{\'e}rieur {\`a} la difficult{\'e}), c'est une blessure grave, sinon c'est une blessure l{\'e}g{\`e}re. En cas d'{\'e}chec critique (r{\'e}sul tat inf{\'e}rieur {\`a} la moiti{\'e} de la difficult{\'e}), c'est une blessure critique (et le coma).
Avec une blessure l{\'e}g{\`e}re, on n'a plus que 3 cartes en main. Avec une blessure grave, on n'en a plus qu'une en main. On r{\'e}cup{\`e}re un niveau de blessure par jour de soins appropri{\'e}s.

\subsubsection{Sp{\'e}cialit{\'e}s et Magie, Exp{\'e}rience, Aventure... }

Les sp{\'e}cialit{\'e}s autorisent les personnages {\`a} rajouter +5 {\`a} leur R{\'e}sultat, si jamais l'action est li{\'e}e au champ de cette sp{\'e}cialit{\'e}. Pour utiliser la magie, il faut poss{\'e}der la sp{\'e}cialit{\'e} ad{\'e}quate :
\begin{itemize}
	\item[] n{\'e}cromancie pour discuter avec des fant{\^o}mes,
	\item[] d{\'e}monisme pour appeler des d{\'e}mons,
	\item[] divination pour savoir des choses,
	\item[] t{\'e}l{\'e}kin{\'e}sie pour d{\'e}placer des objets...
\end{itemize}
La difficult{\'e} d'Arcane, incluant le bonus de +5 pour la sp{\'e}cialit{\'e}, est g{\'e}n{\'e}ralement {\'e}lev{\'e}e, de 20 pour une divination mineure, {\`a} 40 pour la destruction d'un immeuble par la foudre.
Plusieurs tentatives successives sont possibles, mais en cas d'{\'e}chec critique (R{\'e}sultat inf{\'e}rieur {\`a} la moiti{\'e} de la difficult{\'e}), le r{\'e}sultat est catastrophique, par exemple une blessure.
En prenant son temps (une heure par jet d'Arcane), on {\'e}vite le risque d'{\'e}chec critique.~\\

Apr{\`e}s une aventure, un personnage gagne de 1 {\`a} 5 points d'exp{\'e}rience, qui se d{\'e}pensent pour acheter de nouveaux points dans les 5 Couleurs (l'Arcane co{\^u}te toujours deux fois plus cher). Une sp{\'e}cialit{\'e} co{\^u}te 3 points, sauf les sp{\'e}cialit{\'e}s Arcanes qui co{\^u}tent 5 points. 
--- Oup's, nous arrivons d{\'e}j{\`a} en fin de page et je n'ai pas l'espace de vous pr{\'e}senter un univers et un exemple de sc{\'e}nario... D{\'e}brouillez-vous, je vous fais confiance !

\clearpage

\subsection{Autres JdR / Jeux de R{\^o}le utilisant un Tarot}

\begin{itemize}
	\item \emph{Les Lames du Cardinal} (Tarot des Ombres) ; 
	\item Tarot d'inspiration pour \emph{R{\^e}ves de Dragon}, \emph{Mal{\'e}fices}
	\item \emph{Ambre}, \emph{Mage}...
	\item ...
	\item Cas {\`a} part et m{\'e}ritant d'{\^e}tre cit{\'e} ici : \emph{Ch{\^a}teau FalkenStein}, o{\`u} plusieurs jeux de 54 cartes sont utilis{\'e}s {\`a} la place des d{\'e}s en cours de partie afin de r{\'e}soudre les actions des personnages de joueurs.
\end{itemize}

\subsection{Bibliographie\markboth{Bibliographie}{Bibliographie}}

\addcontentsline{toc}{section}{Bibliographie}
\nocite{*}
%toutes references biblio : 6 lettres + 2 chiffres
\bibliography{arcanesJdR}
\bibliographystyle{frplain} % plain or frplain
\end{document}
